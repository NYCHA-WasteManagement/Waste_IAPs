
    \resizebox{\textwidth}{!}{
    \begin{tabular}{l|c|c|c|c|}
    \cline{2-5}
                                                                           & \cellcolor{ccteal}{\color[HTML]{FFFFFF} TDS \#} & \cellcolor{ccteal}{\color[HTML]{FFFFFF} Total Households} & \cellcolor{ccteal}{\color[HTML]{FFFFFF} Official Population} & \cellcolor{ccteal}{\color[HTML]{FFFFFF} Average Family Size} \\ \hline

    \multicolumn{1}{|l|}{\cellcolor{ccteallight}Boynton Avenue Rehab}        & 346                                                   & 82                                                           & 182                                                                & 2.2                                                                \\ \hline\multicolumn{1}{|l|}{\cellcolor{ccteallight}Bronx River}        & 032                                                   & 1,225                                                           & 2,919                                                                & 2.4                                                                \\ \hline\multicolumn{1}{|l|}{\cellcolor{ccteallight}Bronx River Addition}        & 157                                                   & 217                                                           & 232                                                                & 1.1                                                                \\ \hline
    \end{tabular}
    }
    