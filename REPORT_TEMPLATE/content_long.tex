% ----------------------------------------------	
\textbf{\huge Glossary}\\
\textit{Italicized words indicate a term that will be described later in the glossary}

\textbf{Bulk Waste Container --} A 30-cubic-yard bin, typically uncovered, used to hold non-recyclable bulk waste such as furniture, wood, etc.

\textbf{Cardboard baler --} A machine used to automatically compress loose cardboard into bundles as an alternative to manually breaking down boxes and tying or bagging them for curbside collection

\textbf{Compactor Bags --} 40-lb bags of compacted trash from \textit{interior compactors}

\textbf{Consolidation --} Name given to one or many of developments that are managed by the same location or management office and are assigned a unique 3-digit numeric ID in the Tenant Data System (TDS), e.g., the Sumner Consolidation TDS\# 073

\textbf{Containerization --} Storage of waste that is pest-resistant

\textbf{Development --} Individual NYCHA properties that are assigned an individual development TDS number, e.g., 303 Vernon Ave TDS\# 156

\textbf{Drop site --} Also known as secondary waste areas, these are designated areas where waste may be placed by residents for collection and disposal by staff; the site may accommodate both trash and recycling bins. Central office also uses the term ``staff drop sites'' to describe areas where waste is placed before being brought to the \textit{waste yard}

\textbf{DSNY --} City of New York Department of Sanitation

\textbf{E-waste --} Electronics such as TVs and computers that must be discarded through the manufacturer, a recycling location, or one of DSNY's special recycling programs

\textbf{Exterior Compactor --} Often referred to on our developments as an \textit{EZ-pack}. Similar to an \textit{interior compactor}, this machine compacts and containerizes waste into 30-cubic-yard containers before removal by DSNY

\textbf{EZ-Pack --} Another term used at the developments for exterior compactor. The term is also used by DSNY and central office to describe waste containers of various sizes that are designed to be dumped directly into a DSNY truck

\textbf{Hopper Doors --} Doors to trash chutes, traditional hopper doors' areas are $1/3$ the chute's area. Traditional hopper doors comfortably can accommodate a 13-gallon trash bag. Enlarged hopper doors are $2/3$ the chute's area and can comfortably accommodate a 30-gallon trash bag

\textbf{Interior Compactor --} A machine at the base of a \textit{trash chute} that uses a ram to compress waste material and reduce its total volume; mostly located in the basement of developments

\textbf{Mattress Containers --} Locked shipping containers serviced under a mattress recycling contract where staff at participating developments bring mattresses

\textbf{NYCHA 2.0 Waste Management Plan --} NYCHA's comprehensive plan created in 2019 designed to make NYCHA buildings and grounds visibly clean and free of pests

\textbf{Paragraph 45 --} Part of the agreement between HUD, SDNY, and NYCHA pertaining to waste management through inspection, collection, and containerization. The text of the paragraph is as follows:
\begin{quotation}
\small\textit{Within six months of the Effective Date, NYCHA shall, no less than once every 24 hours, inspect the grounds and commons areas of each building for cleaning and maintenance needs, including pests and trash, and correct such conditions. In particular, NYCHA shall ensure that trash on the grounds or common areas of each NYCHA building is collected and either removed from the premises or stored in a manner that prevents access from pests at least once every 24 hours.}
\end{quotation}

\textbf{Recyclable --} All material that is separated and collected for recycling

\textbf{Textiles --} Unwanted clothing, towels, blankets, curtains, shoes, handbags, belts, and other textiles and apparel that can be collected for re-use or recycling

\textbf{Trash --} All material not separated for recycling that will be transported to landfills or incinerators for disposal

\textbf{Trash Chute --} A vertical shaft inside a building used for transferring trash by gravity to the interior compactor at the bottom

\textbf{Types of DSNY Disposal:}
\begin{itemize}
\setlength{\itemsep}{0pt}%
    \setlength{\parsep}{0pt}%
    \setlength{\parskip}{0pt}%
\item \textbf{Curbside --} Material is moved from building compactors and grounds within the development by staff to a secondary storage area until it is placed at DSNY collection locations on sidewalks adjacent to or along the perimeter of the development
\item \textbf{Shared --} Material is moved (aka shared) from one development without assets to ensure containerization to another development that has an exterior compactor
\item \textbf{Exterior Compactor Containerization --} Material is moved from building compactors and grounds within the development and potentially from shared developments to a waste storage area that contains an exterior compactor
\end{itemize}

\textbf{Waste --} All discarded material including both trash and recyclables

\textbf{Waste Yard --} Centralized waste facility for containerized collection including equipment such as exterior compactors, bulk crushers and bulk waste containers. Waste yards may also include storage and equipment for recyclables


\pagecolor{white}

\whitetext{\chapter{\textcolor{ccteal}{Introduction}}}
\whitetext{\textbf{\Huge Introduction}}
\pagecolor{ccteal}
\fontfamily{phv}
\pagestyle{plain}

\definecolor{insetbox_teal}{RGB}{230, 239, 239}
\colorbox{insetbox_teal}{
\begin{minipage}[b]{\textwidth}

\begin{table}[H]
\small

\input{\rootpath/TABLES/overview_table/\tds_overview_table}
\end{table}

\begin{centering}
\input{\rootpath/TABLES/typology_table/\tds_typology}
\end{centering}
\end{minipage}
}

%\vspace*{\stretch{.5}}
\pagebreak

\pagestyle{fancy}
\fancyhf{}
\renewcommand{\chaptermark}[1]{\markboth{#1}{}}
\fancyfoot[LE,RO]{\sffamily\thepage}
\fancyhead[CE,CO]{\textit{\footnotesize\sffamily Data current as of August 2020; verify before using}}
	\textcolor{ccteal}{\section{Overview}}
	\input{\rootpath/TEXT/overview_text/\tds_overview}
\pagecolor{white}

\afterpage{
	\ifodd\value{page}
	\else
	\blankpage
	\pagebreak
	\fi}
	\pagebreak
	\afterpage{%
    %\clearpage% flush all other floats
    \ifodd\value{page}
    \else% uncomment this else to get odd/even instead of even/odd
        \expandafter\afterpage% put it on the next page if this one is odd
    \fi
    {%
    \newgeometry{left=0in, right=0in, top=0in, bottom=0in}
	\fakesection{Context Map}
    \begin{figure}[H]
    	\raggedleft
        \includegraphics[height=11in]{\rootpath/MAPS/context_maps/\tds_context_1.png}%
    \end{figure}
    \clearpage
    \begin{figure}[H]
    	\raggedright
        \includegraphics[height=11in]{\rootpath/MAPS/context_maps/\tds_context_2.png}%
    \end{figure}
    \clearpage
    }%
}
\pagecolor{white}
%-------------------------------------------
\whitetext{\chapter{\textcolor{ccorange}{Waste Services and Assets}}}
\whitetext{\textbf{\Huge Waste Services and Assets}}
\pagecolor{ccorange}
\pagestyle{plain}
\restoregeometry
\pagebreak
\pagestyle{fancy}
\fancyhf{}
\renewcommand{\chaptermark}[1]{\markboth{#1}{}}
\fancyfoot[LE,RO]{\sffamily\thepage}
\fancyhead[CE,CO]{\textit{\footnotesize\sffamily Data current as of August 2020; verify before using}}
\pagecolor{white}

%\restoregeometry
%\KOMAoptions{paper=letter, paper=portrait}
%\recalctypearea
\textcolor{ccorange}{\section{Waste Services and Assets}}
%%% TO-DO: AUTOMATE TEXT GENERATION %%%
At sites where household waste is not picked up curbside, caretakers are responsible for transporting waste from internal compactor rooms and secondary collection sites to external compactors, either at the development in question or another development within the consolidation. Caretakers also transport bulk waste from sites where residents deposit it to centralized holding areas (which house one or more 30-yard bulk containers) at each development; waste is then periodically transferred to a central holding location in each consolidation for pickup by a private carter. Recyclables are typically collected in receptacles around each site; caretakers then empty these receptacles and transport recyclables to curbside sites for pickup by DSNY.
\begin{table}[H]
\small
%%% TO-DO: AUTOMATE WASTE SERVICES AND ASSETS TABLE %%%
%%%%%%%%%%%%%%%%%%%%%%%%%%%%%%%%%%%%%%%%%%%%%%%%%%%%%%%%%%%%%%%%%%%%%
\small
\input{\rootpath/TABLES/waste_services/\tds_waste_services_1.tex}
\bigskip
\input{\rootpath/TABLES/waste_services/\tds_waste_services_2.tex}
\end{table}
\pagebreak

\textcolor{ccorange}{WASTE ASSET MAP}
\begin{figure}[H]
\raggedright
\includegraphics[width=.95\textwidth]{\rootpath/MAPS/asset_maps/\tds_asset_map.png}
\end{figure}
\pagebreak

\textcolor{ccorange}{WASTE ASSETS}

\begin{table}[H]
\begin{threeparttable}
\small

\input{\rootpath/TABLES/waste_assets/\tds_waste_assets}

\begin{tablenotes}
\item [1] Recycling bin data may be incomplete; consult with development staff before using.
\end{tablenotes}
\end{threeparttable}
\end{table}

\textcolor{ccorange}{CONSOLIDATION ASSETS}\\
Vehicles and horticultural equipment are crucial to waste management at NYCHA properties. Skid-steer loaders are used to manipulate waste receptacles, such as EZ-Packs, while trucks are used to transport bagged garbage and recyclables both within and between developments to its proper storage and pick-up locations. Tractors are used to collect debris on development grounds, while sweepers simplify routine cleaning.
\begin{table}[H]
\input{\rootpath/TABLES/consolidation_assets/\tds_consolidation_assets}
\end{table}
\pagebreak

\parbox[T][3in][c]{\textwidth}{
\textcolor{ccorange}{\section{Estimated Waste Volumes}}
%%%% TO-DO: AUTOMATE TEXT GENERATION %%%
\input{\rootpath/TEXT/waste_distribution_top/\tds_wd_top.tex}
}

\begin{table}[H]
\begin{threeparttable}
\small

\input{\rootpath/TABLES/waste_distribution_table/\tds_wd_table_1}

\end{threeparttable}
\end{table}
\pagebreak
%\newgeometry{left=0in, right=.5in, top=.5in, bottom=.5in}
%\parbox[T][3in][c]{\textwidth}{
%\input{\rootpath/TEXT/waste_distribution_bottom/\tds_wd_bottom.tex}}
\begin{table}[H]
\begin{threeparttable}
\small

\input{\rootpath/TABLES/waste_distribution_table/\tds_wd_table_2}

\begin{tablenotes}
\item [1] Assumes 5lbs of waste is produced daily in each unit.
\item [2] Includes miscellaneous garbage as well as uncaptured recyclables, organics, e-waste, and textiles.
\item [3] Primary method of trash collection, via chute. Assumes a 75\% capture rate.
\item [4] Secondary method of trash collection. Assumes a 25\% capture rate
\item [5] Number of drop sites estimated to equal number of buildings.
\item [6] Assumes capture rates of recyclables at NYCHA portfolio-wide of 30\% for MGP, 50\% for Cardboard, and 20\% for Paper. 
%\item[5] Organics, e-waste, and textiles have a capture rate of 0\%.
\end{tablenotes}
\end{threeparttable}
\end{table}


\pagebreak

\textcolor{ccorange}{\section{Capital Improvements}}

The following data on capital improvements describe the status of four major programs underway across NYCHA. However, they are not exhaustive: other improvements and initiatives with bearing on waste management may be underway at particular developments.
\begin{table}[H]
%\resizebox{\textwidth}{\textheight}{
\small
\begin{tabular}{l}
\input{\rootpath/TABLES/capital_projects_table/\tds_capital_projects_1}\\
\input{\rootpath/TABLES/capital_projects_table/\tds_capital_projects_2}
\end{tabular}
%}
\end{table}
\pagebreak

\textcolor{ccorange}{PRIORITIES}
\\\textbf{NOTE: THIS SECTION IS CURRENTLY UNDER REVISION, PENDING FUTHER FEEDBACK AND INITIAL SITE VISITS}
\input{\rootpath/WORK_ORDER_ANALYSIS/image_layouts/\tds_layout}

%-------------------------------------------

\pagestyle{plain}
\whitetext{\Chapter{\textcolor{ccfuschia}{Staffing}}}
\whitetext{\textbf{\Huge Staffing}}
\pagecolor{ccfuschia}
\pagestyle{plain}

\pagebreak
\newpagecolor{white}
\pagestyle{fancy}
\fancyhf{}
\renewcommand{\chaptermark}[1]{\markboth{#1}{}}
\fancyfoot[LE,RO]{\sffamily\thepage}
\fancyhead[CE,CO]{\textit{\footnotesize\sffamily Data current as of August 2020; verify before using}}


\textcolor{ccfuschia}{STAFFING STRUCTURE}
\begin{figure}[H]
	\resizebox{\textwidth}{!}{
	\centering
	\begin{tikzpicture}[node distance=3cm]
	\node (vpes) [process, xshift=-6cm] {VP for Energy and Sustainability};
	\node (evp) [process, above of=vpes] {EVP for Capital Projects};
	\node (dires) [process, below of=vpes] {Director of Sustainability Programs};
	\node (vp) [process] {VP of Operations};
	\node (vpwm) [process, xshift=6cm] {VP of Waste Management and Pest Control};
	\node (dirwm) [process, below of=vpwm] {Director of Waste Management};
	\node (borodr) [process, below of=vp] {Borough Director};
	\node(ram) [processwide, below of=borodr] {Regional Asset Manager};
	\node (pm) [processwide, below of=ram] {Property Manager};
	\node (super) [process, below of=pm, xshift=-5cm] {Superintendent};
	\node (asuper) [process, below of=super] {Assistant Superintendent};
		\node (mtn) [process, below of=asuper, xshift=-3.5cm] {Maintenance Workers};
		\node (spc) [process, below of=asuper] {Supervisor of Caretakers};
			\node (crt) [process, below of=spc] {Caretakers\\(X and J)};
		\node (spg) [process, below of=asuper, xshift=3.5cm] {Supervisor of Grounds};
			\node (crtg) [process, below of = spg] {Caretakers (G)};			
		
	\node (apm)[process, below of=pm, xshift=5cm] {Assistant Property Manager};
		\node (sec) [process, below of=apm, xshift = -2.5cm] {Secretaries};
		\node (asst) [process, below of=apm, xshift = 2.5cm] {Housing Assistants};

	%\node (sub1) [subprocess, below of=pro1] {\nodepart{two} Subprogram};
	%\node (dec1) [decision, below of=sub1, yshift=-1cm] {Decision};
	%\node (com1) [comment, below of=dec1, xshift=-4cm, yshift=-1cm] {STEP 2};
	%\node (stop) [startstop, below of=dec1, yshift=-1cm] {Stop};

	%\draw [arrow] (dec1.west) -- ++(-1,0) node[anchor=south,pos=0.5] {No} |- (sub1.west);
	%\draw [arrow] (dec1) -- node[anchor=west] {Yes} (stop);

	\draw [dotline] (vp) -- (vpwm);
	\draw [dotline] (vp) -- (vpes);
	\draw [arrow] (evp) -- (vpes);
	\draw [arrow] (vpes) -- (dires);
	\draw [arrow] (vpwm) -- (dirwm);
	\draw [arrow] (vp) -- (borodr);
	\draw [arrow] (borodr) -- (ram);
	\draw [arrow] (ram) -- (pm);
	\draw [arrow] (pm) -- (super);
	\draw [arrow] (super) -- (asuper);
		\draw [arrow] (asuper) -- (mtn);
		\draw [arrow] (asuper) -- (spc);
			\draw [arrow] (spc) -- (crt);
		\draw [arrow] (asuper) -- (spg);
			\draw [arrow] (spg) -- (crtg);
	
	\draw [arrow] (pm) -- (apm);
		\draw [arrow] (apm) -- (sec);
		\draw [arrow] (apm) -- (asst);
	\end{tikzpicture}
	}
\end{figure}

\pagebreak
\textcolor{ccfuschia}{ALLOCATED STAFF}
\\\bigskip At the consolidation level, responsibility for waste management falls on caretakers as well as those who manage them -- Supervisor of Caretakers (SOC), and Supervisor of Groundskeepers (SOG). The duties of specific caretaker roles are outlined below:
\begin{itemize}[noitemsep]
\item Caretaker X: Authorized to drive vehicles necessary for large-scale movement of waste, such as skid-steer loaders used to manipulate 30-yard containers. These caretakers may also conduct a range of duties otherwise assigned to caretakers in the G or J titles. 
\item Caretaker G: Primarily responsible for groundskeeping tasks, such as cutting lawns, trimming trees and hedges, and tending to beds. 
\item Caretaker J: Conduct a range of janitorial tasks, including removing garbage from compactor rooms, servicing equipment such as compactors, and cleaning indoor and outdoor spaces of debris. These caretakers may also conduct groundskeeping work, including cutting lawns and trimming hedges.
\end{itemize}
Moving forward, the Department of Waste Management and Pest Control will oversee NYCHA's progress in these areas, manage inspections to assess development cleanliness, and develop new initiatives.
\begin{table}[H]
\begin{threeparttable}

\input{\rootpath/TABLES/staff_table/\tds_staff_table}

\begin{tablenotes}
\small
\item [1] Initial staff allocation recommendations are generated by formula, with the number of employees per consolidation and development determined in large part by number of residents at the time of calculation
\item [2] Includes staff in roles Caretaker J, Caretaker I, and Chief Caretaker
\end{tablenotes}
\end{threeparttable}
\end{table}
\pagebreak
%-------------------------------------------
\pagestyle{plain}
\whitetext{\Chapter{\textcolor{ccgreen}{Analysis}}}
\whitetext{\textbf{\Huge Analysis}}
\pagecolor{ccgreen}
\pagestyle{plain}

\pagebreak
\newpagecolor{white}
\pagestyle{fancy}
\fancyhf{}
\renewcommand{\chaptermark}[1]{\markboth{#1}{}}
\fancyfoot[LE,RO]{\sffamily\thepage}
\fancyhead[CE,CO]{\textit{\footnotesize\sffamily Data current as of August 2020; verify before using}}
\textcolor{ccgreen}{ANALYSIS OF FINDINGS}
\\
\input{\rootpath/TEXT/analysis_text/\tds_analysis}

% ############################################
\pagestyle{plain}
\whitetext{\Chapter{\textcolor{lightBlue}{Appendices}}}
\whitetext{\textbf{\Huge Appendices}}
\pagecolor{lightBlue}
\pagestyle{plain}

\pagebreak
\pagestyle{fancy}
\fancyhf{}
\renewcommand{\chaptermark}[1]{\markboth{#1}{}}
\fancyfoot[LE,RO]{\thepage}
\nopagecolor
\IfFileExists{\rootpath/APPENDICES/site_plans/\tds.pdf}{\includepdf[pages={1-}, picturecommand*={\put(30,750){\textcolor{lightBlue}{Appendix I -- Site Plans}}}]{\rootpath/APPENDICES/site_plans/\tds.pdf}}{}

% ############################################

% ############################################
\begin{comment}

\section{Lists}

Unordered Lists:
\begin{itemize}
\item This is an unordered list. 
\item Item 2.
\item It has three items.
\end{itemize}

Ordered List:
\begin{enumerate}
\item This is an ordered list.
\item Item 2.
\item It has three items.
\end{enumerate}

Ordered List (alphabetical):
\begin{enumerate}[label=\Alph*.]
\item This is an ordered list.
\item Item 2.
\item It has three items.
\end{enumerate}

% ----------------------------------------------
\cleardoublepage
\chapter{Figures}\label{ch:figures}

% ############################################
\section{Images}\label{sec:images}

\autoref{fig:image1} shows how to display images.

\begin{figure}[H]
	\centering
	\includegraphics[width=0.5\textwidth]{example-image-a.pdf}
	\caption{Image}

	\label{fig:image1}
\end{figure}

\begin{figure}[H]
	\centering
	\includegraphics[width=0.5\textwidth]{example-image-b.pdf}
	\caption{Image with Source}
	\captionsource{\cite{mus:16}}	
	\label{fig:image2}
\end{figure}

\begin{figure}[H]
	\centering
	\includegraphics[width=0.5\textwidth]{example-image-c.pdf}
	\caption{Image with Source and Link}
	\captionsource[https://example.org]{J. Doe}	
	\label{fig:image3}
\end{figure}

% ############################################

% ----------------------------------------------
\cleardoublepage
\chapter{Conclusion}\label{ch:conclusion}

\end{comment}

