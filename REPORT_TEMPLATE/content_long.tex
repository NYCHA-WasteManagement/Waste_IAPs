% ----------------------------------------------	
\pagestyle{plain}
\whitetext{\chapter{\textcolor{ccteal}{Introduction}}}
\whitetext{\textbf{\Huge Introduction}}
\pagecolor{ccteal}
\fontfamily{phv}


\definecolor{insetbox_teal}{RGB}{230, 239, 239}
\colorbox{insetbox_teal}{
\begin{minipage}[b]{\textwidth}

\begin{table}[H]
\small

\input{\rootpath/TABLES/overview_table/\tds_overview_table}
\end{table}

\begin{centering}
\input{\rootpath/TABLES/typology_table/\tds_typology}
\end{centering}
\end{minipage}
}

%\vspace*{\stretch{.5}}
\pagebreak
\newpagecolor{white}
\pagestyle{fancy}
\fancyhf{}
\renewcommand{\chaptermark}[1]{\markboth{#1}{}}
\fancyfoot[LE,RO]{\sffamily\thepage}
\fancyhead[CE,CO]{\textit{\footnotesize\sffamily Data current as of August 2020; verify before using}}

	\begin{minipage}[t][.39\textheight][t]{\textwidth}
	\textcolor{ccteal}{\section{Overview}}
	\input{\rootpath/TEXT/overview_text/\tds_overview}
	
	\end{minipage}


\pagebreak

\newgeometry{left=0in, right=0in, top=0in, bottom=0in}
\fakesection{Context Map}
\afterpage{%
    \clearpage% flush all other floats
    \ifodd\value{page}
    %\else% uncomment this else to get odd/even instead of even/odd
        \expandafter\afterpage% put it on the next page if this one is odd
    \fi
    {%
    \begin{figure}[H]
    	\raggedleft
        \includegraphics[height=11in]{\rootpath/MAPS/context_maps/\tds_context_1.png}%
    \end{figure}
    \clearpage
    \begin{figure}[H]
    	\raggedright
        \includegraphics[height=11in]{\rootpath/MAPS/context_maps/\tds_context_2.png}%
    \end{figure}
    \clearpage
    }%
}
\restoregeometry

%-------------------------------------------
\pagestyle{plain}
\whitetext{\chapter{\textcolor{ccorange}{Waste Services and Assets}}}
\whitetext{\textbf{\Huge Waste Services and Assets}}
\pagecolor{ccorange}

\pagebreak
%\newgeometry{right=0in}
\newpagecolor{white}
\pagestyle{fancy}
\fancyhf{}
\renewcommand{\chaptermark}[1]{\markboth{#1}{}}
\fancyfoot[LE,RO]{\sffamily\thepage}
\fancyhead[CE,CO]{\textit{\footnotesize\sffamily Data current as of August 2020; verify before using}}

%\restoregeometry
%\KOMAoptions{paper=letter, paper=portrait}
%\recalctypearea
\textcolor{ccorange}{\section{Waste Services and Assets}}
%%% TO-DO: AUTOMATE TEXT GENERATION %%%
At sites where household waste is not picked up curbside, caretakers are responsible for transporting waste from internal compactor rooms and secondary collection sites to external compactors, either at the development in question or another development within the consolidation. Bulk is transported from bulk waste drop-off sites at each development (which house one or more 30-yard bulk containers) to a central location in each consolidation for pickup by a private carter. Recyclables are typically collected in receptacles around each site; caretakers then empty these receptacles and transport recyclables to curbside sites for pickup by DSNY.
\begin{table}[H]
\small
%%% TO-DO: AUTOMATE WASTE SERVICES AND ASSETS TABLE %%%
%%%%%%%%%%%%%%%%%%%%%%%%%%%%%%%%%%%%%%%%%%%%%%%%%%%%%%%%%%%%%%%%%%%%%
\small
\input{\rootpath/TABLES/waste_services/\tds_waste_services_1.tex}
\bigskip
\input{\rootpath/TABLES/waste_services/\tds_waste_services_2.tex}
\end{table}
\pagebreak

\textcolor{ccorange}{WASTE ASSET MAP}
\begin{figure}[H]
\raggedright
\includegraphics[width=.95\textwidth]{\rootpath/MAPS/asset_maps/\tds_asset_map.png}
\end{figure}
\pagebreak

\textcolor{ccorange}{WASTE ASSETS}

\begin{table}[H]
\begin{threeparttable}
\small

\input{\rootpath/TABLES/waste_assets/\tds_waste_assets}

\begin{tablenotes}
\item [1] Recycling bin data may be incomplete; consult with development staff before using.
\end{tablenotes}
\end{threeparttable}
\end{table}

\textcolor{ccorange}{SUMNER CONSOLIDATION ASSETS}
\begin{table}[H]
\input{\rootpath/TABLES/consolidation_assets/\tds_consolidation_assets}
\end{table}
\pagebreak

\parbox[T][3in][c]{\textwidth}{
\textcolor{ccorange}{\section{Waste Distribution}}
%%%% TO-DO: AUTOMATE TEXT GENERATION %%%
\input{\rootpath/TEXT/waste_distribution_top/\tds_wd_top.tex}
}

\begin{table}[H]
\begin{threeparttable}
\small

\input{\rootpath/TABLES/waste_distribution_table/\tds_wd_table_1}

\end{threeparttable}
\end{table}
\pagebreak
%\newgeometry{left=0in, right=.5in, top=.5in, bottom=.5in}
%\parbox[T][3in][c]{\textwidth}{
%\input{\rootpath/TEXT/waste_distribution_bottom/\tds_wd_bottom.tex}}
\begin{table}[H]
\begin{threeparttable}
\small

\input{\rootpath/TABLES/waste_distribution_table/\tds_wd_table_2}

\begin{tablenotes}
\item [1] Assumes 5lbs of waste is produced daily in each unit.
\item [2] Includes miscellaneous garbage as well as uncaptured recyclables, organics, e-waste, and textiles.
\item [3] Primary method of trash collection, via chute. Assumes a 75\% capture rate.
\item [4] Secondary method of trash collection. Assumes a 25\% capture rate
\item [5] Capture rates of recyclables at NYCHA portfolio-wide: 30\% of MGP, 50\% of Cardboard, and 20\% of Paper. 
%\item[5] Organics, e-waste, and textiles have a capture rate of 0\%.
\end{tablenotes}
\end{threeparttable}
\end{table}


\pagebreak

\textcolor{ccorange}{\section{Capital Improvements}}


\begin{table}[H]
%\resizebox{\textwidth}{\textheight}{
\small
\begin{tabular}{l}
\input{\rootpath/TABLES/capital_projects_table/\tds_capital_projects_1}\\
\input{\rootpath/TABLES/capital_projects_table/\tds_capital_projects_2}
\end{tabular}
%}
\end{table}
\pagebreak

\textcolor{ccorange}{PRIORITIES}
\textbf{NOTE: THIS SECTION IS CURRENTLY UNDER REVISION, PENDING FUTHER FEEDBACK AND INITIAL SITE VISITS}
\input{\rootpath/WORK_ORDER_ANALYSIS/image_layouts/\tds_layout}

%-------------------------------------------
\pagestyle{plain}
\whitetext{\Chapter{\textcolor{ccfuschia}{Staffing}}}
\whitetext{\textbf{\Huge Staffing}}
\pagecolor{ccfuschia}

\pagebreak
\newpagecolor{white}
\pagestyle{fancy}
\fancyhf{}
\renewcommand{\chaptermark}[1]{\markboth{#1}{}}
\fancyfoot[LE,RO]{\sffamily\thepage}
\fancyhead[CE,CO]{\textit{\footnotesize\sffamily Data current as of August 2020; verify before using}}


\textcolor{ccfuschia}{STAFFING STRUCTURE}
\begin{figure}[H]
	\resizebox{\textwidth}{!}{
	\centering
	\begin{tikzpicture}[node distance=3cm]
	\node (vpes) [process, xshift=-6cm] {VP for Energy and Sustainability};
	\node (evp) [process, above of=vpes] {EVP for Capital Projects};
	\node (dires) [process, below of=vpes] {Director of Sustainability Programs};
	\node (vp) [process] {VP of Operations};
	\node (vpwm) [process, xshift=6cm] {VP of Waste Management and Pest Control};
	\node (dirwm) [process, below of=vpwm] {Director of Waste Management};
	\node (borodr) [process, below of=vp] {Borough Director};
	\node(ram) [processwide, below of=borodr] {Regional Asset Manager};
	\node (pm) [processwide, below of=ram] {Property Manager};
	\node (super) [process, below of=pm, xshift=-5cm] {Superintendent};
	\node (asuper) [process, below of=super] {Assistant Superintendent};
		\node (mtn) [process, below of=asuper, xshift=-3.5cm] {Maintenance Workers};
		\node (spc) [process, below of=asuper] {Supervisor of Caretakers};
			\node (crt) [process, below of=spc] {Caretakers\\(X and J)};
		\node (spg) [process, below of=asuper, xshift=3.5cm] {Supervisor of Grounds};
			\node (crtg) [process, below of = spg] {Caretakers (G)};			
		
	\node (apm)[process, below of=pm, xshift=5cm] {Assistant Property Manager};
		\node (sec) [process, below of=apm, xshift = -2.5cm] {Secretaries};
		\node (asst) [process, below of=apm, xshift = 2.5cm] {Housing Assistants};

	%\node (sub1) [subprocess, below of=pro1] {\nodepart{two} Subprogram};
	%\node (dec1) [decision, below of=sub1, yshift=-1cm] {Decision};
	%\node (com1) [comment, below of=dec1, xshift=-4cm, yshift=-1cm] {STEP 2};
	%\node (stop) [startstop, below of=dec1, yshift=-1cm] {Stop};

	%\draw [arrow] (dec1.west) -- ++(-1,0) node[anchor=south,pos=0.5] {No} |- (sub1.west);
	%\draw [arrow] (dec1) -- node[anchor=west] {Yes} (stop);

	\draw [dotline] (vp) -- (vpwm);
	\draw [dotline] (vp) -- (vpes);
	\draw [arrow] (evp) -- (vpes);
	\draw [arrow] (vpes) -- (dires);
	\draw [arrow] (vpwm) -- (dirwm);
	\draw [arrow] (vp) -- (borodr);
	\draw [arrow] (borodr) -- (ram);
	\draw [arrow] (ram) -- (pm);
	\draw [arrow] (pm) -- (super);
	\draw [arrow] (super) -- (asuper);
		\draw [arrow] (asuper) -- (mtn);
		\draw [arrow] (asuper) -- (spc);
			\draw [arrow] (spc) -- (crt);
		\draw [arrow] (asuper) -- (spg);
			\draw [arrow] (spg) -- (crtg);
	
	\draw [arrow] (pm) -- (apm);
		\draw [arrow] (apm) -- (sec);
		\draw [arrow] (apm) -- (asst);
	\end{tikzpicture}
	}
\end{figure}

\pagebreak
\textcolor{ccfuschia}{ALLOCATED STAFF}
\\\bigskip At the consolidation level, responsibility for waste management falls on caretakers as well as those who manage them -- Supervisor of Caretakers (SOC), and Supervisor of Groundskeepers (SOG). The duties of specific caretaker roles are outlined below:
\begin{itemize}[noitemsep]
\item Caretaker X: Authorized to drive vehicles necessary for large-scale movement of waste, such as skid-steer loaders used to manipulate 30-yard containers. These caretakers may also conduct a range of duties otherwise assigned to caretakers in the G or J titles. 
\item Caretaker G: Primarily responsible for groundskeeping tasks, such as cutting lawns, trimming trees and hedges, and tending to beds. 
\item Caretaker J: Conduct a range of janitorial tasks, including removing garbage from compactor rooms, servicing equipment such as compactors, and cleaning indoor and outdoor spaces of debris. These caretakers may also conduct groundskeeping work, including cutting lawns and trimming hedges.
\end{itemize}
Moving forward, the Department of Waste Management and Pest Control will oversee NYCHA's progress in these areas, manage inspections to assess development cleanliness, and develop new initiatives.
\begin{table}[H]
\begin{threeparttable}

\input{\rootpath/TABLES/staff_table/\tds_staff_table}

\begin{tablenotes}
\small
\item [1] Initial staff allocation recommendations are generated by formula, with the number of employees per consolidation and development determined in large part by number of residents at the time of calculation
\item [2] Includes staff in roles Caretaker J, Caretaker I, and Chief Caretaker
\item [3] Actual staff figures do not differentiate between caretaker roles
\end{tablenotes}
\end{threeparttable}
\end{table}

\pagebreak
%-------------------------------------------
\pagestyle{plain}
\whitetext{\Chapter{\textcolor{ccgreen}{Analysis}}}
\whitetext{\textbf{\Huge Analysis}}
\pagecolor{ccgreen}

\pagebreak
\newpagecolor{white}
\pagestyle{fancy}
\fancyhf{}
\renewcommand{\chaptermark}[1]{\markboth{#1}{}}
\fancyfoot[LE,RO]{\sffamily\thepage}
\fancyhead[CE,CO]{\textit{\footnotesize\sffamily Data current as of August 2020; verify before using}}
\textcolor{ccgreen}{ANALYSIS OF FINDINGS}
\\
\input{\rootpath/TEXT/analysis_text/\tds_analysis}

\pagebreak
% ############################################
\pagestyle{plain}
\whitetext{\Chapter{\textcolor{lightBlue}{Appendices}}}
\whitetext{\textbf{\Huge Appendices}}
\pagecolor{lightBlue}

\pagebreak
\newpagecolor{white}
\pagestyle{fancy}
\fancyhf{}
\renewcommand{\chaptermark}[1]{\markboth{#1}{}}
\fancyfoot[LE,RO]{\sffamily\thepage}
\textcolor{lightBlue}{Appendix I}

Appendices are to include staffing schedules, site plans, and floorplans for relevant developments, to the extent that each are available and relevant to our purposes here.

\pagebreak

% ############################################

% ############################################
\begin{comment}

\section{Lists}

Unordered Lists:
\begin{itemize}
\item This is an unordered list. 
\item Item 2.
\item It has three items.
\end{itemize}

Ordered List:
\begin{enumerate}
\item This is an ordered list.
\item Item 2.
\item It has three items.
\end{enumerate}

Ordered List (alphabetical):
\begin{enumerate}[label=\Alph*.]
\item This is an ordered list.
\item Item 2.
\item It has three items.
\end{enumerate}

% ----------------------------------------------
\cleardoublepage
\chapter{Figures}\label{ch:figures}

% ############################################
\section{Images}\label{sec:images}

\autoref{fig:image1} shows how to display images.

\begin{figure}[H]
	\centering
	\includegraphics[width=0.5\textwidth]{example-image-a.pdf}
	\caption{Image}

	\label{fig:image1}
\end{figure}

\begin{figure}[H]
	\centering
	\includegraphics[width=0.5\textwidth]{example-image-b.pdf}
	\caption{Image with Source}
	\captionsource{\cite{mus:16}}	
	\label{fig:image2}
\end{figure}

\begin{figure}[H]
	\centering
	\includegraphics[width=0.5\textwidth]{example-image-c.pdf}
	\caption{Image with Source and Link}
	\captionsource[https://example.org]{J. Doe}	
	\label{fig:image3}
\end{figure}

% ############################################

% ----------------------------------------------
\cleardoublepage
\chapter{Conclusion}\label{ch:conclusion}

\end{comment}

