
\textbf{Inspection and Collection Requirement}

In a compliance questionnaire conducted in the summer of 2020, the Soundview Consolidation appeared to be in compliance with the inspection requirements of Paragraph 45 of the HUD agreement. However, the consolidation reported insufficient staff to meet the collection portion of the requirements. At the time of this interview, the site was an Alternative Work Schedule (AWS) site. 

The Housing Manager, Noelia Guardiola, reported that the Soundview Consolidation does not have enough staff to correct observed deficiencies and caretakers cannot thoroughly complete all of their tasks in a day. NYCHA caretakers pick up trash inside the buildings multiple times a day, including weekends. NYCHA caretakers also conduct ground inspections and pick up litter multiple times a day, including weekends. Staff begins collecting trash at 6:00 AM and ends at 4:15 PM daily. 

\textbf{Removal or Storage Requirement}

The consolidation appeared to be in compliance with the  removal or storage requirement of Paragraph  45 of the HUD Agreement because it has containers in the form of exterior compactors to store waste in a manner that prevents pests on the days DSNY does not come to pick up waste. Based on the same summer of 2020 questionnaire, the consolidation reported the following conditions.

The Soundview Consolidation reported at the time of the interview that DSNY comes when called to collect full compactors. The consolidation also reported that it received ten bulk tickets a month for the removal of bulk waste. Bulk trash sits in a parking lot with an exterior container before being picked up by the vendor. In terms of storage, residents of this consolidation have access to trash chutes and may drop their waste at 24 additional sites on the premises. After the trash is collected from the drop-off sites, it is placed in exterior compactors. Tenants are asked by management to leave their garbage at designated drop-off site if they choose not to use the chutes. Most tenants dispose of their trash on sidewalks, near recycling bins and outside their buildings even when trash chutes are working. Waste is stored in exterior compactors. 

A single site visit on 12 December, 2019 showed exposed trash on the grounds and curbside. It also showed that waste was not stored in a way that prevents pests on that day. 

The consolidation reported that, on average, 100-200 compactor bags (40 lbs. bags) and two 2-yard containers are disposed of from Soundview daily. There are four exterior compactors at this consolidation with two having holes in them. Ms. Guardiola stated that she intended to reach out to contact Arrow Steel to weld the holes. There are also 26 interior compactors with five currently shut down. Ms. Guardiola reports that interior compactors shutting down is a consistent problem.

According to the questionnaire, there are external sources of trash and bulk waste illegally dumped at this site. When it happens, it is from nearby businesses and neighboring households. Ms. Guardiola reports obstacles such as not having enough staff and illegal dumping after hours, which both contribute towards waste management issues. She firmly believes that more staff and cameras to monitor dumping will alleviate the problem.

3. \textbf{Additional Context}

In a June 24, 2020 report, the Monitor Cleanliness Team gave Soundview a C+/D- rating. 