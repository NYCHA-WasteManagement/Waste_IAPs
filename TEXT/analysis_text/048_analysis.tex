
\textbf{Inspection and Collection Requirement}s



In a Compliance interview conducted on December 13, 2019, the Ravenswood consolidation appeared to be in compliance with the inspection and collection requirements of Paragraph 45 of the HUD agreement. The consolidation reported sufficient staff to meet this portion of the requirements. At the time of this interview, the site was not an Alternative Work Schedule (AWS) site. A follow-up with the site in the summer of 2020 confirmed the development was now checking the grounds and removing waste seven days a week. Staff begins picking up waste around 6:00 AM and ends around 3:30 PM.

The Supervisor of Grounds, Daniel Ruiz, reported that Ravenwood consolidation does have enough staff to correct observed deficiencies and caretakers cannot usually complete all of their tasks in a day. NYCHA caretakers pick up trash inside the buildings one to two times a day, now including weekends. NYCHA caretakers also conduct ground inspections and pick up litter at least twice daily. Staff begins collecting trash between 8:00 AM -- 10:00 AM and ends before 4:00 PM daily.

\textbf{Removal or Storage Requirement}



At the time of the Compliance interview, the Ravenswood consolidation appeared to be in compliance with the storage and removal requirement of Paragraph 45 of the HUD Agreement because it does have containers in the form of exterior compactors to store waste in a manner that prevents pests on the days DSNY does not come to pick up the garbage. 



Ravenswood reported at the time of the interview that DSNY comes when the exterior compactors are full. The consolidation also stated that it received seven to eight bulk tickets a month to remove bulk waste. Bulk trash sits in a yard with an exterior container before being picked up by the vendor. In terms of storage, residents of this consolidation have access to trash chutes and may drop their waste at 20 additional sites on the premises. After the trash is collected from the drop-off sites, it is placed exterior compactors. Tenants are asked by management to leave their garbage on development grounds if they choose not to use the chutes. Most tenants dispose of their trash by leaving it curbside in front of their buildings. After waste is picked up from the grounds, it is stored in exterior compactors. 



In the Compliance interview, Mr. Ruiz stated that Ravenswood did not have a pest problem because it uses an exterminator. He also noted that waste can be stored in a manner that prevents pests at Ravenswood.

The consolidation reported that, on average, 100 -- 200 compactor bags (40 lbs. bags) are disposed of daily. There are four exterior compactors at this consolidation that were all in condition at the Compliance interview time.  

Ravenswood does accept waste from other developments', nor does it take its waste to other developments.  According to the Compliance interview, external sources of trash and bulk waste are illegally dumped at this site. When it happens, it is from nearby restaurants and local people. The waste consists mostly of furniture and appliances. According to Mr. Ruiz, Ravenswood's most significant obstacle for waste management is the scheduling of 30-yard pickups. He also noted that the best thing Management/Operations has done to improve trash management is to receive new EZ-Packs.

3. Additional Context

In a June 24, 2020 report, the Monitor Cleanliness Team gave Ravenswood Houses a B+ rating.