
\textbf{Inspection and Collection Requirement} 

 

In a Compliance interview conducted on December 12, 2019, the Monroe Consolidation appeared to be in compliance with the inspection and collection requirements of Paragraph 45 of the HUD agreement. At the time of this interview, the site was an Alternative Work Schedule (AWS) site. 

The Superintendent, John Rivera, reported that the Monroe consolidation does not have enough staff to correct observed deficiencies, but caretakers can usually complete all of their tasks in a day. NYCHA caretakers picked up trash inside the buildings one to two times a day, including weekends. NYCHA caretakers also conducted ground inspections and picked up litter at least twice daily. Staff begins collecting trash between 6:00 AM -- 8:00 AM and ends before 4:00 PM daily.

\textbf{Removal or Storage Requirement} 

 

At the time of the Compliance interview, the Monroe consolidation appeared to be in compliance with the storage and removal requirement of Paragraph 45 of the HUD Agreement because it has containers in the form of exterior compactors to store waste in a manner that prevents pests on the days DSNY does not come to pick up garbage. Despite having the necessary equipment to store waste properly, there seem to be some issues with how to do it effectively, and more follow up is needed.

 

Monroe reported at the time of the interview that DSNY comes when the compactors are full, usually five to six times a week. The consolidation also stated that it received five to six bulk tickets a month to remove bulk waste. Bulk trash sits in a yard with an exterior container before being picked up by the vendor. In terms of storage, residents of this consolidation have access to trash chutes and may not drop their waste at additional sites on the premises. Tenants are not asked by management to leave their garbage on development grounds if they are unable to or choose not to use the chutes. Most tenants dispose of their trash by using the trash chutes. Once the waste is collected from the grounds, it is stored in exterior compactors. 

A single site visit in December showed the grounds had exposed litter and debris and that there were not enough trash bins throughout the campus. It also showed that waste was stored in a way that prevents pests on that day. In the Compliance interview, Mr. Rivera stated that the Monroe consolidation did have a pest problem and that waste could not be stored in a way that prevents pests. He said that Monroe is using an exterminator to treat the problem. 

The consolidation reported that, on average, less than 100 compactor bags (40 lbs. bags)  are disposed of from Monroe daily. There are three exterior compactors at this consolidation that were all in good condition at the time of the interview.  

This consolidation can bring its waste to Soundview Houses if necessary and vice versa. According to the Compliance interview, there are external sources of trash and bulk waste illegally dumped at this site. 

\textbf{Additional Context} 

In a June 24, 2020 report, the Monitor Cleanliness Team gave Monroe Houses a D/D- rating. 