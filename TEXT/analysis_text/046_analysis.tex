

\textbf{Inspection and Collection Requirement}

In a compliance interview conducted on November 4th, 2019, the consolidation appeared to be in compliance with the inspection requirements of Paragraph 45 of the HUD agreement. The consolidation reported sufficient staff to meet the collection portion of the requirements. At the time of this interview, the site was not an Alternative Work Schedule (AWS) site. (IF PHONE CONFIRMATION: A follow up phone call to the site in the summer of 2020 confirmed that the development [WAS/WAS NOT] checking the site and removing waste seven days a week.)

---

The Supervisor of Caretakers, Rosalyn Perry], reported that it does have enough staff to correct observed deficiencies and caretakers can usually not complete all of their tasks in a day, due to the amount of garbage and employee disinterest. NYCHA caretakers pick[ed] up trash inside the buildings 1-2 times a day, including weekends. NYCHA caretakers also conduct[ed] ground inspections and pick[ed] up litter 1-2 times a day, including weekends. Staff begins collecting trash between 8:00 AM -- 10:00 AM and ends before 4:00 PM daily.

\textbf{Removal or Storage Requirement}

This site is curbside and because DSNY does not pick up from the curb everyday there is a high likelihood that this site is not in compliance as they cannot store waste in an exterior compactor on days when DSNY cannot pick up.

---

Boulevard reported at the time of the interview that DSNY comes on Mondays, Wednesdays, and Saturdays. The consolidation also reported that it received 7-8 bulk tickets for the removal of bulk waste, a decrease from their much-needed 10. Bulk trash sits curbside before being picked up by the vendor. In terms of storage, residents of this consolidation have access to trash chutes and may drop their waste at 12 additional sites on the premises. After the trash is collected from the drop-off sites, it is taken off the premises. Tenants are asked by management not to leave their garbage on development grounds if they choose not to use the chutes. Most tenants dispose of their trash using trash shoots. Waste is stored at the curb before it is taken off the premises. 

A single site visit on November 4th, 2019 showed little to no trash on the grounds, along with trash bins with open lids placed throughout the premises. It also showed that waste was not stored in a way that prevents pests on that day. Ms. Perry stated in the Compliance Interview that Boulevard did have a pest problem. There is no exterior compactor at this location and tenants may not dispose of trash as instructed. An exterminator was called and meetings with tenants have been held about trash disposal along with fliers distributed.

The consolidation reported that on average, 100-200 compactor bags (40 lbs. bags) are disposed of from Boulevard daily. 

According to the Compliance interview, there are external sources of trash and bulk waste illegally dumped at this site. When it happens, it is from passersby and nearby construction. The Supervisor has stated that tenants not disposing of trash as instructed is an obstacle which prevents staff from keeping the consolidation free of litter and trash. 

3. \textbf{Additional Context}

In a June 9th, 2020 report, the Monitor Cleanliness Team gave Boulevard a B/B+ rating. 