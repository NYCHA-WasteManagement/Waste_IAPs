
\textbf{Inspection and Collection Requirement}

Amsterdam Consolidation is in compliance with the inspection and collection requirement of Paragraph 45 of the HUD Agreement. At the time of the Compliance Interview, conducted on December 12, 2019, Amsterdam was not an AWS site and not in compliance. However, it was confirmed on July 7, 2020, that Amsterdam now inspects the grounds and collects trash from the buildings seven days a week. The Superintendents, James Artis and Anthony Bono, reported, during the December Compliance Interview, that staff patrols the grounds for cleaning and maintenance but there is insufficient manpower to correct observed deficiencies. However, the Superintendents stated that caretakers are usually able to complete all of their tasks in a day.

\textbf{Removal or Storage Requirement}

Amsterdam Consolidation is in compliance with the storage and removal requirement of Paragraph 45 of the HUD Agreement because waste is stored in a manner that prevents pests (e.g., exterior compactors). DSNY comes when Amsterdam's compactors are full, usually one to two times a week.

In terms of storage and disposing of litter into interior trash chutes, there are no additional drop-off sites for tenants to use when not using the trash chutes. Tenants are not asked by management to leave their garbage curbside or in front of their buildings if they choose not to use the chutes, but most tenants leave trash in front of their buildings anyway. There are three exterior compactors at this location that were all in good condition as of December.

Amsterdam Consolidation has three bulk containers and 27 interior compactor rooms that were all accessible and working at the time of the Compliance Interview. Amsterdam disposes of 300 - 400 compactor bags (40 lbs. bags) daily. A site visit in December showed that Amsterdam is able to be store their waste in a way to prevent pests (e.g., trash bins). However, the Superintendents stated that Amsterdam does have a pest problem, but it is not as bad as it was before due to help from consistent extermination and glue traps. 

Amsterdam does not take its waste to any other developments, nor does it allow any other developments to dispose of its waste on its property. According to the Compliance Interview, there are external sources of trash and bulk waste illegally dumped at this site, primarily by construction companies, nearby restaurants and local people. The waste consists mostly of construction material, food, furniture, and appliances. According to the  Superintendents, the most significant obstacles Amsterdam faces are staffing shortages and resident behaviors. The residents do not use the hopper doors and dump illegally. 

In a June 24, 2020 report, the Monitor Cleanliness Team gave Amsterdam an A grade. Amsterdam Addition and Harborview Terrace had not yet been evaluated by the team. 