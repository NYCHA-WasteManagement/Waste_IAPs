 

\textbf{Inspection and Collection Requirement}

In a Compliance interview conducted on January 8, 2020, the Berry consolidation appeared to be in compliance with the inspection and collection requirements of Paragraph 45 of the HUD agreement. At the time of this interview, the site was an Alternative Work Schedule (AWS) site. 

The Superintendent, Earl Lindsey, reported that the Berry consolidation does not have enough staff to correct observed deficiencies, and caretakers cannot usually complete all of their tasks in a day because of AWS. NYCHA caretakers pick up trash inside the buildings one to two times a day, including weekends. NYCHA caretakers also conduct ground inspections and pick up litter one to two times a day. Staff begins collecting trash between 8:00 AM -- 10:00 AM and ends before 4:00 PM daily.

\textbf{Removal or Storage Requirement}

At the time of the Compliance interview, the Berry consolidation appeared to be in compliance with the storage and removal requirement of Paragraph 45 of the HUD Agreement because it does have containers in the form of exterior compactors to store waste in a manner that prevents pests on the days DSNY does not come to pick up garbage.

Berry reported at the time of the interview that DSNY comes Mondays and Wednesdays. The consolidation also stated that it received three to four bulk tickets a month to remove bulk waste. Bulk trash sits in a yard with an exterior container before being picked up by the vendor. In terms of storage, residents of this consolidation have access to trash chutes and may drop their waste at 17 additional sites on the premises. After the trash is collected from the drop-off sites, it is placed in exterior compactors. Tenants are asked by management not to leave their garbage on development grounds if they choose not to use the chutes. Most tenants dispose of their trash by leaving it at the drop-off site, but not in the bins. Once the waste is collected from the grounds, it is stored in an exterior compactor.  

A single site visit in January showed debris and litter on the grounds, but lids were closed on trash bins and containers.  This visit also showed that waste could not be stored in a way that prevents pests on that day. However, Mr. Lindsey stated in the Compliance Interview that Berry did not have a pest problem even though he thought the consolidation was unable to store waste in a way that prevents pests. 

The consolidation reported that, on average, less than 100 compactor bags (40 lbs. bags) are disposed of from Berry daily. There is one exterior compactor at this consolidation that has holes in it. Mr. Lindsey planned on reaching out to Arrow Steel to weld the holes. 

The Berry consolidation does not bring waste to other developments, nor allow waste from other developments. According to the Compliance interview, external sources of trash and bulk waste are illegally dumped at this site. When it happens, it is construction materials from nearby houses. Mr. Lindsey said the most significant obstacle Berry faces with waste management is that the EZ pack breaks twice a week.

3. \textbf{Additional Context} 

Berry Houses has not yet been grade by the Monitor Cleanliness Team.