
\textbf{Inspection and Collection Requirement}

The consolidation appeared to be in compliance with the inspection and collection requirements of Paragraph 45 of the HUD agreement. Compliance could not conduct a site visit during the 2019-2020 period; however, in a survey conducted in the summer of 2020, the consolidation reported the following conditions.

The Assistant Property Maintenance Supervisor, Hashim Gittens, reported that Patterson does not have enough staff to correct observed deficiencies and caretakers cannot usually complete all of their tasks in a day. NYCHA caretakers pick up trash inside the buildings three to four times a day, including weekends. NYCHA caretakers also conduct ground inspections and pick up litter at least twice daily. Staff begins collecting trash around 6:00 AM and ends before 4:00 PM daily.

\textbf{Removal or Storage Requirement}

The consolidation appeared to be in compliance with the removal or storage requirement of Paragraph  45 of the HUD Agreement because it has containers in the form of exterior compactors to store waste in a manner that prevents pests on the days DSNY does not come to pick up waste. Based on the same summer of 2020 survey, the consolidation reported the following conditions. 

At the time of the survey, Patterson reported that DSNY comes when the compactors are full, at least four times a week. The consolidation also stated that it received nine to twelve bulk tickets for the removal of bulk waste. Bulk trash sits in a yard with an exterior container before being picked up by the vendor. In terms of storage, residents of this consolidation have access to trash chutes and may drop their waste at 15 additional sites on the premises. After the trash is collected from the drop-off sites, it is placed in the exterior compactors. Tenants are asked by management not to leave their garbage on development grounds if they choose not to use the chutes. Most tenants dispose of their trash by leaving it on the floors inside the stair halls, throwing it out the windows, or leaving it in the elevators. Once the waste is collected from the buildings, it is stored in the exterior compactors.  

In an unannounced site visit in December 2019, showed that Patterson had bulk waste in front of multiple buildings. These items included  a kitchen cabinet, mattresses, and washer/dryer combo unit. Mr. Gittens stated in the survey that Patterson Consolidation does have a pest problem. However, the consolidation does have an exterminator treat the rat burrows regularly.

The consolidation reported that, on average, 150 compactor bags (40 lbs. bags) are disposed of from Patterson daily. There are four exterior compactors at this consolidation that are all in good condition with no holes as of the survey time.  

Patterson reports that it only accepts waste from other developments when the other developments are having mechanical issues.  According to the survey, there are external sources of trash and bulk waste illegally dumped at this site. When it happens, it is from nearby stores and private entities. Mr. Gittens stated that the best thing Management/Operations has done to improve waste management is purchase rat proof trash cans.

3. \textbf{Additional Context}

In a June 24, 2020 report, the Monitor Cleanliness Team gave Patterson Houses a C- rating. 