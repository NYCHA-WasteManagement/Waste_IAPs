
\textbf{Inspection and Collection Requirement}  

In a Compliance interview conducted on November 1, 2019, the Pelham Parkway Consolidation appeared to be in compliance with the inspection and collection requirements of Paragraph 45 of the HUD agreement. At the time of this interview, the site was not an Alternative Work Schedule (AWS) site. A follow-up phone call to the site in the summer of 202 confirmed that the development was now checking the sites and removing waste seven days a week.

The Supervisor Caretaker, Antonio Garcia, reported that Pelham Parkway does not have enough staff to correct observed deficiencies, and caretakers cannot usually complete all of their tasks in a day due to AWS. NYCHA caretakers pick up trash inside the buildings one to two times a day, including weekends. Additionally, NYCHA caretakers also conduct ground inspections and pick up litter at least twice daily. Staff begins collecting trash between 7:00 AM and ends around 5:00 PM daily.

 

\textbf{Removal or Storage Requirement}  

At the time of the compliance interview, the Pelham Parkway Consolidation appeared to be in compliance with the storage and removal requirement of Paragraph 45 of the HUD Agreement because it does have containers in the form of exterior compactors to store waste in a manner that prevents pests on the days DSNY does not come to pick up the garbage. Despite having the potential ability to store waste in exterior compactors, this consolidation seems to lack resources to effectively implement this requirement.

 

Pelham Parkway reported at the time of the interview that DSNY comes when the exterior compactors are full, usually one to two times a week. The consolidation also stated that it received over eight bulk tickets to remove bulk waste a month. However, it noted that ten tickets would be better. Bulk trash sits in a yard with an exterior container before being picked up by the vendor. In terms of storage, residents of this consolidation have access to trash chutes and may drop their waste at 18 additional sites on the premises. After the trash is collected from the drop-off sites, it is placed in exterior compactors. Tenants are asked by management to leave their garbage on development grounds if they choose not to use the chutes. Most tenants dispose of their trash using the trash chutes. Waste is stored in designated areas with trash bins before its removal offsite. Once caretakers remove waste from the buildings, it is stored in exterior compactors.

 

A single site visit in November 2019 showed exposed trash all over the grounds. The visit noted that Pelham Parkway did not have enough trash bins throughout the campus. It also showed that waste was not stored in a way that prevents pests on that day. Furthermore, Mr. Garcia echoed this in the Compliance Interview by saying that Pelham Parkway did have a pest problem and did not store its waste in a way that prevented pests. He also noted that an exterminator treats the basement monthly. 

The consolidation reported that, on average, less than 100 compactor bags (40 lbs. bags)  are disposed of from Pelham Parkway daily.  There are three exterior compactors at this consolidation that were all in good condition at the time of the Compliance interview.   

The Pelham Parkway Consolidation does not take its waste to any other developments nor take garbage from other developments.  According to the Compliance interview, external sources of trash and bulk waste are illegally dumped at this site. When it happens, it is from local people and consists of construction materials from nearby houses, furniture and appliances, lawn clippings, and bathroom fixtures.  According to Mr. Garcia, the biggest obstacle Pelham Parkway faces in regards to waste management is tenants placing trash out all day. 

\textbf{Additional Context}

In a June 24, 2020 report, the Monitor Cleanliness Team gave Pelham Parkway Houses a D+/C- rating.  Boston Road Plaza has not been graded at this time.  