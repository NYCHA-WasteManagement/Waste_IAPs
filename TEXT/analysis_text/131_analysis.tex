
\textbf{Inspection and Collection Requirement}

In a compliance interview conducted on September 26th, 2019, the consolidation appeared to be in compliance with the inspection requirements of Paragraph 45 of the HUD agreement. The consolidation reported insufficient staff to meet the collection portion of the requirements. At the time of this interview, the site was an Alternative Work Schedule (AWS) site. 

---

The Superintendent, Rigoberto Charriez, reported that it does not have enough staff to correct observed deficiencies and caretakers can usually not complete all of their tasks in a day. NYCHA caretakers picked up trash inside the buildings more than four times a day, including weekends. NYCHA caretakers also conducted ground inspections and picked up litter more than four times a day, including weekends. Staff begins collecting trash between 8:00 AM -- 10:00 AM and ends before 4:00 PM daily.

\textbf{Removal or Storage Requirement}

At the time of the compliance interview, Tompkins appeared to be in compliance with the storage and removal requirement of Paragraph 45 of the HUD Agreement because it does have containers in the form of exterior compactors to store waste in a manner that prevents pests on the days DSNY does not come to pick up waste. 

---

Tompkins reported at the time of the interview that DSNY comes on Tuesdays, Thursdays, Saturdays, or simply when the compactor is full. The consolidation also reported that it received 3-4 bulk tickets for the removal of bulk waste. The Superintendent reported at the time of the interview that more tickets were needed. Bulk trash sits in a yard with an exterior container before being picked up by the vendor. In terms of storage, residents of this consolidation do not have access to trash chutes and may drop their waste at four additional sites on the premises. After the trash is collected from the drop-off sites, it is placed in one of two exterior compactors. Tenants are asked by management not to leave their garbage on development grounds if they choose not to use the chutes. Most tenants dispose of their trash using trash bins. Waste is stored in the bulk container located on the premises. . 

A single site visit on September 26th, 2019 showed exposed trash and litter on the development grounds. It also showed that waste was not stored in a way that prevents pests on that day due to exposed trash bags. Rigoberto Charriez stated in the Compliance Interview that Tompkins did not have a pest problem.

The consolidation reported that on average, 100-200 compactor bags (40 lbs. bags) are disposed of from Tompkins daily. There are two exterior compactors at this consolidation.

Tompkins reports that if necessary, it can take its trash to Marcy, Sumner, and Farragut but does not receive trash from other developments. According to the Compliance interview, there are external sources of trash and bulk waste illegally dumped at this site. When it happens, it is from food, furniture, and appliances. Mr. Charriez reports not having enough staff for tasks as being an obstacle the development faces. 

3. \textbf{Additional Context} 

In a June 24, 2020 report, the Monitor Cleanliness Team gave Tompkins an A-/B+ rating. 

*At the time of interview the consolidation noted needing more tickets. Further inquiry is needed to verify that this is still the case.