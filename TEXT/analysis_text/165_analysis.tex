

\textbf{Inspection and Collection Requirement}

The consolidation appeared to be in compliance with the inspection and collection requirements of Paragraph 45 of the HUD agreement. Compliance could not conduct a site visit during the 2019-2020 period; however, in a phone interview conducted in the summer of 2020, the consolidation reported the following conditions.

---

The Assistant Superintendent, David Falcon, reported that Beach 41st Street does not have enough staff to correct observed deficiencies and caretakers cannot usually complete all of their tasks in a day. NYCHA caretakers pick up trash inside the buildings as needed throughout the day, including weekends. NYCHA caretakers also conduct ground inspections and pick up litter as needed throughout the day, including weekends. Staff begins collecting trash at 6:00 AM and ends before 4:00 PM daily. 

\textbf{Removal or Storage Requirement}

The consolidation appeared to be in compliance with the  removal or storage requirement of Paragraph  45 of the HUD Agreement because it has containers in the form of exterior compactors to store waste in a manner that prevents pests on the days DSNY does not come to pick up waste. Based on the same summer of  2020 survey, the consolidation reported the following conditions.

---

Beach 41st Street reported at the time of the survey that DSNY comes on Mondays, Wednesdays, and Saturdays. The consolidation also reported that it received 7-10 bulk tickets a month for the removal of bulk waste. The Assistant Superintendent stated that a bulk crusher is needed at this consolidation. Bulk trash sits in a yard with an exterior container before being picked up by the vendor. In terms of storage, residents of this consolidation have access to trash chutes and may drop their waste at 11 additional sites on the premises. After the trash is collected from the drop-off sites, it is placed in the exterior compactor. Tenants are not asked by management to leave their garbage on development grounds if they are unable or choose not to use the chutes. Most tenants dispose of their trash in front of their buildings or in front of shoot doors. Once waste is collected from the grounds, waste is stored in the exterior compactor. 

The consolidation reported that, on average, 100 compactor bags (40 lbs. bags) are disposed of from Beach 41st Street daily. There are 2 exterior compactors at this consolidation.

According to the survey, there are external sources of trash and bulk waste illegally dumped at this site. When it happens, it is from residents and construction. Short staffing due to AWS is an obstacle preventing caretakers from keeping Beach 41st Street free from litter and trash, according to Mr. Falcon. 

3. \textbf{Additional Context} 

In a June 15th, 2020 report, the Monitor Cleanliness Team gave Beach 41st Street a B/B- rating. 