
\textbf{Inspection and Collection Requirement}

In a Compliance interview conducted on November 8, 2019 the Parkside consolidation appeared to be in compliance with the inspection and collection requirements of Paragraph 45 of the HUD agreement. At the time of this interview, the site was not an Alternative Work Schedule (AWS) site. A follow-up with the site in the summer of 2020 confirmed the development was now checking the site and removing waste seven daily.

Jorge Rosado, the Assistant Superintendent, reported that the Parkside Consolidation has enough staff to correct observed deficiencies and caretakers can usually complete all of their tasks in a day. NYCHA caretakers pick up trash inside the buildings three to four times a day, including weekends. NYCHA caretakers also conduct ground inspections and pick up litter at least twice daily. 

\textbf{Removal or Storage Requirement}

At the time of the Compliance interview, the Parkside consolidation appeared to be in compliance with the storage and removal requirement of Paragraph 45 of the HUD Agreement because the consolidation does have containers in the form of exterior compactors to store waste in a manner that prevents pests on the days DSNY does not come to pick up the waste.  

At the time of the Compliance interview, Parkside reported that DSNY comes when the exterior compactors are full, usually one to two times a week. The consolidation also stated that it received three to four bulk tickets a month to remove bulk waste. Bulk trash sits in a yard with an exterior container before being picked up by the vendor. In terms of storage, residents of this consolidation have access to trash chutes and may drop their waste at 20 additional sites on the premises. After the trash is collected from the drop-off sites, it is placed in the exterior compactors. Tenants are asked by management not to leave their garbage on development grounds if they choose not to use the chutes. However, most tenants dispose of their trash using the trash chutes. After waste is picked up from the buildings, it is stored in exterior compactors.

A single site visit in November showed exposed trash on the grounds, and that waste was unable to be stored in a way that prevents pests on that day. Furthermore, Mr. Rosado stated in the Compliance interview in November that Parkside Consolidation did have a pest problem. He noted that the consolidation is using the exterminator's door sweepers to help treat the problem.

The consolidation reported that, on average, 100 -  200 compactor bags (40 lbs. bags) are disposed of from Parkside daily. There are two exterior compactors at this consolidation that were both in good condition at the time of the interview. 

If necessary, the Parkside consolidation reports that its' can take its trash to other developments like Gunhill Houses, Eastchester Houses, and Pelham Parkway Houses. Also, those developments can bring their waste to Parkside if necessary. According to the Compliance interview, external sources of trash and bulk waste are illegally dumped at this site. When it happens, it is from local people and usually consists of furniture and appliances. According to Mr. Rosado, the biggest obstacle the Parkside consolidation faces for waste management is that residents throw trash everywhere.  

3. \textbf{Additional Context} 

In a June 24, 2020 report, the Monitor Cleanliness Team gave Parkside Houses an A rating.