Forest Consolidation Analysis: 

1.     \textbf{Inspection and Collection Requirement}s 

In a compliance interview on October 29, 2019, the consolidation appeared to be in compliance with the inspection requirements of Paragraph 45 of the HUD Agreement, although the consolidation reported insufficient staff to meet the collection portion of the requirements. At the time of the interview, this was and continues to be an AWS site.  

Although the site is inspected and cleaned multiple times per day, the Supervisor of Caretakers reported that it does not have enough staff to correct observed deficiencies and caretakers are usually not able to complete all of their tasks in a day. NYCHA caretakers reported picking up trash inside the buildings three to four times a day, including weekends. NYCHA caretakers also reported conducted ground inspections and picked up litter three to four times a day, including weekends. Staff begins collecting trash between 8:00 AM -- 10:00 AM and end before 4:00 PM daily.  

 

2.     \textbf{Removal or Storage Requirement} 

At the time of the compliance interview, the Forest Consolidation did not appear to be in compliance with the storage and removal requirement of Paragraph 45 of the HUD Agreement because it did not store waste in a manner that prevents pests, due to reported staffing deficiencies that prevented caretakers from being able to collect all waste scattered around the site.  Semi-containerization at the drop-off sites may help ease the staffing burden by providing storage in cases where there is insufficient manpower to bring all of the waste on the grounds to the exterior compactor. 

DSNY comes when the compactor is full, usually three to four times a week. Bulk trash sits in a yard with an exterior container before being picked up by the vendor. In terms of storage, residents of this consolidation do have access to trash chutes and may drop their waste at 21 additional sites on the premises.  After the trash is collected from the drop-off sites, it is placed in an exterior compactor. Tenants are asked by management to leave their garbage on development grounds if they choose not to use the chutes. Most tenants dispose of their trash using the trash chutes. Waste is securely stored in exterior compactors.  

A single site visit on October 29, 2019 showed significant amounts of exposed trash.  However, it should be noted that this is a single day of observation for the consolidation. The Supervisor Caretaker stated in the Compliance Interview that the Forest Consolidation did not have a pest problem.  

The development reported that on average, 100-200 compactor bags (40 lbs. bags) are disposed of from the consolidation daily. There are five exterior compactors at this consolidation with at least one in need of repair. The Supervisor Caretaker stated that they intended to reach out to contact Arrow Steel to make repairs. 

The Forest Consolidation reports that if necessary, it can take its trash to the Adams, St. Mary's and Butler Developments and vice versa. According to the Compliance Interview, there are external sources of trash and bulk waste illegally dumped at this site. When it happens, it is from local people, nearby residents with waste consisting of food, furniture and appliances. The Supervisor Caretaker states that the main obstacles facing the consolidation is that tenants do not dispose of waste in proper locations and throw trash out the windows. 

In a June 24, 2020 report, the Monitor Cleanliness Team gave the Forest Consolidation an A-/B+ rating.