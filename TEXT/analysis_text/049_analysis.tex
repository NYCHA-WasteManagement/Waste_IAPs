

\textbf{Inspection and Collection Requirement}

In a Compliance interview conducted on November 26, 2019, the Marble Hill consolidation appeared to be in compliance with the inspection and collection requirements of Paragraph 45 of the HUD agreement. At the time of this interview, the site was an Alternative Work Schedule (AWS) site. 

The Superintendent, Rodney McNeil, reported that Marble Hill does have enough staff to correct observed deficiencies, but caretakers cannot usually complete all of their tasks in a day. NYCHA caretakers picked up trash inside the buildings one to two times a day, including weekends. Additionally, NYCHA caretakers also conducted ground inspections and picked up litter one to two times a day. Staff begins collecting trash between 6:00 AM -- 8:00 AM and ends after 5:00 PM daily.

\textbf{Removal or Storage Requirement}

At the time of the Compliance interview, the Marble Hill consolidation appeared to be in compliance with the storage and removal requirement of Paragraph 45 of the HUD Agreement because it does have containers in the form of exterior compactors to store waste in a manner that prevents pests on the days DSNY does not come to pick up the waste. Despite having the equipment to prevent pests, the consolidation may have some issues preventing pests.  

Marble Hill reported at the time of the interview that DSNY comes when the exterior compactors are full, usually one to two times a week. The consolidation also stated that it received seven to eight bulk tickets a month to remove bulk waste. Bulk trash sits in a yard with an exterior container before being picked up by the vendor. In terms of storage, residents of this consolidation have access to trash chutes and may drop their waste at 11 additional sites on the premises. After the trash is collected from the drop-off sites, it is placed in exterior compactors. Tenants are asked by management not to leave their garbage on development grounds if they choose not to use the chutes. Most tenants dispose of their waste using trash chutes. Once the waste is collected, it is stored in the exterior compactors.  

A single site visit in November showed that grounds with some exposed trash and not many trash bins throughout the campus. It also showed that waste was not stored in a way that prevents pests on that day. Furthermore, Mr. McNeil stated in the Compliance interview that Marble Hill did have a pest problem and waste could not be stored in a way that prevented pests. However, exterminators are working to fix this issue.

The consolidation reported that, on average, less than 100 compactor bags (40 lbs. bags) are disposed of from Marble Hill daily. There are three exterior compactors at this consolidation. There are holes in some of these compactors, and Mr. McNeil states that the consolidation planned on reaching out to Arrow Steel to weld the holes. 

The consolidation reports that it can take its trash to Fort Independence and vice versa if necessary. According to the Compliance interview, external sources of trash and bulk waste are illegally dumped at this site. The illegal waste is usually construction materials from nearby houses and furniture and appliances.  

3. \textbf{Additional Context} 

The Monitor Cleanliness Team has not yet given marble Hill Houses a grade.