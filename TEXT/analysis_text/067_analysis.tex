

\textbf{Inspection and Collection Requirement}

The consolidation appears to be in compliance with the inspection and collection requirements of Paragraph 45 of the HUD agreement. Compliance could not conduct a site visit during the 2019-2020 period; however, in a phone interview conducted in the summer of 2020, the consolidation reported the following conditions.

The Assistant Property Maintenance Supervisor, Joel Parrish, reported that Sotomayor Consolidation does not have enough staff to correct observed deficiencies and caretakers cannot usually complete all of their tasks in a day. NYCHA caretakers pick up waste inside the buildings multiple times a day, including weekends. NYCHA caretakers also conduct ground inspections and pick up litter multiple times a day, including weekends. Staff begins collecting waste at 6:00 AM and ends at 7:00 PM daily. 

\textbf{Removal or Storage Requirement}

This consolidation partially has its waste collected from the curbside and because DSNY does not pick up from the curb everyday there is a high likelihood that this site is not in compliance as they cannot store waste in an exterior compactor on days when DSNY cannot pick up. 1471 Walton Avenue is the only development in this consolidation that does not use exterior compactors to store waste.

Sotomayor Consolidation reported at the time of the survey that DSNY comes Mondays, Wednesdays, Fridays and Saturdays. The consolidation also reported that it received nine bulk tickets a month for the removal of bulk waste. Mr. Parrish reports needing more bulk tickets. Residents of this consolidation have access to trash chutes and may drop their waste at additional sites on the premises. Most tenants dispose of their waste at undesignated areas such as hallways, lobbies and out their windows. Once waste is collected from the grounds, waste is stored in an external compactor. 

The consolidation reported that, on average, 200-300 compactor bags (40 lbs. bags) are disposed of from Sotomayor Consolidation daily. There are four exterior compactors and two 30-yard containers at this consolidation. Mr. Parrish also reports that there are 32 interior compactors.

According to the survey, there are external sources of waste and bulk illegally dumped at this site.  Mr. Parrish reports needing more grounds staff to address waste management needs. He also noted that exterminators come daily to combat pests.

3. \textbf{Additional Context} 

In a June 24, 2020 report, the Monitor Cleanliness Team gave the Sotomayor Houses a D/D- rating, the 1471 Watson Ave development a , Glebe Ave-Westchester Ave 