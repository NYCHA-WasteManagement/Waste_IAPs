Edenwald Consolidation Analysis

\textbf{Inspection and Collection Requirement}

Edenwald Consolidation appeared to be in compliance with inspection and collection requirements of Paragraph 45 of the HUD Agreement, according to a Compliance Interview conducted on September 21, 2019. Both building and grounds inspections are conducted one to two times a day. Litter is picked up by caretakers one to two times a day as well. The consolidation is an AWS site which may contribute to the compliance of the consolidation. The Supervisor of Grounds, Dennis Norford, reports that despite the staff's ability to comply with Paragraph 45, understaffing continues to be an issue. Mr. Norford stated that caretakers are not usually able to complete all of their tasks in a day.  

\textbf{Removal or Storage Requirement}

Edenwald Consolidation is in compliance with Paragraph 45 of the Hud Agreement because waste can be stored in a manner that prevents pests (e.g., exterior compactors).

DSNY comes to collect trash on Tuesdays and Fridays. They also come on Saturdays when needed. An average of 14 bulk tickets is created for the removal of bulk waste. Bulk trash sits in a yard with an exterior container before it is picked up.

In terms of storage, in addition to disposing of litter into interior trash chutes, residents of this consolidation may drop their waste at eleven additional sites on the premises. Tenants are not asked by management to leave their garbage curbside if they choose not to use the chutes. Most tenants who do not use chutes leave it in front of their building or at non-designated drop-off sites. Waste is stored in front of buildings both in bins and exposed bags prior to offsite removal. Once waste is collected it is brought to one of the exterior compactors where it is stored in a manner that prevents pests. Mr. Norford reports that one interior compactor is inaccessible due to flood conditions at 4040 Laconia Avenue.

Edenwald reports that if necessary the consolidation can bring waste to the Parkside, Eastchester and Gun Hill developments. According to the Compliance Interview, there are external sources of trash and bulk waste illegally dumped at this site, mostly from private households and random passersby. The majority of this waste is from construction materials, food, and furniture and appliances from private homes. Mr. Norford reports that the largest obstacle facing the consolidation is understaffing. 

The Monitor Cleanliness Team has not given Edenwald Houses a cleanliness rating as of July 7, 2020. 