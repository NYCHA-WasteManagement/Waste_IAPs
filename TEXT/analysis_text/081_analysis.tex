 

\textbf{Inspection and Collection Requirement} 

 

The consolidation appeared to be in compliance with the inspection and collection requirements of Paragraph 45 of the HUD Agreement. Compliance could not conduct a site visit during the 2019-2020 period; however, in a survey conducted in the summer of 2020, the consolidation reported the following conditions.

The Supervisor of Grounds, Jerell Redd, reported that the Manhattanville consolidation does not have enough staff to correct observed deficiencies. Caretakers cannot usually complete all of their tasks in a day but give 100% effort to try despite being very short-staffed. NYCHA caretakers pick up trash inside the buildings two to three times a day, including weekends. NYCHA caretakers also conduct ground inspections and pick up litter at least twice daily. The staff begins collecting trash at 6:00 AM and ends at 4:30 PM daily.

\textbf{Removal or Storage Requirement}  

 

This site is partially curbside. Because DSNY does not pick up from the curb every day, there is a high likelihood that this site is not in compliance as they cannot store waste in an exterior compactor at the curbside locations on days when DSNY cannot pick up. Curbside developments at this consolidation include Manhattanville Rehab (Group 2) and  Manhattanville Rehab (Group 3).  Based on the same summer of  2020  survey, the consolidation reported the following conditions.

 

Manhattanville reported at the time of the survey that DSNY comes three to four times a week. The consolidation also stated that it received 12 bulk tickets a month for the removal of bulk waste. Bulk trash sits in a yard with an exterior container before being picked up by the vendor. In terms of storage, residents of this consolidation have access to trash chutes and may drop their waste at six additional sites on the premises.  Most tenants dispose of their trash by using the trash chutes. Once the waste is collected from the grounds, waste is stored in the exterior compactors. It is unclear how waste is stored at the curbside locations, and more follow-up is needed.

 

Mr. Redd stated in the survey that consolidation did have a pest problem. There are three exterior compactors at this consolidation. Manhattanville does not take waste to, nor accept trash from other developments. According to the survey, there are no external sources of trash and bulk waste illegally dumped at this site. Mr.  Redd said the most significant thing Management/Operations has done for trash management is to increase the number of bulk tickets per month. He also said the biggest obstacle Manhattanville faces is being short-staffed.

\textbf{Additional Context}  

In a June 24, 2020 report, the Monitor Cleanliness Team gave Manhattanville an A rating. Manhattanville Rehab (Group 2) and  Manhattanville Rehab (Group 3) have not yet been graded.

 