
\textbf{Inspection and Collection Requirement}

In a compliance interview conducted on October 16th, 2019, the consolidation appeared to be in compliance with the inspection requirements of Paragraph 45 of the HUD agreement. The consolidation reported insufficient staff to meet the collection portion of the requirements. At the time of this interview, the site was an Alternative Work Schedule (AWS) site. 

---

The Supervisor of Grounds, Adrian Welch, reported that it does not have enough staff to correct observed deficiencies and caretakers can usually not complete all of their tasks in a day. NYCHA caretakers picked up trash inside the buildings 1-2 times a day, including weekends. NYCHA caretakers also conducted ground inspections and picked up litter 1-2 times a day, including weekends. Staff begins collecting trash between 6:00 AM -- 8:00 AM and ends between 4:00 PM - 5:00 PM daily. 

\textbf{Removal or Storage Requirement}

At the time of the compliance interview, Cypress Hills appeared to be in compliance with the storage and removal requirement of Paragraph 45 of the HUD Agreement because it does have containers in the form of exterior compactors to store waste in a manner that prevents pests on the days DSNY does not come to pick up waste.

---

Cypress Hills reported at the time of the interview that DSNY comes Tuesdays and Fridays, or whenever they call because the compactor is full. The consolidation also reported that it received 3-4 bulk tickets for the removal of bulk waste. Bulk trash sits in a yard with an exterior container before being picked up by the vendor. In terms of storage, residents of this consolidation do not have access to trash chutes and may drop their waste at 30 additional sites on the premises. After the trash is collected from the drop-off sites, it is placed in the exterior compactor. Tenants are asked by management to leave their garbage on development grounds if they choose not to use the chutes. Most tenants dispose of their trash using trash bins in front of their buildings. Waste is stored in exterior compactors. 

A single site visit on October 16th, 2019 showed exposed trash and litter on the grounds and rash bins placed throughout the area with open lids. It also showed that waste was stored in a way that prevents pests on that day, thanks to the exterior compactor. Adrian Welch stated in the Compliance Interview that Cypress Hills did have a pest problem due to garbage kept on the premises before removal. Mr. Welch stated that outside contractors have been called to solve the infestation problem and close contact is kept with DSNY to ensure a speedy pickup.

The consolidation reported that on average, 300-400 compactor bags (40 lbs. bags) and 1-2 2-yard containers are disposed of from Cypress Hills daily. There are three exterior compactors at this consolidation.

Cypress Hills reports that if necessary, it can take its trash to Pink Houses, Linden Houses, and Tilden Houses and may receive trash from Pink Houses and Vandalia Houses. According to the Compliance interview, there are external sources of trash and bulk waste illegally dumped at this site. When it happens, it is from nearby restaurants, passersby, and construction. The Supervisor of Grounds stated that due to AWS, there isn't enough staff to take care of problems on most days. 

3. \textbf{Additional Context} 

In a June 9th, 2020 report, the Monitor Cleanliness Team gave Cypress Hills an A- rating. 