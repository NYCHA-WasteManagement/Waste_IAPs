

\textbf{Inspection and Collection Requirement}

In a compliance interview conducted on October 1st, 2019, the consolidation appeared to be in compliance with the inspection requirements of Paragraph 45 of the HUD agreement. The consolidation reported sufficient staff to meet the collection portion of the requirements. At the time of this interview, the site was an Alternative Work Schedule (AWS) site. (IF PHONE CONFIRMATION: A follow up phone call to the site in the summer of 2020 confirmed that the development [WAS/WAS NOT] checking the site and removing waste seven days a week.)

---

The Sylvia Correa, the Supervisor Carketaker, reported that it does have enough staff to correct observed deficiencies and caretakers can usually complete all of their tasks in a day. NYCHA caretakers pick[ed] up trash inside the buildings 1-2 times a day, including weekends. NYCHA caretakers also conduct[ed] ground inspections and pick[ed] up litter 1-2 times a day, including weekends but with less staff. Staff begins collecting trash between 8:00 AM -- 10:00 AM and ends after 5:00 PM daily. 

\textbf{Removal or Storage Requirement}

At the time of the compliance interview, Farragut appeared to be in compliance with the storage and removal requirement of Paragraph 45 of the HUD Agreement because it does have containers in the form of exterior compactors to store waste in a manner that prevents pests on the days DSNY does not come to pick up waste.

---

Farragut reported at the time of the interview that DSNY comes when requested, if the compactor is full which is about 1-2 times a week. The consolidation also reported that it received 5-6 bulk tickets for the removal of bulk waste. Bulk trash sits in a yard with an exterior container before being picked up by the vendor. In terms of storage, residents of this consolidation do not have access to trash chutes and may drop their waste at 7 additional sites on the premises. After the trash is collected from the drop-off sites, it is placed in the exterior compactor. Tenants are asked by management not to leave their garbage on development grounds if they choose not to use the chutes. Most tenants dispose of their trash using trash bins located at drop-off sites, at drop-off sites but not in the bin, or at nondesignated sites throughout the development. Waste is stored in the exterior compactor. 

A single site visit on October 1st, 2019 showed little to no litter on the grounds upon both arrival and departure. Trash and recycling bins are placed throughout the site with open lids. It also showed that waste was stored in a way that prevents pests on that day. Sylvia Correa stated in the Compliance Interview that Farragut did have a pest problem. Epoxy has been placed on the floor of compactor rooms and an exterminator comes on call in order to fix this issue. 

The consolidation reported that on average, 100-200 compactor bags (40 lbs. bags) are disposed of from Farragut daily. There are three exterior compactors at this consolidation.

According to the Compliance interview, there are external sources of trash and bulk waste illegally dumped at this site. When it happens, it is from construction and a nearby Church pantry. Sylvia Correa explained that AWS has made managing staff and assignments extremely difficult.  

3. \textbf{Additional Context}

In a My 27th, 2020 report, the Monitor Cleanliness Team gave Farragut an A-/B+ rating. 