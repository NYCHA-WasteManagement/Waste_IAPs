
\textbf{Inspection and Collection Requirement}

In a compliance interview conducted on October 23, 2019, the Morrisania Air Rights Consolidation did not appear to be in compliance with the inspection requirements of Paragraph 45 of the HUD agreement. The consolidation reported insufficient staff to meet the collection portion of the requirements. At the time of this interview, the site was not an Alternative Work Schedule (AWS) site.  A follow up phone call to the site in the summer of 2020 confirmed that the development was checking the site and removing waste seven days a week.

The Supervisor of Grounds, Kareem Jones, reported that Morrisania Air Rights does not have enough staff to correct observed deficiencies and caretakers cannot usually complete all of their tasks in a day. NYCHA caretakers picked up trash inside the buildings more than four times a day, not including weekends and thus they were not in compliance. NYCHA caretakers also conducted ground inspections and picked up litter three to four times a day, not including weekends and thus they were not in compliance. Staff began collecting trash between 6:00 AM -- 8:00 AM and ends before 4:00 PM daily. The follow up phone call indicates that Morrisania Air Rights is likely in compliance with the inspection and collection requirements.

\textbf{Removal or Storage Requirement}

At the time of the compliance interview, the Morrisania Air Rights appeared to be in compliance with the storage and removal requirement of Paragraph 45 of the HUD Agreement because it has containers in the form of exterior compactors to store waste in a manner that prevents pests on the days DSNY does not come to pick up waste.

Morrisania Air Rights reported at the time of the interview that DSNY comes when called for collection. The consolidation also reported that it received 13 bulk tickets a month for the removal of bulk waste. Bulk trash sits in a yard with an exterior container before being picked up by the vendor. In terms of storage, residents of this consolidation have access to trash chutes and may drop their waste at 12 additional sites on the premises. After the trash is collected from the drop-off sites, it is placed in an exterior compactor. Tenants are not asked by management to leave their garbage on development grounds if they choose not to use the chutes. Most tenants dispose of their trash using trash chutes. Waste is stored in exterior compactors. 

A single site visit on October 23, 2019 showed exposed trash on the grounds and curbside. There were trash and recycling bins throughout the site, but not all of them had lids. For these reasons, the Monitor decided that waste was not stored in a way that prevents pests on that day. Mr. Jones stated in the compliance interview that Morrisania Air Rights did have a pest problem.

The consolidation reported that, on average, 100-200 compactor bags (40 lbs. bags) are disposed of from Morrisania Air Rights daily. There are four exterior compactors at this consolidation with at least one that has a hole. Mr. Jones stated that he intended to reach out to contact Arrow Steel to weld the hole.

Morrisania Air Rights reports that if necessary, it can take its trash to Melrose and may receive trash from Melrose as well. According to the compliance interview, there are external sources of trash and bulk waste illegally dumped at this site. When it happens, it is from passerby's, often dumping furniture and appliances. Mr. Jones reports that obstacles facing the consolidation include tenants not placing trash in designated areas and being short of staff. The consolidation was planning to provide new trash cans at the time of the interview

3. \textbf{Additional Context}

In a June 24, 2020 report, the Monitor Cleanliness Team gave Morrisiania Air Rights an F rating. 