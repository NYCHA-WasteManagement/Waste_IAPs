
\textbf{Inspection and Collection Requirement}

In a Compliance interview conducted on September 25, 2019, the LaGuardia Consolidation appeared to be in compliance with the inspection and collection requirements of Paragraph 45 of the HUD agreement. At the time of this interview, the site was not an Alternative Work Schedule (AWS) site. A follow-up survey with the site in the summer of 2020 confirmed that the development is now checking the grounds and removing waste seven days a week, but it feels short-staffed due AWS. 

The Property Manager Superintendent, Paula Davenport, reported that the LaGuardia Consolidation does not have enough staff to correct observed deficiencies, and caretakers cannot usually complete all of their tasks in a day. NYCHA caretakers pick up trash inside the buildings three times a day. NYCHA caretakers also conduct ground inspections and pick up litter daily. Staff begins collecting trash around 6:00 AM and ends around 6:00 PM.

\textbf{Removal or Storage Requirement}

This site is partially curbside. Since waste is not removed from the curb every day, there is a high likelihood that this site is not in compliance with the removal or storage requirement of Paragraph  45 of the HUD Agreement as it cannot store waste in an exterior compactor on days when DSNY or EISI cannot pick up. More follow up is needed. The curbside developments at this consolidation is Two Bridges U.R.A (Site 7.)

LaGuardia reported its containers are picked up on scheduled days by IESI (a private hauler) when full. DSNY occasionally assists with drop-off sites.  The consolidation also stated that it received six bulk tickets a month for the removal of bulk waste. Bulk trash sits in a yard with an exterior container before being picked up by the vendor. In terms of storage, residents of this consolidation have access to trash chutes and may drop their waste at six additional sites on the premises. After the trash is collected from the drop-off sites, it is placed containers in the compound. Most tenants dispose of their trash using the trash chutes or leaving at the drop sites. After waste is collected, it is stored in exterior compactors.  

According to the LaGuardia Rat Reduction Action Plan, as of January 2019, there were 169 burrows while a year prior, there were 110. A single site visit in September showed satisfactory grounds conditions with little debris. It also showed that waste was stored in a way that prevented pests on that day. Furthermore, Ms. Davenport stated in the summer survey that LaGuardia did not have a pest problem. The consolidation reported that, on average, 100 -- 200 compactor bags (40 lbs. bags) are disposed of from LaGuardia daily. There are two exterior compactors at this consolidation that were both in good condition. 

The LaGuardia consolidation reports that it does not take waste to any other developments, but other developments sometimes reach out to dump their waste at LaGuardia. According to the survey, external sources of trash and bulk waste are illegally dumped at this site. When it happens, it is from local people dumping furniture and appliances and household trash. LaGuardia reports that the most significant obstacles it faces regarding waste management are being short-staffed because of AWS and the added city food program waste due to COVID-19.

\textbf{Additional Context} 

In a June 24, 2020 report, the Monitor Cleanliness Team gave LaGuardia Houses an A rating and LaGuardia Addition an A rating. Two Bridges U.R.A (Site 7) has not yet been graded. 