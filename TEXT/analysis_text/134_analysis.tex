Chelsea Analysis: 

\textbf{Inspection and Collection Requirement} 

 

The consolidation appeared to be in compliance with the inspection and collection requirements of Paragraph 45 of the HUD Agreement. Compliance could not conduct a site visit during the 2019-2020 period; however, in a phone interview conducted in the summer of 2020, the consolidation reported the following conditions.

The Assistant Superintendent, Raymond Santiago, reported that the Chelsea consolidation does have enough staff to correct observed deficiencies, and caretakers can usually complete all of their tasks in a day. NYCHA caretakers pick up trash inside the buildings two to three times a day, including weekends. NYCHA caretakers also conduct ground inspections and pick up litter at least twice daily. The staff begins collecting trash at 6:00 AM and ends at 7:00 PM daily.

\textbf{Removal or Storage Requirement} 

The consolidation appeared to be in compliance with the removal or storage requirement of Paragraph  45 of the HUD Agreement because it has containers in the form of exterior compactors to store waste in a manner that prevents pests on the days DSNY does not come to pick up the waste. Based on the same summer of  2020  phone interview, the consolidation reported the following conditions.

 

Chelsea reported at the time of the interview that DSNY comes four to five times a week. The consolidation also stated that it received six bulk tickets a month for the removal of bulk waste. Bulk trash sits in a yard with an exterior container before being picked up by the vendor. In terms of storage, residents of this consolidation have access to trash chutes and may drop their waste at six additional sites on the premises. Most tenants dispose of their trash by ``leaving it wherever they want.'' Once the waste is collected from the grounds, it is stored in the exterior compactors.  

 

Mr. Santiago stated in the interview that consolidation did have a small pest problem. It often comes and goes, but exterminators treat the area. It recently got worse due to ongoing construction. The consolidation reported that, on average, 200 compactor bags (40 lbs. bags)  are disposed of from Chelsea daily. There are two exterior compactors at this consolidation that were both in relatively good condition at the time of the interview.

Chelsea reports that if necessary, it can take its trash to nearby mixed-finance developments and may receive trash from those developments. According to the interview, there are rarely external sources of trash and bulk waste illegally dumped at this site. When it happens, it is from contractors. Mr. Santiago said the most significant obstacle Chelsea faces regarding waste management is the homeless population. The COVID-19 pandemic has exacerbated the situation and tasks that used to take 30 minutes may now take up to two hours. He also stated the best thing Management/Operations has done for trash management is to send out newsletters and flyers to the residents about proper trash disposal.  

\textbf{Additional Context}  

In a June 24, 2020 report, the Monitor Cleanliness Team gave Chelsea Houses, Chelsea Addition, and Elliot Houses A ratings. 

 