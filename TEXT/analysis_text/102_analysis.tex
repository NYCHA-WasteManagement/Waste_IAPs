

\textbf{Inspection and Collection Requirement}

The consolidation appeared to be in compliance with the inspection and collection requirements of Paragraph 45 of the HUD agreement. Compliance could not conduct a site visit during the 2019-2020 period; however, in a phone interview conducted in the summer of 2020, the consolidation reported the following conditions.

The Supervisor of Caretakers, Kevin Briggs, reported that the Morris consolidation does not have enough staff to correct observed deficiencies and caretakers cannot usually complete all of their tasks in a day, but they try. NYCHA caretakers pick up trash inside the buildings throughout the day. NYCHA caretakers also conduct ground inspections and pick up litter at least twice daily. Staff begins collecting trash at 6:00 AM and ends at 7:00 PM daily. 

\textbf{Removal or Storage Requirement}

The consolidation appeared to be in compliance with the removal or storage requirement of Paragraph  45 of the HUD Agreement because it has containers in the form of exterior compactors to store waste in a manner that prevents pests on the days DSNY does not come to pick up the garbage. Based on the same summer of  2020 survey, the consolidation reported the following conditions.

Morris reported at the time of the survey that DSNY comes when the exterior compactors are full. The consolidation also stated that it received ten bulk tickets a month for the removal of the bulk waste but could use more. Bulk trash sits in a yard with an exterior container before being picked up by the vendor. In terms of storage, residents of this consolidation have access to trash chutes and may drop their waste at additional sites on the premises. After the trash is collected from the drop-off sites, it is placed in exterior compactors. Tenants are asked by management to leave their garbage on development grounds if they are unable or choose not to use the chutes. Most tenants dispose of their trash by leaving it in front of the buildings. Once the waste is collected from the grounds, it is stored in the exterior compactors.  

In the survey, Mr. Briggs stated that Morris did have a pest problem, but exterminators are treating the area. According the Morris II Rat Reduction Plan, this specific development had 95 burrows in the summer of 2017 and was down to 14, a 70% reduction, by December 2018. The consolidation reported that, on average, 150 -- 250 compactor bags (40 lbs. bags) are disposed of from Morris daily. There are five exterior compactors at this consolidation that are all brand new and in good condition.  

The Morris consolidation does not take waste to any other development, nor accept trash from other developments. According to the survey, there are external sources of trash and bulk waste illegally dumped at this site. When it happens, it is from the stores and complex across the street.  The biggest obstacles this consolidation faces for was management are lack of vehicles to remove trash and bulk, and lack of communication with the residents. The most significant improvement Management/Operations has done to improve trash management is post signage about bad behavior. 

3. \textbf{Additional Context}

In a June 24, 2020 report, the Monitor Cleanliness Team gave Morris I a B/B+ and Morris II an A rating. 