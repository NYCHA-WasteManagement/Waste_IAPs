
\textbf{Inspection and Collection Requirement}

In a compliance interview conducted on November 21st, 2019, the consolidation appeared to be in compliance with the inspection requirements of Paragraph 45 of the HUD agreement. The consolidation reported sufficient staff to meet the collection portion of the requirements. At the time of this interview, the site was not an Alternative Work Schedule (AWS) site. (IF PHONE CONFIRMATION: A follow up phone call to the site in the summer of 2020 confirmed that the development [WAS/WAS NOT] checking the site and removing waste seven days a week.)

---

The Supervisor of Grounds, Kimberly Grant, reported that it does have enough staff to correct observed deficiencies and caretakers can usually complete all of their tasks in a day. NYCHA caretakers pick[ed] up trash inside the buildings 3-4 times a day,including weekends. NYCHA caretakers also conduct[ed] ground inspections and pick[ed] up litter 1-2 times a day, including weekends. Staff begins collecting trash between 8:00 AM -- 10:00 AM and ends before 4:00 PM daily. 

\textbf{Removal or Storage Requirement}

At the time of the compliance interview, the Roosevelt Consolidation appeared to be in compliance with the storage and removal requirement of Paragraph 45 of the HUD Agreement because it does have containers in the form of exterior compactors to store waste in a manner that prevents pests on the days DSNY does not come to pick up waste.

---

Roosevelt reported at the time of the interview that DSNY comes on Tuesdays, Thursdays, and Saturdays. The consolidation also reported that it received 5-6 bulk tickets for the removal of bulk waste. Bulk trash sits in a yard with an exterior container before being picked up by the vendor. In terms of storage, residents of this consolidation do not have access to trash chutes and may drop their waste at 18 additional sites on the premises. After the trash is collected from the drop-off sites, it is placed in the exterior compactor. Tenants are asked by management not to leave their garbage on development grounds if they choose not to use the chutes. Most tenants dispose of their trash using nondesignated drop-off sites. Waste is stored in the exterior compactor.  

A single site visit on November 21st, 2019 showed exposed trash and litter on the grounds upon arrival and departure. Trash bins are pleased throughout the site with open lids. It also showed that waste was not stored in a way that prevents pests on that day. Kimberly Grant stated in the Compliance Interview that Roosevelt did have a pest problem due to holes in walls and residents not disposing of trash as instructed. Exterminators have been contacted, walls are being covered, and resident outreach is taking place to tackle these issues.

The consolidation reported that on average, 100-200 compactor bags (40 lbs. bags) and 1-2 2-yard containers are disposed of from Roosevelt daily. There are two exterior compactors at this consolidation.

Roosevelt reports that if necessary, it can take its trash to Stuyvesant Gardens, and may receive trash from Stuyvesant Gardens. According to the Compliance interview, there are external sources of trash and bulk waste illegally dumped at this site. When it happens, it is from restaurants, construction, and passersby. Kimberly Grant stated that illegal dumping and residents placing trash in undesignated areas are obstacles preventing caretakers from keeping Roosevelt free of trash and litter.

3. \textbf{Additional Context} 

In a May 27th, 2020 report, the Monitor Cleanliness Team gave Roosevelt a B+/A- rating. 