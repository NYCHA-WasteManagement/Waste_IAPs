

\textbf{Inspection and Collection Requirement}

The consolidation appeared to be in compliance with the inspection and collection requirements of Paragraph 45 of the HUD agreement. Compliance could not conduct a site visit during the 2019-2020 period; however, in a survey conducted in the summer of 2020, the consolidation reported the following conditions.

The Supervisor of Grounds, Candido Betances, reported that the Webster consolidation does have enough staff to correct observed deficiencies and caretakers can usually complete all of their tasks in a day. NYCHA caretakers pick up trash inside the buildings three times a day, including weekends. NYCHA caretakers also conduct ground inspections and pick up litter at least twice daily. Staff begins collecting trash around 6:00 AM and ends at 3:30 PM daily. 

\textbf{Removal or Storage Requirement}

The consolidation appeared to be in compliance with the removal or storage requirement of Paragraph  45 of the HUD Agreement because it has containers in the form of exterior compactors to store waste in a manner that prevents pests on the days DSNY does not come to pick up the waste. Based on the same summer of  2020 survey, the consolidation reported the following conditions.

Webster reported at the time of the survey that DSNY comes Wednesdays and Saturdays. The consolidation also stated that it received six to eight bulk tickets a month to remove bulk waste. Bulk trash sits in a yard with an exterior container before being picked up by the vendor. In terms of storage, residents of this consolidation have access to trash chutes and may not drop their waste at additional sites on the premises. Tenants are not asked by management to leave their garbage on development grounds if they are unable or choose not to use the chutes. Most tenants dispose of their trash using the trash chutes. Once the waste is collected from the grounds, waste is stored in the exterior compactors.  

The supervisor stated in the survey that Webster did have a pest problem, but exterminators come weekly to treat the situation. According to the Webster Rat Reduction Plan, burrows have decreased from 201 in the summer of 2017 to 16 as of January 2019. However, according to the Morrisania Rat Reduction Plan, burrows have increased from 58 in February 2018 to 65 in January 2019. The consolidation reported that, on average, 80 compactor bags (40 lbs. bags) are disposed of from Webster daily. There are two exterior compactors at this consolidation that are in good condition at the time of the survey.

Webster reports that it can take its trash to other developments if necessary, but this is a rare occurrence.  According to the survey, there are external sources of trash and bulk waste illegally dumped at this site. When it happens, it is usually from a specific deli on 168th and Webster that drops garbage at this development. According to Mr. Betances, the most significant obstacle Webster faces in regards to trash management is a situation with D.O.T and pickup. He also stated the best thing Management/Operations has done to improve waste management is to provide more trash cans.  