

\textbf{Inspection and Collection Requirement}

The consolidation appeared to be in compliance with the inspection requirements of Paragraph 45 of the HUD agreement. Compliance could not conduct a site visit during the 2019-2020 period; however, in a phone interview conducted in the summer of 2020, the consolidation gave the following responses.

---

The Property Manager, Patricia Williams-Brewster, reported that it does not have enough staff to correct observed deficiencies and caretakers can usually not complete all of their tasks in a day. NYCHA caretakers pick up trash inside the buildings 3 times a day, including weekends. NYCHA caretakers also conduct ground inspections and pick up litter 3 times a day, including weekends. Staff begins collecting trash at 8:00 AM and ends at 4:15 PM daily.

\textbf{Removal or Storage Requirement}

The consolidation appeared to be in compliance with the  removal or storage requirement of Paragraph  45 of the HUD  Agreement Based on the same summer of  2020 phone interview, the consolidation gave the following responses.

---

Breukelen reported at the time of the interview that DSNY comes on Mondays, Wednesdays, and Fridays. The consolidation also reported that it received 10 bulk tickets for the removal of bulk waste. Bulk trash sits in a yard with an exterior container before being picked up by the vendor. In terms of storage, residents of this consolidation have access to trash chutes and may drop their waste at 30 additional sites on the premises. After the trash is collected from the drop-off sites, it is placed in the exterior compactor. Tenants are asked by management not to leave their garbage on development grounds if they choose not to use the chutes. Most tenants dispose of their trash using shoots. Waste is stored in the exterior compactor. 

The consolidation reported that on average, 200-300 compactor bags (40 lbs. bags) are disposed of from Breukelen daily. There are four exterior compactors at this consolidation. 

According to the phone interview, there are external sources of trash and bulk waste illegally dumped at this site. When it happens, it is from stores and neighboring homeowners. Ms. Williams-Brewster stated that residents dumping garbage out of their windows was an obstacle preventing caretakers from keeping the development free from litter and trash.