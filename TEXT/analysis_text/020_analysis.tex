


\textbf{Inspection and Collection Requirement}s 

Lincoln Consolidation is in compliance with the inspection and collection requirement of Paragraph 45 of the HUD Agreement, according to a Compliance interview conducted on July 6, 2020. The Property Manager, Kathie Shoulders, reported that staff patrols the grounds for cleaning and maintenance but that  there is insufficient manpower to correct all observed deficiencies. Ms. Shoulders stated that caretakers are not usually able to complete all of their tasks in a day due to staff shortages and an overwhelming amount of trash.  NYCHA caretakers pick up trash inside the buildings and monitor the grounds twelve to fifteen times a day, including weekends. Daily trash collection begins between 8:00 AM -- 10:00 AM and ends before 4:00 PM. 

 

\textbf{Removal or Storage Requirement} 

Lincoln Consolidation is in compliance with the storage and removal requirement of Paragraph 45 of the HUD Agreement because they are able to store waste in a manner that prevents pests (e.g., exterior compactors).



DSNY comes when compactors are full to pick-up trash, usually three to four times a week. An average of ten bulk tickets are created each month for the removal of bulk waste. Bulk trash sits in a yard with an exterior container before it is picked up.  

 

In terms of storage and disposing of litter into interior trash chutes, residents of this consolidation may drop their waste at additional sites on the premises. Once collected from the drop-off site, trash is taken to EZ Pack containers (2-yard containers). Tenants are not asked by management to leave their garbage curbside if they choose not to use the chutes, but most tenants leave their trash in front of their buildings. Waste is stored curbside in exposed trash bags at the premises before being moved offsite. There are two exterior compactors. One of them is extremely old and too small for their needs. It has many holes in it. Ms. Shoulders has reached out to Arrow Steel to weld the holes. 

 

Lincoln has one bulk container and 20 interior compactor rooms that were all accessible and working at the time of the Interview. Lincoln disposes of around 25 compactor bags (40 lbs. bags) daily. When the trash is not able to be removed from the premises, it is  usually able to be stored in a way to prevent pests (e.g., trash bins), according to the Compliance Interview. If the trash is not moved from the premises, it goes to EZ Packs containers or exterior compactors. The supervisor also stated that Lincoln does not have a pest problem. They used to have a pest problem, but since joining the Neighborhood Rate Reduction late last year, there numbers have dropped from 500 burrows to single digits.  



Lincoln reports that, if necessary, it can take its trash to nearby developments like Wagner Houses, Jackie Robinson Houses, Harlem River Houses, and Wilson Houses and vice versa. According to the Interview, there are external sources of trash and bulk waste illegally dumped at this site. The waste consists mostly of construction material and waste from a nearby grocery shop. According to Ms. Shoulders, the biggest obstacles the sites faces are the tenants dumping everywhere, being short staffed, and being in a high traffic zone for people commuting to work. People cut across the property to get to different subway stations and do not throw away their waste in a proper manner. 

 

The Monitor Cleanliness Team has not yet given Lincoln Houses a rating.