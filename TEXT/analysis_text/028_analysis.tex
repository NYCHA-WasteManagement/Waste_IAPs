
\textbf{Inspection and Collection Requirement}

In a compliance interview conducted on October 23, 2019, the Melrose consolidation appeared to be in compliance with the inspection and collection requirements of Paragraph 45 of the HUD agreement. However, at the time of this interview, the site was not an Alternative Work Schedule (AWS) site. A follow-up phone call to the site in the summer of 2020 confirmed that the consolidation is now checking the site and removing waste seven days a week.

The Supervisor of Grounds reported that Melrose does not have enough staff to correct observed deficiencies, but caretakers usually complete all of their tasks in a day. NYCHA caretakers pick up trash inside the buildings at least two times a day, including weekends. NYCHA caretakers also conduct ground inspections and pick up litter twice daily. Staff begins collecting trash between 8:00 AM -- 10:00 AM and ends before 4:00 PM daily.

\textbf{Removal or Storage Requirement}

At the time of the Compliance interview, the Melrose Consolidation appeared to be in compliance with the storage and removal requirement of Paragraph 45 of the HUD Agreement because it does have containers in the form of exterior compactors to store waste in a manner that prevents pests on the days DSNY does not come to pick up the waste.

Melrose reported at the time of the interview that DSNY comes when the compactors are full, one to two times a week. The consolidation also reported that it received eight bulk tickets a month for the removal of bulk waste. Bulk trash sits in a yard with an exterior container before being picked up by the vendor. In terms of storage, residents of this consolidation have access to trash chutes and may not drop their waste at additional sites on the premises. Tenants are asked by management not to leave their garbage on development grounds if they choose not to use the chutes. Most tenants dispose of their trash using the chutes. Waste is stored in exterior compactors once collected.

A single site visit in October 2019, showed grounds conditions unsatisfactory with exposed trash and litter/debris on the grounds. It also showed that waste was not stored in a way that prevents pests on that day. However, the Supervisor of Grounds stated in the Compliance Interview that Melrose did not have a pest problem. Additionally, according to the  Melrose Rat Reduction Action Plan, there were very few visible burrows as of March 28, 2019.

The consolidation reported that, on average, 100 -- 200 compactor bags (40 lbs. bags) are disposed of from Melrose daily. There are three exterior compactors at this consolidation that were all in good condition at the interview time.

If necessary, Melrose reports that it can take its trash to  Jackson/Morrisania Air Rights Houses and vice versa. According to the Compliance interview, external sources of trash and bulk waste are illegally dumped at this site. When it happens, it is from construction companies, nearby restaurants, and locals. The waste consists mostly of construction materials, furniture, and appliance. According to the  Supervisor, the most significant obstacle Melrose Houses faces is that residents bring their trash out after caretakers have already picked it up.  

3. \textbf{Additional Context}

The Monitor Cleanliness Team has not yet given Melrose a rating.