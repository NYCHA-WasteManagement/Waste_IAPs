Saint Nicholas Analysis: 

\textbf{Inspection and Collection Requirement} 

 

The consolidation appeared to be in compliance with the inspection and collection requirements of Paragraph 45 of the HUD Agreement. Compliance could not conduct a site visit during the 2019-2020 period; however, in a survey conducted in the summer of 2020, the consolidation reported the following conditions.

The Superintendent, Clarence Samuel, reported that Saint Nicholas does not have enough staff to correct observed deficiencies, and caretakers cannot usually complete all of their tasks in a day. NYCHA caretakers pick up trash inside the buildings once a day, including weekends. NYCHA caretakers also conduct ground inspections and pick up litter at least twice daily. The staff begins collecting trash at 6:00 AM.

\textbf{Removal or Storage Requirement} 

The consolidation appeared to be in compliance with the removal or storage requirement of Paragraph  45 of the HUD Agreement because it has containers in the form of exterior compactors to store waste in a manner that prevents pests on the days DSNY does not come to pick up the waste. Based on the same summer of  2020  survey, the consolidation reported the following conditions.

  

The Saint Nicholas consolidation reported at the time of the survey that DSNY comes three times a week. The consolidation also stated that it received ten bulk tickets a month for the removal of bulk waste. Bulk trash sits in a yard with an exterior container before being picked up by the vendor.  In terms of storage, residents of this consolidation have access to trash chutes and may not drop their waste at additional sites on the premises.  However, most tenants dispose of their trash by leaving it outside in front of their buildings and do not utilize the trash chutes as much. Once the waste is collected from the grounds, it is stored in the exterior compactors.  

 

Mr. Samuels stated in the survey that consolidation did have a pest problem. The consolidation has been sealing off pest's entry points to buildings and collapsing rat burrows. The consolidation reported that, on average, 150 compactor bags (40 lbs. bags)  are disposed of from Saint Nicholas daily. There are three exterior compactors at this consolidation that were all in good condition at the time of the survey.  

Saint Nicholas does not take its waste to any other developments nor accept waste from developments. According to the survey, there are external sources of trash and bulk waste illegally dumped at this site. When it happens, it is from grocery stores. Mr. Samuels said the most significant obstacle Saint Nicholas faces regarding waste management is cars continually parking in front of the containers and compactors on Frederick Douglas Blvd due to not have parking signs installed by DOT. Sanitation has, on numerous occasions, been unable to pick up containers causing garbage to accumulate due to full machines. He also stated the best thing Management/Operations has done for trash management is to provide more dump tickets when needed.  

\textbf{Additional Context}  

In a June 24, 2020 report, the Monitor Cleanliness Team gave Saint Nicholas a C rating.  

 