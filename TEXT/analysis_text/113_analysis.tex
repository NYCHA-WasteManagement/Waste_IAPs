Butler Analysis: 

\textbf{Inspection and Collection Requirement} 

 

The consolidation appeared to be in compliance with the inspection and collection requirements of Paragraph 45 of the HUD Agreement. Compliance could not conduct a site visit during the 2019-2020 period; however, in a phone interview conducted in the summer of 2020, the consolidation reported the following conditions.

The Housing Manager, Orlando Padro, reported that the Butler consolidation does not have enough staff to correct observed deficiencies, and caretakers cannot usually complete all of their tasks in a day. NYCHA caretakers attempt to pick up trash inside the buildings two to three times a day, including weekends. NYCHA caretakers also conduct ground inspections and pick up litter at least twice daily. The staff begins collecting trash at 6:00 AM and ends at 7:00 PM daily. 

\textbf{Removal or Storage Requirement} 

The consolidation appeared to be in compliance with the removal or storage requirement of Paragraph  45 of the HUD Agreement because it has containers in the form of exterior compactors to store waste in a manner that prevents pests on the days DSNY does not come to pick up the waste. Based on the same summer of  2020  phone interview, the consolidation reported the following conditions.

 

Butler reported at the time of the interview that DSNY comes when the compactors are full, usually three times a week. The consolidation also said that it does not receive bulk tickets because it has a bulk crusher.  In terms of storage, residents of this consolidation have access to trash chutes and may not drop their waste at additional sites on the premises. Most tenants dispose of their trash by leaving in on the floor or sometimes bringing it to the front of the buildings. Once the waste is collected from the grounds, it is stored in the exterior compactors.

 

In the interview, Mr. Padro stated that consolidation did have a pest problem due to construction and the Amtrak line right behind the buildings. The consolidation reported that, on average, 240 compactor bags (40 lbs. bags)  are disposed of from Butler daily.  There are four exterior compactors at this consolidation that are all new and in working condition. 

Butler reports that if necessary, it can receive trash from nearby developments like Morris Houses and Forest Houses.  According to the interview, there are external sources of trash and bulk waste illegally dumped at this site. When it happens, it is from nearby restaurants. Mr. Padro said the most significant obstacles Butler faces regarding waste management are the AWS schedule and tenants not placing trash in the proper locations. He also stressed that he has been working at Butler since 1999 and nothing has changed. He said the only thing that will help the trash situation at Butler is more staff.

\textbf{Additional Context}  

In a June 24, 2020 report, the Monitor Cleanliness Team gave Butler Houses a D rating.  

 