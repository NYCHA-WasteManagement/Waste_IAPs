

\textbf{Inspection and Collection Requirement}

The consolidation appeared to be in compliance with the inspection and collection requirements of Paragraph 45 of the HUD agreement. Compliance could not conduct a site visit during the 2019-2020 period; however, in a phone interview conducted in the summer of 2020, the consolidation reported the following conditions.

The Property Manager, Chukwuma Ndukah, reported that Throggs Neck does not have enough staff to correct observed deficiencies and caretakers cannot usually complete all of their tasks in a day. NYCHA caretakers pick up trash inside the buildings at least once a day, including weekends. NYCHA caretakers also conduct ground inspections and pick up litter multiple times a day, including weekends. Staff begins collecting trash 6:00 AM and ends at 4:00 PM daily.

\textbf{Removal or Storage Requirement}

The consolidation appeared to be in compliance with the  removal or storage requirement of Paragraph  45 of the HUD Agreement because it has containers in the form of exterior compactors to store waste in a manner that prevents pests on the days DSNY does not come to pick up waste. Based on the same summer of  2020 survey, the consolidation reported the following conditions.

Throggs Neck reported at the time of the survey that DSNY comes when called for collection. The consolidation also reported that it received 17 bulk tickets a month for the removal of bulk waste. Bulk trash sits in a yard with an exterior container before being picked up by the vendor. In terms of storage, residents of this consolidation have access to trash chutes and may drop their waste at 18 additional sites on the premises. After the trash is collected from the drop-off sites, it is placed in an exterior compactor. Tenants are asked by management to leave their garbage on development grounds if they are unable or choose not to use the chutes. Most tenants dispose of their trash using trash chutes or drop-off sites. Once waste is collected from the grounds, waste is stored in interior compactors. 

The consolidation reported that, on average, 100-200 compactor bags (40 lbs. bags) are disposed of from Throggs Neck daily. There are six exterior compactors at this consolidation with one waiting for repairs by a vendor. There are three 30-yard containers and 71 interior compactors. All interior compactors are functioning except for three at 515 Calhoun, 2789 Schley and 2787 Schley.

According to the survey, there are unknown external sources of trash and bulk waste illegally dumped at this site. Mr. Ndukah reports that there are infestations of opossums, racoons and squirrels. He also noted that the main obstacle facing the consolidation is the AWS system, which leaves them short staffed.

3. \textbf{Additional Context} 

In a June 24, 2020 report, the Monitor Cleanliness Team gave Throggs Neck an A rating. 