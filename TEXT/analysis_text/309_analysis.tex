
\textbf{Inspection and Collection Requirement}

The consolidation appeared to be in compliance with the inspection and collection requirements of Paragraph 45 of the HUD agreement. Compliance could not conduct a site visit during the 2019-2020 period; however, in a phone interview conducted in the summer of 2020, the consolidation reported the following conditions.

The Superintendent, Hector Colon, reported that Fort Washington Ave Rehab does not have enough staff to correct observed deficiencies and caretakers cannot usually complete all of their tasks in a day. NYCHA caretakers pick up trash inside the buildings three times a day, including weekends. NYCHA caretakers also conduct ground inspections and pick up litter three times a day, including weekends. Staff begins collecting trash at 6:00 AM and ends at 7:15 PM daily. 

\textbf{Removal or Storage Requirement}

The developments at this consolidation dispose of their waste at the curbside and because DSNY does not pick up from the curb everyday there is a high likelihood that this site is not in compliance as they cannot store waste in an exterior compactor on days when DSNY cannot pick up.

Fort Washington Rehab Ave reported at the time of the survey that DSNY comes three days a week. The consolidation also reported that it received four bulk tickets a month for the removal of bulk waste. Bulk trash sits in a yard with an exterior container before being picked up by the vendor. In terms of storage, not all residents of this consolidation have access to trash chutes. Those developments that do not have chutes have multiple drop-sites. Most tenants dispose of their trash using trash chutes or drop-sites depending on the development. Once waste is collected from the grounds, waste is placed on the curbside for collection by DSNY

The consolidation reported that, on average, fewer than 100 compactor bags (40 lbs. bags) are disposed of from Fort Washington Ave Rehab daily. There are no exterior compactors, but there is one 30-yard bulk container. Mr. Colon reported that there are 16 interior compactors, with two shutdown due to pests. 

The two buildings with shutdown interior compactors temporarily bring their waste to Harlem River Houses for storage. According to the survey, there are external sources of trash and bulk waste illegally dumped at this site, but the source is unknown. Mr. Colon cites staffing issues and continuous pest infestations as obstacles towards better waste management. The consolidation has been working to seal holes and use exterminators with help from vendors and IPMS. 