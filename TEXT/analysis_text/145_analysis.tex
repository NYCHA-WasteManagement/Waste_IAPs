

\textbf{Inspection and Collection Requirement}

In a Compliance interview conducted on October 15, 2019, the Mitchel Consolidation appeared to be in compliance with the inspection and collection requirements of Paragraph 45 of the HUD Agreement. At the time of this interview, the site was not an Alternative Work Schedule (AWS) site. 

The Supervisor of Caretakers, Mr. Lopez, reported that Mitchel does not have enough staff to correct observed deficiencies and caretakers cannot usually complete all of their tasks in a day. However, NYCHA caretakers picked up trash inside the buildings one to two times a day, including weekends. NYCHA caretakers also conducted ground inspections and picked up litter at least twice daily. Staff begins collecting trash around 10:00 AM and ends before 4:00 PM daily. 

\textbf{Removal or Storage Requirement}

At the time of the Compliance interview, the Mitchel consolidation appeared to be in compliance with the storage and removal requirement of Paragraph 45 of the HUD Agreement because it has containers in the form of exterior compactors to store waste in a manner that prevents pests on the days DSNY does not come to pick up the waste. Despite having the capacity for storage, there are problems keeping litter off the grounds which might reflect staffing needs. More follow up is needed. 

Mitchel reported at the time of the interview that DSNY comes Wednesdays and Saturdays. The consolidation also stated that it received over eight bulk tickets a month to remove bulk waste, but it needs more bulk tickets. Bulk trash sits in a yard with an exterior container before being picked up by the vendor. In terms of storage, residents of this consolidation have access to trash chutes and may drop their waste at ten additional sites on the premises. After the trash is collected from the drop-off sites, it is placed in exterior compactors. Tenants are not asked by management to leave their garbage on development grounds if they are unable or choose not to use the chutes. However, most tenants dispose of their trash by leaving it in front of each building. Once the waste is collected from the grounds, it is stored in the exterior compactors.  

A single site visit in October 2019 showed exposed litter and debris throughout the campus grounds. It also showed that waste was not stored in a way that prevents pests on that day. Furthermore, Mr. Lopez stated in the Compliance interview that Mitchel did not store garbage in a manner that prevented pests, had a pest problem, and used an exterminator to treat the area.

The consolidation reported that, on average, less than 100 compactor bags (40 lbs. bags) are disposed of from Mitchel daily. There are four exterior compactors at this consolidation that had holes in them. Mr. Lopez said he was going to contact Arrow Steel to weld the holes.  

The Mitchel consolidation reports that it can take its trash to Mill Brook Houses and Mott Haven Houses if necessary. Those developments, in addition to Patterson Houses, can bring their waste to Mitchel Houses if needed. According to the Compliance interview, there are external sources of trash and bulk waste illegally dumped at this site. When it happens, it is from local people and residents. The waste consists primarily of food, furniture, and appliances. Mr. Lopez said the most significant obstacle Mitchel faces regarding waste management is tenants continually throwing their trash in non-drop off areas.

3. \textbf{Additional Context}

In a June 24, 2020 report, the Monitor Cleanliness Team gave Mitchell Houses an F rating. 