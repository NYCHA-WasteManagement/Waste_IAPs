

\textbf{Inspection and Collection Requirement}

The consolidation appeared to be in compliance with the inspection and collection requirements of Paragraph 45 of the HUD agreement. Compliance could not conduct a site visit during the 2019-2020 period; however, in a phone interview conducted in the summer of 2020, the consolidation reported the following conditions.

The Property Manager, Toshia Smith, reported that Ocean Hill does have enough staff to correct observed deficiencies and caretakers can usually complete all of their tasks in a day. NYCHA caretakers pick up trash inside the buildings three times a day, including weekends. NYCHA caretakers also conduct ground inspections and pick up litter three times a day, including weekends. More follow up is needed to verify the hours in which staff collect waste on the premises.

\textbf{Removal or Storage Requirement}

This site is consists entirely of curbside pickup developments and because DSNY does not pick up from the curb everyday there is a high likelihood that this site is not in compliance as they cannot store waste in an exterior compactor on days when DSNY cannot pick up.

Ocean Hill reported at the time of the survey that DSNY comes Tuesdays, Thursdays and Saturdays. The consolidation also reported that it received three bulk tickets a month for the removal of bulk waste. Mr. Smith stated that the consolidation consistently requires four tickets per month. Bulk trash sits in a yard with an exterior container before being picked up by the vendor. In terms of storage, residents of this consolidation have access to trash chutes and may drop their waste at three additional sites on the premises. Tenants are asked by management to leave their garbage on development grounds if they are unable or choose not to use the chutes. Once waste is collected from the grounds, waste is stored in at drop-sites for collection by DSNY. Most tenants dispose of their trash at the drop-sites instead of using the trash chutes

The consolidation reported that, on average, fewer than 100 compactor bags (40 lbs. bags)  are disposed of from Ocean Hill daily. There are non exterior compactors, but there are four interior compactors, which are all accessible. There is also one 30-yard bulk waste container.

According to the survey, there are external sources of trash and bulk waste illegally dumped at this site. The source of the illegal dumping comes from new construction in the area and is placed on site after work hours. Mr. Smith reports that resident placement of trash is a main obstacle to waste management. The additional Caretaker X's has allowed for pick up of trash seven days a week. He believes that waste collection should increase to four days a week. Pests do not appear to be a problem according to Mr. Smith because there is frequent extermination.