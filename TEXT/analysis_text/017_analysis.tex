
\textbf{Inspection and Collection Requirement}

Johnson Consolidation is in compliance with the inspection and collection requirement of Paragraph 45 of the HUD Agreement. At the time of the Compliance Interview, conducted on October 8, 2019, Johnson was not an AWS site and not in compliance. However, it was confirmed on July 7, 2020, that Johnson is now an AWS site and it inspects the grounds and collects trash seven days a week. The Supervisor of Grounds, Serena Dowe, reported that they did not have enough staff to correct all observed deficiencies, but caretakers are usually able to complete all of their tasks in a day. NYCHA caretakers picked up trash inside the buildings one to two times a day, including weekends. However, NYCHA caretakers conducted ground inspections and picked up litter three to four times a day but excluded weekends, and thus made Johnson not in compliance based on the October interview. Staff begins collecting trash between 8:00 AM -- 10:00 AM and ends before 4:00 PM daily.

\textbf{Removal or Storage Requirement}

Johnson Consolidation was not in compliance with the storage and removal requirement of Paragraph 45 of the HUD Agreement at the time of the Compliance Interview. Even though Johnson should have the capacity to store waste in a manner that prevents waste (e.g., exterior compactors, trash bins, recycling bins), waste was not being properly stored based on self-reporting and a site visit. Follow up is needed to understand if this is still the case. 

DSNY comes Wednesdays and Sundays. An average of seven to eight bulk tickets are created each week for the removal of bulk waste. Bulk trash sits in a yard with an exterior container before being picked up by the vendor.

In terms of storage, in addition to disposing of litter into interior trash chutes, residents of this consolidation may drop their waste at six additional sites on the premises. After the trash is collected from the drop-off sites, it is placed in an exterior compactor. Tenants are not asked by management to leave their garbage on development grounds if they choose not to use the chutes. Most tenants dispose of their trash using the trash chutes or leaving it curbside. Waste is stored in exposed trash bags in front of the building, curbside, before being moved off-site. Once the garbage is picked up by caretakers, it is taken to a drop-site to be picked up by sanitation workers. A single site visit on October 8, 2019 showed exposed trash and litter on the grounds throughout the day. It also showed that waste was not stored in a way that prevents pests on that day. Ms. Dowe stated in the Compliance Interview that Johnson Houses did not have a pest problem because an exterminator has treated the property prior.

Johnson Consolidation has one bulk container and 17 interior compactor rooms that are all in service and accessible as of July 7, 2020. On average, 100 -- 200 compactor bags (40 lbs. bags) and three to four 2-yard containers are disposed of from Johnson daily. There are two exterior compactors at this location, of which at least one has a hole in it that requires welding. Ms. Dowe stated that she intended to reach out to contact Arrow Steel to weld the hole.

Johnson reports that if necessary, it can take its trash to Clinton Houses, Taft Houses, and Carver Houses and vice versa. According to the Compliance Interview, there are rarely external sources of trash and bulk waste illegally dumped at this site. If it ever happens, it is from construction sites. Ms. Dowe also stated that meeting with residents and having recycling bins and dog waste bins were the primary ways to improve waste management at Johnson Houses. 

In a June 24, 2020 report, the Monitor Cleanliness Team gave Johnson Houses a B/C rating.