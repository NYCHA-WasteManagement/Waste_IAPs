1010 East 178th Street Analysis: 

\textbf{Inspection and Collection Requirement} 

 

The consolidation appeared to be in compliance with the inspection and collection requirements of Paragraph 45 of the HUD agreement. Compliance could not conduct a site visit during the 2019-2020 period; however, in a phone interview conducted in the summer of 2020, the consolidation reported the following conditions.

The Property Manager, Angelinah Adegboyega, reported that the 1010 East 178th Street consolidation does have enough staff to correct observed deficiencies, but caretakers can usually complete all of their tasks in a day. NYCHA caretakers pick up trash inside the buildings three to four times a day, including weekends. NYCHA caretakers also conduct ground inspections and pick up litter at least twice daily. Staff begins collecting trash around 6:30 AM and ends at 5:00 PM daily. 

 

\textbf{Removal or Storage Requirement} 

 

This site is at least partially curbside. Because DSNY does not pick up from the curb every day, there is a high likelihood that this site is not in compliance as they cannot store waste in an exterior compactor at all developments on days when DSNY cannot pick up.  Curbside developments at this consolidation include 1010 East 178th Street and Twin Parks East (Site 9). The survey stated that waste was all taken to the exterior compactor 

 

The consolidation reported at the time of the survey that DSNY comes for 1010 East 178th Street on Mondays and Thursdays; Twin Parks on Tuesdays, Thursdays, and Saturdays; and Monterey whenever the exterior compactor is full. This consolidation receives five bulk tickets a month, but sometimes needs more. Bulk trash sits in a yard with an exterior container at Twin Parks (also known as 2070 Clinton Avenue) before being picked up by the vendor.  In terms of storage, residents of this consolidation have access to trash chutes and may drop their waste at three additional sites on the premises. Most tenants dispose of their trash by leaving it in front of the buildings at the drop-off sites. Once the waste is collected from the grounds, it is stored in the exterior compactors. More follow up is needed to determine if Twin Parks East and 1010 East 178th can store their waste in a manner that prevents pests.  

 

Ms. Adegboyega stated in the survey that the consolidation did not have a pest problem. The consolidation reported that, on average, 35 -- 40 compactor bags (40 lbs. bags)  are disposed of from the consolidation daily. There is one exterior compactor at this consolidation with no holes but requires replacement. 

This consolidation does bring its waste to other developments nor accept trash from other developments. According to the survey, there are external sources of trash and bulk waste illegally dumped at this site. When it happens, it is from local people dumping household bulk. According to Ms. Adegboyega, the most significant obstacle the consolidation faces in waste management is the occasional shortage of dump tickets and illegal dumping. Also. She stated that best thing Management/Operations has done for this consolidation to improve trash management is to give it extra dump tickets when available.