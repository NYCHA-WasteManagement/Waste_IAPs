 

\textbf{Inspection and Collection Requirement} 

 

The consolidation appeared to be in compliance with the inspection and collection requirements of Paragraph 45 of the HUD Agreement. Compliance could not conduct a site visit during the 2019-2020 period; however, in a survey conducted in the summer of 2020, the consolidation reported the following conditions.

The Property Manager, Dean Robinson, reported that the South Beach consolidation does not have enough staff to correct observed deficiencies. Still, caretakers can usually complete all of their tasks in a day. NYCHA caretakers pick up trash inside the buildings two times a day, including weekends. NYCHA caretakers also conduct ground inspections and pick up litter at least twice daily. The staff begins collecting trash at 8:00 AM and ends at 4:00 PM daily.

\textbf{Removal or Storage Requirement}  

 

This site is at least partially curbside. Because DSNY does not pick up from the curb every day from the curbside site, there is a high likelihood that this site is not in compliance as they cannot store waste in an exterior compactor on days when DSNY cannot pick up.  Based on the same summer of  2020  survey, the consolidation reported the following conditions. The curbside development at this consolidation is the New Lane Area. 

 

South Beach reported at the time of the survey that DSNY comes Tuesdays and Fridays. The consolidation also stated that it received five bulk tickets a month for the removal of bulk waste.  Bulk trash sits in a yard with an exterior container before being picked up by the vendor. In terms of storage, residents of this consolidation have access to trash chutes and may drop their waste at 15 additional sites on the premises. After the trash is collected from the drop-off sites, it is placed in the exterior compactor. Most tenants dispose of their trash by using the trash chutes and drop sites. Once the waste is collected from the grounds, waste is stored in the exterior compactor. Since New Lane Area is a curbside site, more information is needed to determine if the trash is stored in a way that is in compliance with Paragraph 45.  

 

Mr. Robinson stated in the survey that consolidation did not have a pest problem. The consolidation reported that, on average, 100 compactor bags (40 lbs. bags)  are disposed of from South Beach daily.  There is one exterior compactor at this consolidation that was in good condition at the time of the survey. 

South Beach does not take its waste to any other developments nor accept waste from developments. According to the survey, there are no external sources of trash and bulk waste illegally dumped at this site. Mr. Robinson said the most significant obstacle South Beach faces regarding waste management is being short-staffed and the AWS schedule. He also stated the best thing Management/Operations has done for trash management is to send out newsletters and flyers to the residents about proper trash disposal.  

\textbf{Additional Context}  

In a June 24, 2020 report, the Monitor Cleanliness Team gave South Beach Houses a B+ rating and New Lane Area an A rating.  

 