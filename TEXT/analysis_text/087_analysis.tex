 

\textbf{Inspection and Collection Requirement} 

 

The consolidation appeared to be in compliance with the inspection and collection requirements of Paragraph 45 of the HUD agreement. Compliance could not conduct a site visit during the 2019-2020 period; however, in a survey conducted in the summer of 2020, the consolidation reported the following conditions. 

The Superintendent, Juan Diaz, reported that the Grant consolidation does not have enough staff to correct observed deficiencies, and caretakers cannot usually complete all of their tasks in a day since AWS. NYCHA caretakers pick up trash inside the buildings two times a day, including weekends. NYCHA caretakers also conduct ground inspections and pick up litter at least twice daily. Staff begins collecting trash around 6:00 AM and ends before 4:00 PM daily.

\textbf{Removal or Storage Requirement} 

The consolidation appeared to be in compliance with the removal or storage requirement of Paragraph  45 of the HUD Agreement because it has containers in the form of exterior compactors to store waste in a manner that prevents pests on the days DSNY does not come to pick up waste. Based on the same summer of  2020 survey, the consolidation reported the following conditions.

 

Grant reported at the time of the survey that DSNY comes when the compactors are full. The consolidation also stated that it received 11 bulk tickets a month for the removal of bulk waste.  Bulk trash sits in a yard with an exterior container before being picked up by the vendor. In terms of storage, residents of this consolidation have access to trash chutes and may drop their waste at additional sites on the premises. After the trash is collected from the drop-off sites, it is placed in exterior compactors. Tenants are asked by management to leave their garbage on development grounds if they choose not to or are unable to use the chutes. After caretakers collect waste from the grounds, it is stored in exterior compactors. 

 

In the survey, Mr. Diaz stated that Grant did have a pest problem, but since implementing IMP under the Mayor's Rat Reduction Program, it is being treated regularly. 

The consolidation reported that, on average, 100 -- 200 compactor bags (40 lbs. bags)  are disposed of from Grant daily. There are four exterior compactors at this consolidation. Two are new, but the other two are beyond repair. Mr. Diaz feels they should be condemned as the costs of fixing them outweigh the costs of new ones. The contractors said they could not be welded anymore.

According to the survey, there are external sources of trash and bulk waste illegally dumped at this site. When it happens, it is from nearby commercial stores. Mr. Diaz stated that the most prominent obstacle Grant faces regarding waste management is the amount of garbage is thrown on the grounds and out the windows.  

\textbf{Additional Context} 

In a June 24, 2020 report, the Monitor Cleanliness Team gave Grant Houses an A rating. 