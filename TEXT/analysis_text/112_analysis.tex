Wilson Analysis: 

\textbf{Inspection and Collection Requirement} 

 

The consolidation appeared to be in compliance with the inspection and collection requirements of Paragraph 45 of the HUD Agreement. Compliance could not conduct a site visit during the 2019-2020 period; however, in a survey conducted in the summer of 2020, the consolidation reported the following conditions.

The Property Manager Supervisor, Florian Santiago, reported that the Wilson consolidation does not have enough staff to correct observed deficiencies. However, caretakers can usually complete all of their tasks in a day. NYCHA caretakers pick up trash inside the buildings four times a day, including weekends. NYCHA caretakers also conduct ground inspections and pick up litter at least twice daily. The staff begins collecting trash at 6:00 AM and ends at 5:30 PM daily.

\textbf{Removal or Storage Requirement} 

The consolidation appeared to be in compliance with the removal or storage requirement of Paragraph  45 of the HUD Agreement because it has containers in the form of exterior compactors to store waste in a manner that prevents pests on the days DSNY does not come to pick up the waste. Based on the same summer of  2020  survey, the consolidation reported the following conditions.

 

Wilson reported that at the time of the survey DSNY comes at least every two days. The consolidation also stated that it received six bulk tickets a month for the removal of bulk waste. Bulk trash sits in a yard with an exterior container before being picked up by the vendor. In terms of storage, residents of this consolidation have access to trash chutes and may drop their waste at additional sites on the premises. Most tenants dispose of their trash by leaving it on the grounds behind the buildings. Once the waste is collected from the grounds, it is stored in exterior compactors.  

 

Mr. Santiago stated in the survey that consolidation did not have a pest problem. The consolidation reported that, on average, 55 compactor bags (40 lbs. bags)  are disposed of from Wilson daily. There are two exterior compactors at this consolidation. Both have been welded several times and #2 needs welding again.

Wilson reports that it does not take its waste to, nor accept waste from, any other developments. According to the survey, there are no external sources of trash and bulk waste illegally dumped at this site. According to the survey, there are external sources of trash and bulk waste illegally dumped at this site. When it happens, it is household garbage and furniture. Mr.  Santiago said the most significant obstacle Wilson faces regarding waste management is residents not disposing of trash where they are supposed to. He also stated the best thing Management/Operations has done for trash management is to provide extra dump tickets for more frequent pick-ups. 

\textbf{Additional Context}  

In a June 24, 2020 report, the Monitor Cleanliness Team gave Wilson Houses a D rating. White Houses and Metro North Plaza have not yet been graded. 

 