

\textbf{Inspection and Collection Requirement}

In a compliance interview conducted on October 9, 2019, the Jefferson Consolidation appeared to be in compliance with the inspection requirements of Paragraph 45 of the HUD agreement. However, the consolidation reported insufficient staff to meet the collection portion of the requirements. At the time of this interview, the site was not an Alternative Work Schedule (AWS) site. 

The Supervisor of Grounds, Gerardo Rivera, reported that Jefferson does not have enough staff to correct observed deficiencies, but caretakers can usually complete all of their tasks in a day. NYCHA caretakers picked up trash inside the buildings more than four times a day, including weekends. NYCHA caretakers also conducted ground inspections and picked up litter at least one to two times a day, including weekends. Staff begins collecting trash between 8:00 AM -- 10:00 AM and ends before 4:00 PM daily.

\textbf{Removal or Storage Requirement}

At the time of the compliance interview, the Jefferson Consolidation appeared to be in compliance with the storage and removal requirement of Paragraph 45 of the HUD Agreement because it has containers in the form of exterior compactors to store waste in a manner that prevents pests on the days DSNY does not come to pick up waste.

The Jefferson Consolidation reported at the time of the interview that DSNY comes Tuesdays and Fridays. They also call DSNY when their compactors are full. The consolidation also reported that it received nine bulk tickets a month for the removal of bulk waste. Mr. Rivera stated that this was an insufficient amount and that at least 12 per month would be sufficient.  In terms of storage, residents of this consolidation have access to trash chutes and may drop their waste at 36 additional sites on the premises. After the trash is collected from the drop-off sites, it is placed in an exterior compactor. Tenants are asked by management to leave their garbage on development grounds if they choose not to use the chutes. Most tenants dispose of their trash outside of their buildings. Waste is stored in exterior compactors. 

The consolidation reported that, on average, 100-200 compactor bags (40 lbs. bags) and three to four 2-yard containers are disposed of from Jefferson daily. There are four exterior compactors at this consolidation.

According to the compliance interview, there are external sources of trash and bulk waste illegally dumped at this site. When it happens, it is from construction companies, nearby restaurants and passerby's. At the time of the interview, the consolidation reported understaffing on weekends as an obstacle to waste management. More follow up is required to see if that is still the situation since becoming an AWS site.

3. \textbf{Additional Context} 

In a June 24, 2020 report, the Monitor Cleanliness Team gave the Jefferson Consolidation a C rating. 