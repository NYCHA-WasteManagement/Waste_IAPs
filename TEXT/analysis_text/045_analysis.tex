Sedgwick Analysis: 

\textbf{Inspection and Collection Requirement} 

 

The Sedgwick consolidation appeared to be in compliance with the inspection and collection requirements of Paragraph 45 of the HUD agreement. Compliance could not conduct a site visit during the 2019-2020 period; however, in a survey conducted in the summer of 2020, the consolidation reported the following conditions.

The Supervisor of Caretakers, Eli Balaguer, reported that the Sedgwick Consolidation does not have enough staff to correct observed deficiencies and caretakers cannot usually complete all of their tasks in a day. NYCHA caretakers pick up trash inside the buildings four times a day, including weekends. Additionally, NYCHA caretakers also conduct ground inspections and pick up litter at least twice daily. Staff begins collecting trash around 6:00 AM and ends around 6:00 PM daily. 

 

\textbf{Removal or Storage Requirement} 

The Sedgwick consolidation appeared to be in compliance with the removal or storage requirement of Paragraph  45 of the HUD Agreement because it has containers in the form of exterior compactors to store waste in a manner that prevents pests on the days DSNY does not come to pick up waste. Based on the same summer of  2020 phone interview, the consolidation reported the following conditions.

  

The Sedgwick consolidation reported at the time of the interview that DSNY comes Thursdays and Saturdays. The consolidation also stated that it received five bulk tickets a month for the removal of bulk waste. Bulk trash sits in a yard with an exterior container before being picked up by the vendor.  In terms of storage, residents of this consolidation have access to trash chutes and may drop their waste at three additional sites on the premises. After the trash is collected from the drop-off sites, it is placed in exterior compactors. Tenants are asked by management to leave their garbage on development grounds, at the drop-sites, if they choose not to use the chutes. Most tenants dispose of their trash using trash chutes. After waste is collected, it is stored in exterior compactors.

 

In the survey, Mr. Balaguer stated that the Sedgwick consolidation did not have a pest problem and was able to store its waste in a way that prevents pests. The consolidation reported that, on average, 55 -- 60 compactor bags (40 lbs. bags) are disposed of from Sedgwick daily. There are two exterior compactors at this consolidation that are in good condition at the time of the survey.

Sedgwick reports that it does not take its waste to any other developments nor take garbage from other developments.  According to the survey, external sources of trash and bulk waste are illegally dumped at this site. When it happens, it is mainly from construction companies.

\textbf{Additional Context}

In a June 24, 2020 report, the Monitor Cleanliness Team gave Sedgwick Houses a C rating and West Tremont Avenue -- Sedgwick Avenue area an A rating.  