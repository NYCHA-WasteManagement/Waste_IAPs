
\textbf{Inspection and Collection Requirement}

The consolidation appeared to be in compliance with the inspection and collection requirements of Paragraph 45 of the HUD agreement. Compliance could not conduct a site visit during the 2019-2020 period; however, in a phone interview conducted in the summer of 2020, the consolidation reported the following conditions.

The Property Manager, Jokotade Shanu, reported that Sack Wern does not have enough staff to correct observed deficiencies, but caretakers can usually complete all of their tasks in a day. NYCHA caretakers pick up trash inside the buildings three times a day, including weekends. NYCHA caretakers also conduct ground inspections and pick up litter three times a day, including weekends. Staff begins collecting trash at 8:15 AM  and ends before 4:00 PM daily. 

\textbf{Removal or Storage Requirement}

This developments of this consolidation have their waste collected from the curbside and because DSNY does not pick up from the curb everyday there is a high likelihood that this site is not in compliance as they cannot store waste in an exterior compactor on days when DSNY cannot pick up.

Sack Wern reported at the time of the survey that DSNY comes Monday through Friday. The consolidation also reported that it received seven bulk tickets a month for the removal of bulk waste. Bulk trash sits in a yard with an exterior container before being picked up by the vendor. In terms of storage, residents of this consolidation have access to trash chutes and may drop their waste at seven additional sites on the premises. Waste remains at the drop-site for collection by DSNY. Tenants are asked by management to leave their garbage on development grounds if they are unable or choose not to use the chutes. Most tenants dispose of their trash using trash chutes. Once waste is collected from the grounds, waste is placed on the curbside for collection. 

The consolidation reported that, on average, fewer than 100 compactor bags (40 lbs. bags) are disposed of from Sack Wern daily. There are no exterior compactors for containerization, but there are two 30-yard bulk containers. Ms. Shanu also reported that there are seven interior compactors, all of which are accessible. 

According to the survey, there are high quantities of external trash and bulk waste illegally dumped at this site. The source of this dumping comes mostly from neighboring residential buildings and consists of wood, metal and raw garbage. Ms. Shanu reported needing more staff to address waste management efforts. Sack Wern has actively combated pests through extermination and blocking entrances through windows and basements of buildings.

3. \textbf{Additional Context} 

In a June 24, 2020 report, the Monitor Cleanliness Team gave Sack Wern an A rating. 