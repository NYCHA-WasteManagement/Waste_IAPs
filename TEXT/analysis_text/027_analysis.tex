
\textbf{Inspection and Collection Requirement} 

Smith Consolidation is in compliance with inspection and collection requirement of Paragraph 45 of the HUD Agreement, according to a Compliance Interview conducted on October 24, 2019. The Supervisor Caretaker, Frederick Brown, reported that staff patrols the grounds for cleaning and maintenance and has sufficient manpower to correct most observed deficiencies. Mr. Brown stated that caretakers are usually able to complete all of their tasks in a day.  NYCHA caretakers pick up trash inside the buildings one to two times a day, including weekends. They conduct ground inspections more than four times a day, including weekends. They pick up litter from the ground one to two times a day. Daily trash collection begins between 8:00 AM -- 10:00 AM and ends before 4:00 PM. 

 

\textbf{Removal or Storage Requirement} 

Smith Consolidation is in compliance with the storage and removal of Paragraph 45 of the HUD Agreement because they are able to store waste in a manner that prevents pests (e.g., exterior compactors). 

DSNY comes when the compactors are full to pick-up trash, usually three to four times a week. At the time of the Compliance Interview, Smith reported having an insufficient amount of bulk tickets. An average of five to six bulk tickets are created each month for the removal of bulk waste but says it could use more in the range of eight to ten bulk tickets a month. Bulk trash sits in a yard with an exterior container before it is picked up by the vendor.  

 

In terms of storage and disposing of litter into interior trash chutes, there are no additional drop-off sites for residents to use. Tenants are not asked by management to leave their garbage curbside or in  front of their buildings if they choose not to use the chutes, but most tenants leave their trash in front of their buildings anyways. Waste is stored in front of each building in exposed trash bags at the premises before being moved offsite to the exterior compactors. There are four exterior compactors at Smith. At least one had a hole that required welding at the time of the interview. They intended to reach out to Arrow Steel to mend the hole.

 

Smith has two bulk containers and 12 interior compactor rooms that were all accessible and working at the time of the Compliance Interview. Smith disposes of 100 -- 200 compactor bags (40 lbs. bags) daily. When the trash is not able to be removed from the premises, it is able to be stored in a way to prevent pests (e.g., trash bins), according to the Compliance Interview. The supervisor also stated that Smith does not have a pest problem. According to the Rat Reduction Plan, Smith has seen its rat burrows decrease from 148 to 42 as of January 2019.

Smith reports that it does not take in neighboring developments waste nor take its waste to other developments. According to the Compliance Interview, there are external sources of trash and bulk waste illegally dumped at this site from construction sites, nearby restaurants, and local people. The waste consists mostly of construction material from nearby houses, food, furniture, appliances, and sheetrock. According to Mr. Brown, the biggest obstacle the site faced is that residents bring their trash out after caretakers have already cleaned it up. 

 

In a June 24, 2020 report, the Monitor Cleanliness Team gave Smith Houses an A rating.  

 