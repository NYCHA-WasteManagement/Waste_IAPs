
\textbf{Inspection and Collection Requirement}

The consolidation appeared to be in compliance with the inspection and collection requirements of Paragraph 45 of the HUD agreement. Compliance could not conduct a site visit during the 2019-2020 period; however, in a survey conducted in the summer of 2020, the consolidation reported the following conditions.

The Property Manager, Jasmine Williams, reported that Fort Independence does not always have enough staff to correct observed deficiencies, but caretakers can usually complete all of their tasks in a day. NYCHA caretakers pick up trash inside the buildings two times a day, including weekends. NYCHA caretakers also conduct ground inspections and pick up litter two times a day, including weekends. Staff begins collecting trash at 6:00 AM and ends at 7:00 PM daily.

\textbf{Removal or Storage Requirement}

The consolidation appeared to be in compliance with the  removal or storage requirement of Paragraph  45 of the HUD Agreement because it has containers in the form of exterior compactors to store waste in a manner that prevents pests on the days DSNY does not come to pick up waste. Based on the same summer of  2020 survey, the consolidation reported the following conditions.

Fort Independence reported at the time of the survey that DSNY comes one to two times . The consolidation also reported that it received one two three bulk tickets a month for the removal of bulk waste. Bulk trash sits in a yard with an exterior container before being picked up by the vendor. In terms of storage, residents of this consolidation have access to trash chutes and may drop their waste at two additional sites on the premises. After the trash is collected from the drop-off sites, it is placed in an exterior compactor. Most tenants dispose of their trash using trash chutes or the drop-off sites. Once waste is collected from the grounds, waste is stored in exterior compactors. 

The consolidation reported that, on average, fewer than 100 compactor bags (40 lbs. bags) are disposed of from Fort Independence daily. There are two exterior compactors and one 30-yard bulk container. Ms. Williams reported that there are two functioning interior compactors as well.

Fort Independence reports that, when necessary, the Marble Hill development brings waste to this consolidation. According to the survey, there are external sources of trash and bulk waste illegally dumped at this site, mostly from local stores and churches. Ms. Williams notes that obstacles such as the AWS system, illegal dumping and the steepness of the property make waste collection difficult for caretakers. She also noted that the consolidation has taken strong measures against pests through extermination, cleaning of soiled areas and ordering tilt trucks to keep waste off the floor.