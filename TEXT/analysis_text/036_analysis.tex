
\textbf{Inspection and Collection Requirement}

In a compliance interview conducted on November 18th, 2019, the consolidation appeared to be in compliance with the inspection requirements of Paragraph 45 of the HUD agreement. The consolidation reported sufficient staff to meet the collection portion of the requirements. At the time of this interview, the site was not an Alternative Work Schedule (AWS) site. (IF PHONE CONFIRMATION: A follow up phone call to the site in the summer of 2020 confirmed that the development [WAS/WAS NOT] checking the site and removing waste seven days a week.)

---

The Supervisor of Caretakers, Robert Smith, reported that it does have enough staff to correct observed deficiencies and caretakers can usually complete all of their tasks in a day. NYCHA caretakers pick[ed] up trash inside the buildings 1-2 times a day, and once on weekends. NYCHA caretakers also conduct[ed] ground inspections and pick[ed] up litter 1-2 times a day, and once a day on weekends. Staff begins collecting trash after 10 AM and ends between 4:00 PM and 5:00 PM daily. 

\textbf{Removal or Storage Requirement}

At the time of the compliance interview, the Sheepshead Bay consolidation appeared to be in compliance with the storage and removal requirement of Paragraph 45 of the HUD Agreement because it does have containers in the form of exterior compactors to store waste in a manner that prevents pests on the days DSNY does not come to pick up waste.

---

The Sheepshead Bay consolidation reported at the time of the interview that DSNY comes on Mondays, Wednesdays, and Fridays: 5-6 times a week. The consolidation also reported that it received no bulk tickets for the removal of bulk waste, due to the fact that they have a bulk crusher. Bulk trash sits in a yard with an exterior container before being processed by the bulk crusher. In terms of storage, residents of this consolidation do not have access to trash chutes and may drop their waste at 9 additional sites on the premises. After the trash is collected from the drop-off sites, it is placed in the exterior compactor. Tenants are asked by management not to leave their garbage on development grounds if they choose not to use the chutes. Most tenants dispose of their trash in front of their buildings with no bins. Waste is stored in the exterior compactor. 

A single site visit on November 18th, 2019 showed exposed trash on the grounds upon arrival, but little to no trash upon departure. Recycling bins are placed throughout the site with open lids. It also showed that waste was not stored in a way that prevents pests on that day because pests can access trash in the compactor yard. Robert Smith stated in the Compliance Interview that Sheepshead Bay did have a pest problem. Exterminators have been contacted to remedy the situation.

The consolidation reported that on average, 100-200 compactor bags (40 lbs. bags) are disposed of from Sheepshead Bay daily. There are five exterior compactors at this consolidation, with at least one having a hole in need of welding. Robert Smith stated that he intended to reach out to contact Arrow Sheet to weld the hole.

According to the Compliance interview, there are external sources of trash and bulk waste illegally dumped at this site. When it happens, it is from construction and passersby. 

3. \textbf{Additional Context} 

In a June 9th, 2020 report, the Monitor Cleanliness Team gave the Nostrand and Sheepshead Bay developments a B+/A and a B rating, respectively. 