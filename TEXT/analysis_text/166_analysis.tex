

\textbf{Inspection and Collection Requirement}

The consolidation appeared to be in compliance with the inspection and collection requirements of Paragraph 45 of the HUD agreement. Compliance could not conduct a site visit during the 2019-2020 period; however, in a questionnaire conducted in the summer of 2020, the consolidation reported the following conditions.

The Superintendent, David Rios, reported that Carey Gardens do not have enough staff to correct observed deficiencies and caretakers can usually complete all of their tasks in a day. NYCHA caretakers pick up trash inside the buildings multiple times a day as needed, including weekends. NYCHA caretakers also conduct ground inspections and pick up litter multiple times a day as needed, including weekends. Staff begins collecting trash at 6:00 AM and ends at 4:14 everyday.

\textbf{Removal or Storage Requirement}

This consolidation partially has its waste collected from the curbside and because DSNY does not pick up from the curb everyday there is a high likelihood that this site is not in compliance as they cannot store waste in an exterior compactor on days when DSNY cannot pick up. The Haber Houses are the only development that does not store waste in exterior compactors.

Carey Gardens reported at the time of the survey that DSNY comes three days a week. The consolidation also reported that it received nine bulk tickets a month for the removal of bulk waste. In terms of storage, residents of this consolidation have access to trash chutes and may not drop their waste at additional sites on the premises. Most tenants dispose of their trash in the front or back of their building. Once waste is collected from the grounds, waste is stored in an exterior compactor or brought to the curbside depending on the development.

The consolidation reported that, on average, 100-200 compactor bags (40 lbs. bags)  are disposed of from Carey Gardens daily. There are two exterior compactors at this consolidation and two 30-yard containers. Mr. Rios also reported that there are seven interior compactors.

The consolidation reported that waste is brought from Haber Houses to the Carey Gardens development when waste cannot be collected from the curbside by DSNY. According to the survey, there are external sources of trash and bulk waste illegally dumped at this site. The source of the illegal dumping is unknown, but consists mostly of trash and bulk waste. Management sends out notices to residents to improve trash management and uses exterminators once a week to deal with ongoing pest problems.

3. \textbf{Additional Context} 

In a June 24, 2020 report, the Monitor Cleanliness Team gave Carey Gardens a C+ rating. 