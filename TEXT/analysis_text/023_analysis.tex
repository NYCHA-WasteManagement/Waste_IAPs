


\textbf{Inspection and Collection Requirement}s 

Wald Consolidation is in compliance with the inspection and collection requirement of Paragraph 45 of the HUD Agreement, according to a Compliance Interview conducted on October 04, 2019. The Supervisor of Grounds, Victor Matri, reported that staff patrols the grounds for cleaning and maintenance and has sufficient manpower to correct all observed deficiencies. Mr. Matri stated that caretakers are usually able to complete all of their tasks in a day.  NYCHA caretakers pickup trash inside the buildings and conduct ground/building inspection one to two times a day, including weekends. Additionally, caretakers pickup litter from the ground one to two times a day. Daily trash collection begins between 8:00 AM -- 10:00 AM and ends before 4:00 PM. 

 

\textbf{Removal or Storage Requirement} 

Wald Consolidation is in compliance with the storage and removal requirement of Paragraph 45 of the HUD Agreement because are able to store their waste in a way that prevented pests. At the time of the interview, they were not able to store their waste in a manner that prevented pests and were not in compliance, but since the time of the Compliance Interview, Wald has three new exterior compactors which makes them in compliance. 



DSNY comes when compactors are full to pick-up trash, usually three to four times a week. An average of five to six bulk tickets are created each month for the removal of bulk waste. Bulk trash sits in a yard with an exterior container before it is picked up.  

 

In terms of storage and disposing of litter into interior trash chutes, residents of this consolidation may drop their waste at seventeen additional sites on the premises. Once collected from the drop-off site, trash is taken to EZ Pack containers (2-yard containers). Tenants are not asked by management to leave their garbage curbside or in front of the building if they choose not to use the chutes, but most tenants leave their trash curbside anyways. There are now three new exterior compactors that were installed late November 2019. 

 

Wald has two bulk containers and sixteen interior compactor rooms that were all accessible at the time of the Compliance Interview. However, one was not working (709 FDR Drive) due to flooding at that time. Follow up is needed to determine the current status of the interior compactor. Wald disposes of 100 -- 200 compactor bags (40 lbs. bags) daily. At the time of the interview, Wald self-reported that when trash is not able to be removed from the premises, it is not able to be stored in a way to prevent pests. If the trash is not moved from the premises, it goes to EZ Pack containers. However, this was before the procurement of three exterior compactors. Mr. Matri stated that Wald does not have a pest problem. It used to be worse, but there has been much improvement. This is conflicting information because according to the Wald Rat Reduction Plan, Wald has seen an increase in burrows.  In June 2019, there were around 35 burrows, and by October 2019, there were about 110. By February 2020, there were 130 rat burrows. This needs more follow up.



Wald reports that, if necessary, it can take its trash to Baruch Houses, Riis Houses, and Gompers Houses and vice versa. According to the Compliance Interview, there are external sources of trash and bulk waste illegally dumped at this site. The waste consists mostly of construction material from nearby houses, food, flyers, furniture and appliances. According to Mr. Matri, the biggest obstacles the site faced are that there are no exterior compactors (which is has since been addressed) and a lack of space for dumping.

 

In a June 24, 2020 report, the Monitor Cleanliness Team gave Wald Houses a B rating.