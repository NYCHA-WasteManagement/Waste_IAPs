
\textbf{Inspection and Collection Requirement}

In a compliance interview conducted on October 16th, 2019, the consolidation appeared to be in compliance with the inspection requirements of Paragraph 45 of the HUD agreement. The consolidation reported sufficient staff to meet the collection portion of the requirements. At the time of this interview, the site was not an Alternative Work Schedule (AWS) site. 

---

The Assistant Superintendent, Samara Singleton, reported that it does have enough staff to correct observed deficiencies and caretakers can usually complete all of their tasks in a day. NYCHA caretakers picked up trash inside the buildings 1-2 times a day, including weekends. NYCHA caretakers also conducted ground inspections and picked up litter 1-2 times a day, including weekends. Staff begins collecting trash between 6:00 AM -- 8:00 AM and ends before 4:00 PM daily.

\textbf{Removal or Storage Requirement}

At the time of the compliance interview, Gowanus did not appear to be in compliance with the storage and removal requirement of Paragraph 45 of the HUD Agreement because it does not have containers in the form of exterior compactors to store waste in a manner that prevents pests on the days DSNY does not come to pick up waste. While exterior compactors are on the premises for storage, they were in need of repair at the time of the interview. 

---

Gowanus reported at the time of the interview that DSNY comes on Tuesdays, Thursdays, and Saturdays. The consolidation also reported that it received 5-6 bulk tickets for the removal of bulk waste. Ms. Singleton states that 10 are actually needed. Bulk trash sits in a yard with an exterior container at Douglas St & Hoyt St before being picked up by the vendor. In terms of storage, residents of this consolidation have access to trash chutes and may not drop their waste at additional sites on the premises. Tenants are asked by management not to leave their garbage on development grounds if they choose not to use the chutes. Most tenants dispose of their trash using shoots and nondesignated drop-off sites on the grounds. Waste is stored in the exterior compactors.

A single site visit on October 16th, 2019 showed exposed trash and bulk debris on development grounds, along with open-lid trash bins and exposed trash bags on the grounds. It also showed that waste was not stored in a way that prevents pests on that day. Samara Singleton stated in the Compliance Interview that Gowanus has a pest problem because of a flawed compactor. Contractors were in the process of treating areas on the development affected by trash at the time of the interview, according to Ms. Singleton. 

The consolidation reported that on average, 100-200 compactor bags (40 lbs. bags) are disposed of from Gowanus daily. There are three exterior compactors at this consolidation, and at least one of them has a hole that needs repairing. Ms. Singleton stated that she intended to reach out to contact NYCHA welders through TSD waste to weld the hole.

According to the Compliance interview, there are external sources of trash and bulk waste illegally dumped at this site. When it happens, it is from passersby and construction.

3. \textbf{Additional Context}

In a May 27, 2020 report, the Monitor Cleanliness Team gave Gowanus an A rating. 

*At the time of interview the consolidation noted needing more tickets. Further inquiry is needed to verify that this is still the case.