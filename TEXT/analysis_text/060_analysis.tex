
\textbf{Inspection and Collection Requirement}

In a compliance interview conducted on October 4, 2019, the Baruch Consolidation did not appear to be in compliance with the inspection requirements of Paragraph 45 of the HUD agreement. The consolidation reported insufficient staff to meet the collection portion of the requirements. At the time of this interview, the site was not an Alternative Work Schedule (AWS) site.  A follow up phone call to the site in the summer of 2020 confirmed that the development was checking the site and removing waste seven days a week.

The Supervisor of Grounds, William Orellana, reported that Baruch does not have enough staff to correct observed deficiencies, but caretakers can usually complete all of their tasks in a day. NYCHA caretakers picked up trash inside the buildings more than four times a day, not including weekends and thus they were not in compliance. NYCHA caretakers also conducted ground inspections and picked up litter more than four times a day, not including weekends and thus they were not in compliance. However, the follow up phone call confirmed that inspections and collection are conducted at least daily. Staff begins collecting trash between 8:00 AM -- 10:00 AM and ends between 4:00 PM - 5:00 PM daily. 

\textbf{Removal or Storage Requirement}

At the time of the compliance interview, Baruch appeared to be in compliance with the storage and removal requirement of Paragraph 45 of the HUD Agreement because it does have containers in the form of exterior compactors to store waste in a manner that prevents pests on the days DSNY does not come to pick up waste. 

Baruch reported at the time of the interview that DSNY comes Mondays, Wednesdays and Fridays. The consolidation also reported that it received seven to eight bulk tickets for the removal of bulk waste, which they noted as being a sufficient amount. Bulk trash sits in a yard with an exterior container before being picked up by the vendor. In terms of storage, residents of this consolidation have access to trash chutes and may drop their waste at 18 additional sites on the premises.  After the trash is collected from the drop-off sites, it is placed in an exterior compactor. Tenants are asked by management not to leave their garbage on development grounds if they choose not to use the chutes. Most tenants dispose of their trash in front of their buildings. Waste is then stored in exterior compactors. 

The consolidation reported that, on average, 100-200 compactor bags (40 lbs. bags) are disposed of from Baruch daily. There are four exterior compactors at this consolidation with at least one reported having a hole . Mr. Orellana stated that he intended to reach out to contact NYCHA to weld the hole. 

Baruch reports that if necessary, it  may receive trash from Wald Houses. According to the compliance interview, there are external sources of trash and bulk waste illegally dumped at this site. When it happens, it is from construction companies and nearby restaurants. The consolidation reported obstacles such as illegal dumping and tenants parking at drop-sites labeled as no-standing zones.

3. \textbf{Additional Context} 

In a June 24, 2020 report, the Monitor Cleanliness Team gave Baruch a D rating. 