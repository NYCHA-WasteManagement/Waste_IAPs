 Stapleton Consolidation Analysis: 

\textbf{Inspection and Collection Requirement} 

In a compliance interview conducted on December 18, 2020, the consolidation appeared to be in compliance with the inspection requirements of Paragraph 45 of the HUD agreement. However, the consolidation reported insufficient staff to meet the collection portion of the requirements. At the time of this interview, the site was not an Alternative Work Schedule (AWS) site. 

The Supervisor Caretaker, Jose Martinez, reported that it does not have enough staff to correct observed deficiencies and caretakers are  usually not able to complete all of their tasks in a day at the time of the interview. As a current AWS site, NYCHA caretakers now pick up trash inside the buildings one to two times a day, including weekends.  NYCHA caretakers also now conduct ground inspections and pick up litter one to two times a day, including weekends. Staff begins collecting trash between 8:00 AM -- 10:00 AM and end before 4:00 PM daily. 

 

\textbf{Removal or Storage Requirement} 

 

At the time of the compliance interview, Stapleton appeared to be in compliance with the storage and removal requirement of Paragraph 45 of the HUD Agreement because it does have containers in the form of exterior compactors to store waste in a manner that prevents pests on the days DSNY does not come to pick up waste. 

  

The consolidation reported at the time of the interview that DSNY comes Mondays and Thursdays. The consolidation also reported that they received five to six bulk tickets for the removal of bulk waste. They expressed that they needed more bulk tickets. Bulk trash sits in a yard with an exterior container before being picked up by the vendor.  In terms of storage, residents of this consolidation have access to trash chutes and may drop their waste at 12 additional sites on the premises. After the trash is collected from the drop-off sites, it is placed in exterior compactors. Tenants are asked by management not to leave their garbage on development grounds if they choose not to use the chutes. However, most tenants dispose of their trash using trash chutes. Waste is stored in exterior compactors.

 

A single site visit on December 18, 2019 showed the ground with exposed trash on curbside and development grounds, but it was cleaned upon departure. It also showed that waste was not stored in a way that prevents pests on that day. However, Mr. Martinez stated in the Compliance Interview that Stapleton did not have a pest problem. 

The development reported that on average, less than 100  compactor bags (40 lbs. bags)  are disposed of from Stapleton daily. There are two exterior compactors at this consolidation and reports that there are holes in them. At the time of the interview, there was no plan to fix the holes. 

Stapleton reports that if necessary, it can take its trash to other developments like South Beach Houses, but it does not receive trash from developments. According to the Compliance Interview, there are external sources of trash and bulk waste illegally dumped at this site. When it happens, it is usually from construction companies. Mr. Martinez says the main obstacle the consolidation faces is tenants throwing trash out the window and construction at the development.  

\textbf{Additional Context}

In a June 24, 2020 report, the Monitor Cleanliness Team gave Stapleton a B+/D rating.  