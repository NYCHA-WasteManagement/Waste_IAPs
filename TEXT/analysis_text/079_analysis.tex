

\textbf{Inspection and Collection Requirement}

In a compliance interview conducted on October 16, 2019, the Red Hook West Consolidation did not appear to be in compliance with the inspection requirements of Paragraph 45 of the HUD agreement. The consolidation reported insufficient staff to meet the collection portion of the requirements. At the time of this interview, the site was not an Alternative Work Schedule (AWS) site. A follow up phone call to the site in the summer of 2020 confirmed that the development was checking the grounds and removing waste seven days a week.

The Supervisor Caretaker, Robert Taylor, reported that Red Hook West does have enough staff to correct observed deficiencies, but caretakers can usually complete all of their tasks in a day. NYCHA caretakers picked up trash inside the buildings one to two times a day, not including weekends and thus they were not in compliance. NYCHA caretakers also conducted ground inspections and picked up litter one to two times a day, not including weekends and thus they were not in compliance. The follow up phone call confirmed that inspections and collections were conducted at least once a day. Staff begins collecting trash after10:00 AM and ends before 4:00 PM daily. 

\textbf{Removal or Storage Requirement}

At the time of the compliance interview, the Red Hook West Consolidation appeared to be in compliance with the storage and removal requirement of Paragraph 45 of the HUD Agreement because it has containers in the form of exterior compactors to store waste in a manner that prevents pests on the days DSNY does not come to pick up waste. 

Red Hook West reported at the time of the interview that DSNY comes Tuesdays and Fridays. The consolidation also reported that it received five to six bulk tickets a month for the removal of bulk waste. Bulk trash sits in a yard with an exterior container before being picked up by the vendor. In terms of storage, residents of this consolidation have access to trash chutes and may drop their waste at eight additional sites on the premises. After the trash is collected from the drop-off sites, it is placed in exterior compactors Tenants are not asked by management  to leave their garbage on development grounds if they choose not to use the chutes. Most tenants dispose of their trash in front of their buildings. Waste is stored in exterior compactors. 

A single site visit on October 16, 2019 showed satisfactory ground conditions. However, trash and recycling bins were not found throughout the entire site and where there were bins, they did not have lids on them. For this reason, the inspectors determined that waste was not stored in a way that prevents pests on that day. Mr. Taylor stated in the compliance interview that Red Hook West did not have a pest problem.

The consolidation reported that, on average, 100-200 compactor bags (40 lbs. bags) are disposed of from Red Hook West daily. There are two exterior compactors at this consolidation. 

According to the compliance interview, there are external sources of trash and bulk waste illegally dumped at this site. When it happens, it is mostly from passerby's. Mr. Taylor sites a short staff and tenants dropping off waste in incorrect areas as main obstacles towards better waste management. New garbage cans were ordered which may alleviate the issues observed in the site visit.

3. \textbf{Additional Context} 

In a June 24, 2020 report, the Monitor Cleanliness Team gave Red Hook West a B+/C- rating. 