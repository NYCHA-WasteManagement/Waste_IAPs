

\textbf{Inspection and Collection Requirement}

The consolidation appeared to be in compliance with the inspection and collection requirements of Paragraph 45 of the HUD agreement. Compliance could not conduct a site visit during the 2019-2020 period; however, in a survey conducted in the summer of 2020, the consolidation reported the following conditions.

The Property Manager, Isaac Perry, reported that Woodside does not have enough staff to correct observed deficiencies and caretakers cannot usually complete all of their tasks in a day. NYCHA caretakers pick up trash inside the buildings once a day, including weekends. NYCHA caretakers also conduct ground inspections and pick up litter once a day, including weekends. Staff begins collecting trash at 8:00 AM and ends at 4:00 PM daily.

\textbf{Removal or Storage Requirement}

The consolidation appeared to be in compliance with the  removal or storage requirement of Paragraph  45 of the HUD Agreement because it has containers in the form of exterior compactors to store waste in a manner that prevents pests on the days DSNY does not come to pick up waste. Based on the same summer of  2020 survey, the consolidation reported the following conditions.

Woodside reported at the time of the survey that DSNY comes Tuesdays and Thursdays. The consolidation also reported that it received four bulk tickets a month for the removal of bulk waste. Mr. Perry reports always requesting more tickets each month. Bulk trash sits in a yard with an exterior container before being picked up by the vendor. In terms of storage, residents of this consolidation have access to trash chutes and may drop their waste at three additional sites on the premises. Most tenants dispose of their trash in the trash chutes or outside their building. Once waste is collected from the grounds, waste is stored in an exterior compactor. 

The consolidation reported that, on average, fewer than 100 compactor bags (40 lbs. bags) are disposed of from Woodside daily. There are three exterior compactors  and one 30-yard bulk container at this consolidation. Mr. Perry also reported that there are 55 interior compactors, all of which are operational.

According to the survey, there are external sources of trash and bulk waste illegally dumped at this site. The source of the illegal dumping is unknown, but comes after the caretakers have completed their shifts. The creation of drop-off sites and dispersing of leaflets has helped with waste management issues at this consolidation. Mr. Perry also reports that there is ongoing extermination of the basements. The main barrier to better waste management is understaffing, particularly understaffing of caretakers.