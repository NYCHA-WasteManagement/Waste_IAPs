
\textbf{Inspection and Collection Requirement}

Sumner Consolidation is in compliance with the inspection and collection requirement of  Paragraph 45 of the HUD Agreement, according to a Compliance Interview conducted on November 21, 2019. The Supervisor of Grounds, Dirk Jacob, reported they have sufficient manpower to correct all observed deficiencies. NYCHA caretakers conduct ground inspections and remove trash from the buildings one to two times a day, including weekends. They also pick up litter from the grounds one to two times a day. The caretakers begin picking up trash each day between 8:00 AM -- 10:00 AM and stop between 4:00 PM -- 5:00 PM. 

\textbf{Removal or Storage Requirement}

Sumner Consolidation is in compliance with the storage and removal requirement of Paragraph 45 of the HUD Agreement as they are able to store waste in a manner that prevent pests (e.g., exterior compactors).

 

DSNY comes Tuesdays, Fridays and Saturdays. An average of five to six bulk tickets are created each month for the removal of bulk waste. Bulk trash sits in a yard with an exterior container before being picked up by the vendor.

In terms of storage, in addition to disposing of litter into interior trash chutes, residents of this consolidation may drop their waste at 13 additional sites on the premises. Tenants are asked by management to leave their trash in the front of each building, either in trash cans or in exposed trash bags for pick up by caretakers if they choose not to use the chutes. Most tenants dispose of their trash using the drop-off sites. Waste is taken to one of four exterior compactors after being taken from the drop-off sites. All exterior compactors are in good shape and do not require maintenance at the time of reporting. When the trash is not removed from the premises, it is stored in a way that prevents pests (e.g., trash bins).

Sumner has two bulk containers and 31 interior compactor rooms. Of the 31 interior compactor rooms, two were inaccessible: 67 Marcus Garvey Boulevard due to pests and 987 Myrtle Avenue due to flooding. Further information is needed to see what the current statues is of the interior compactors. Sumner disposes of approximately 100 -- 200 compactor bags (40 lbs. Bags). The supervisor also stated that Sumner did not have a pest problem and treated any pest problems by collapsing the burrows. According to the Sumner Rat Reduction Plan, in the summer of 2018, the site had 61 rat burrows, but as of March 13, 2019, they have very few burrows. Further research is needed to quantify the number of borrows.

Sumner reports that, if necessary, they can take the trash from the developments to Tompkins Houses, Marcy Houses, and Roosevelt Houses and vice versa. According to the compliance report, there are external sources of waste and bulk being illegally dumped at this site, primarily from construction companies, stores, and unknown sources. According to Mr. Jacob, the biggest obstacles the site faces are insufficient staffing and that resident outreach was the primary way to improve trash management. 

In a June 24, 2020 report, the Monitor Cleanliness Team gave 303 Vernon a B- rating and Sumner a B rating. The team has not yet evaluated Bedford-Stuyvesant Rehab.  