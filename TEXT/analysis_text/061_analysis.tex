
\textbf{Inspection and Collection Requirement}

In a compliance interview conducted on October 7, 2019, the Van Dyke I consolidation appeared to be in compliance with the inspection requirements of Paragraph 45 of the HUD agreement. The consolidation reported insufficient staff to meet the collection portion of the requirements. At the time of this interview, the site was an Alternative Work Schedule (AWS) site. 

The Supervisor Caretaker, Takeisha McDougal, reported that Van Dyke I does not have enough staff to correct observed deficiencies, but caretakers can usually complete all of their tasks in a day. NYCHA caretakers pick up trash inside the buildings three to four times a day on weekdays and one to two times a day on weekends. NYCHA caretakers also conduct ground inspections and pick up litter three to four times a day on weekdays and one to two times a day on weekends. Staff begins collecting trash between 6:00 AM --8:00 AM and ends between 4:00 PM -5:00 PM daily. 

\textbf{Removal or Storage Requirement} 

At the time of the compliance interview, the Van Dyke I Consolidation appeared to be in compliance with the storage and removal requirement of Paragraph 45 of the HUD Agreement because it does have containers in the form of exterior compactors to store waste in a manner that prevents pests on the days DSNY does not come to pick up waste. 

Van Dyke I reported at the time of the interview that DSNY comes when it calls, usually three to four times a week. The consolidation also reported that it received no bulk tickets for the removal of bulk waste because they have a Bulk Crusher. Bulk trash sits on the curbside before being picked up by the vendor. In terms of storage, residents of this consolidation have access to trash chutes and may drop their waste at 30 additional sites on the premises. After the trash is collected from the drop-off sites, it is placed in an exterior compactor. Tenants are asked by management to leave their garbage on development grounds if they choose not to use the chutes. Most tenants dispose of their trash at various undesignated areas throughout the site. Waste is placed on the curbside before off-site removal.

A single site visit on October 7, 2019 showed little to no debris throughout the site. It also showed that waste was stored in a way that prevents pests on that day. Ms. McDougal stated in the Compliance Interview that Van Dyke I did not have a pest problem.

The consolidation reported that, on average, 500-600 compactor bags (40 lbs. bags) are disposed of from Van Dyke I daily. There are three exterior compactors at this consolidation.

According to the compliance interview, there are external sources of trash and bulk waste illegally dumped at this site.  When it happens, it is from construction companies, nearby restaurants and passerby's. Ms. McDougal noted obstacles such as poor placement of waste by residents creating spillages. Waste disposal flyers are hung by the hopper doors to encourage better disposal habits. 

3. \textbf{Additional Context}

In a June 24, 2020 report, the Monitor Cleanliness Team gave Van Dyke I  an A rating. 