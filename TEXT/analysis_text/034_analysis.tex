Eastchester Gardens Analysis: 

\textbf{Inspection and Collection Requirement} 

 

The consolidation appeared to be in compliance with the inspection and collection requirements of Paragraph 45 of the HUD Agreement. Compliance could not conduct a site visit during the 2019-2020 period; however, in a survey conducted in the summer of 2020, the consolidation reported the following conditions.

The Supervisor of Grounds, Khadija Hill, reported that the Eastchester Gardens consolidation does not have enough staff to correct observed deficiencies. Caretakers cannot usually complete all of their tasks in a day. NYCHA caretakers pick up trash inside the buildings five times a day, including weekends. NYCHA caretakers also conduct ground inspections and pick up litter at least twice daily. The staff begins collecting trash at 6:00 AM and ends between 4:00  - 6:30 PM daily.

\textbf{Removal or Storage Requirement} 

  

This site is at least partially curbside, and because DSNY does not pick up from the curb every day, there is a high likelihood that this site is not in compliance as they cannot store waste in an exterior compactor on days when DSNY cannot pick up. The curbside development at this consolidation is Middletown Plaza. Based on the same summer of  2020 survey, the consolidation reported the following conditions.

 

Eastchester Gardens reported at the time of the survey that DSNY comes two times a week. Middletown Plaza's trash is removed on Mondays and Thursdays. The consolidation also stated that it received seven bulk tickets a month for the removal of bulk waste. Bulk trash sits in a yard with an exterior container before being picked up by the vendor. In terms of storage, residents of this consolidation have access to trash chutes and may drop their waste at seven additional sites on the premises. Tenants are not asked by management to leave their garbage on development grounds if they are unable or choose not to use the chutes. Most tenants dispose of their trash by leaving it in front of the buildings. Once the waste is collected from the grounds, it is stored in the exterior compactors. Regarding the curbside site, it is stored in the compactor rooms until it is ready for pick-up. More follow up is needed to determine if this meets compliance.

 

The Supervisor stated in the survey that Eastchester Gardens did not have a pest problem. The consolidation reported that, on average, 150 compactor bags (40 lbs. bags) are disposed of from Eastchester Gardens daily. There are two exterior compactors at this consolidation. One is new, but both are in good condition and do not need welding. 

According to the survey, there are external sources of trash and bulk waste illegally dumped at this site.  When it happens, it consists mainly of food, concrete, furniture, and boxes. The biggest obstacle the consolidation faces regarding waste management is being understaffed. The Supervisor also stated that the best thing Management/Operations has done is requesting another 30-yard container.

\textbf{Additional Context}  

In a June 24, 2020 report, the Monitor Cleanliness Team gave Eastchester Gardens a B-rating. Middletown Plaza has not yet been graded. 