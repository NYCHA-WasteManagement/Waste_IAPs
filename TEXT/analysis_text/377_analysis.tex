

\textbf{Inspection and Collection Requirement}

The consolidation appeared to be in compliance with the inspection and collection requirements of Paragraph 45 of the HUD agreement. Compliance could not conduct a site visit during the 2019-2020 period; however, in a survey conducted in the summer of 2020, the consolidation reported the following conditions.

The Assistant Superintendent, Kelly Story, reported that Frederick Samuel Apartments does not have enough staff to correct observed deficiencies and caretakers cannot usually complete all of their tasks in a day. NYCHA caretakers pick up trash inside the buildings multiple times a day, including weekends. NYCHA caretakers also conduct ground inspections and pick up litter multiple times a day, including weekends. Staff begins collecting trash between 8:00 AM -- 10:00 AM and ends before 4:00 PM daily. 

\textbf{Removal or Storage Requirement}

This consolidation brings waste to a different development for storage. However, the current schedule does not allow overtime for the single truck driver at the consolidation to transport waste on the weekends. For this reason, there is a high likelihood that Frederick Samuel Apartments is not in compliance as they cannot store waste in an exterior compactor on weekends when DSNY cannot pick up.

Frederick Samuel Apartments reported at the time of the survey that DSNY comes Mondays, Thursdays and Saturdays. The consolidation also reported that it does not receive bulk tickets. In terms of storage, most residents of this consolidation have access to trash chutes and may drop their waste at 28 additional sites on the premises. Most tenants dispose of their trash using the trash chutes. Once waste is collected from the grounds, waste is brought to Drew-Hamilton for storage or left on the curbside depending on the days the truck driver is scheduled to work.

The consolidation reported that, on average, 100-200 compactor bags (40 lbs. bags) are disposed of from Frederick Samuel Apartments daily. There are 32 interior compactors at eight of the buildings, three of which are currently shutdown.

According to the survey, there are external sources of trash and bulk waste illegally dumped at this site coming mostly from neighboring buildings in the form of litter. Mr. Story reports that there are staffing issues which have been exacerbated by the AWS system. He also noted that pest issues are being dealt with through extermination and sealing holes.