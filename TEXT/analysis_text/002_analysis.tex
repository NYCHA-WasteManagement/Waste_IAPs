
\textbf{Inspection and Collection Requirement}s

Williamsburg Consolidation is in compliance with the inspection and collection requirement of Paragraph 45 of the HUD Agreement. At the time of the Compliance Interview, conducted on September 26, 2019, Williamsburg was not an AWS site and not in compliance. However, it was confirmed on July 7, 2020, that Williamsburg is now an AWS site and it inspects the grounds and collects trash seven days a week. The Superintendent, Elizabeth Diaz, reported during the September Compliance Interview, that staff patrols the grounds for cleaning and maintenance, but says they have insufficient manpower to correct observed deficiencies. Ms. Diaz stated that caretakers were not usually able to complete all of their tasks in a day, especially on the weekends, due to staff shortages.

\textbf{Removal or Storage Requirement}

Williamsburg Consolidation is in compliance with the storage and removal of Paragraph 45 of the HUD Agreement because waste can be stored in manner that prevents pests. Each block has different pickup days for DSNY, but they come to the consolidation three to four times a week. An average of seven to eight bulk are tickets created each month.  Bulk trash sits in a yard with an exterior container before being picked up by the vendor. 

In terms of storage, in addition to disposing of litter into interior trash chutes, residents of this consolidation may drop their waste at 136 additional sites on the premises. Tenants are not asked by management to leave their garbage on development grounds if they choose not to use the chutes. However, on average, to dispose of  their waste, residents primarily leave their waste in front of their buildings. Waste is stored in front of the building in trash bins, to prevent pests, before being moved. Once the garbage is picked up by caretakers, it is taken to a drop site to be picked up by sanitation workers. When the trash is not able to be removed from the premises, it is stored in a way to prevent pests, (e.g. trash bins), according to the Compliance Interview. The supervisor also stated that Williamsburg did not have a pest problem and handled any issues it had by having exterminators treat the area, seal holes in basements, shut down the interior compactors, and seal burrows. There are no exterior compactors at this location.

Williamsburg Consolidation has one bulk container and has 136 interior compactor rooms that were all technically accessible at the time of the Compliance Interview but were all shut down due to safety conditions for NYCHA staff. This internal compactor issues do not arise from outdated equipment but inadequate spacing and the ram seizing onto the chassis. With the current layout at Williamsburg, there is no place for the compactor sleeve to push out due to such a short roller and causes the compactor machine to have nose blockages continuously run, thus causing the motor to burn out the machine. Additionally, if rollers were moved, it would cause a trip hazard and create unsafe conditions because they would have to climb over the rollers and compactor sleeve and risk falling and/or getting cut. Currently, the caretakers utilize a cultivator and shovel to remove garbage from the compactor thorough the nose, with the chute gate in the closed position. On a typical day, only one to two compactor/linear bags (40 lbs. bags) are removed from Williamsburg as most residents bring their garbage outside due to how small the hoppers are.  

Williamsburg does not take its waste to any other developments, nor does it allow any other developments to dispose of its waste on their property. According to the Compliance Interview, there are external sources of trash and bulk waste illegally dumped at this site, primarily from construction companies and nearby restaurants. The waste consists mostly of furniture, food, and appliances. According to Ms. Diaz, the biggest obstacles the site faces are debris from neighboring buildings flying onto the grounds and that a street sweeper is needed. Ms. Diaz also stated that meeting with residents, attending public meetings, and being in contact with sanitation were the primary ways to improve waste management. 

In a June 24, 2020 report, the Monitor Cleanliness Team gave Williamsburg a B+.