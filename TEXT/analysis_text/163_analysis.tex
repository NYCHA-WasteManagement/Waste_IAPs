
\textbf{Inspection and Collection Requirement}

The consolidation appeared to be in compliance with the inspection and collection requirements of Paragraph 45 of the HUD agreement. Compliance could not conduct a site visit during the 2019-2020 period; however, in a phone interview conducted in the summer of 2020, the consolidation reported the following conditions.

The Property Manager, Shelisa Reid, reported that Wyckoff Gardens does have enough staff to correct observed deficiencies thanks to per diem workers and caretakers can usually complete all of their tasks in a day. NYCHA caretakers pick up trash inside the buildings three to four times a day, including weekends. NYCHA caretakers also conduct ground inspections and pick up litter three to four times a day, including weekends. Staff begins collecting trash at 6:00 AM and ends at 6:30 PM daily. 

\textbf{Removal or Storage Requirement}

The consolidation appeared to be in compliance with the  removal or storage requirement of Paragraph  45 of the HUD Agreement because it has containers in the form of exterior compactors to store waste in a manner that prevents pests on the days DSNY does not come to pick up waste. Based on the same summer of  2020 survey, the consolidation reported the following conditions.

Wyckoff Gardens reported at the time of the survey that DSNY comes when called for collection. The consolidation also reported that it received four bulk tickets a month for the removal of bulk waste. Ms. Reid reports that the consolidation requests three or more additional tickets per month.  Bulk trash sits in a yard with an exterior container before being picked up by the vendor. In terms of storage, residents of this consolidation have access to trash chutes and may drop their waste at two additional sites on the premises. After the trash is collected from the drop-off sites, it is placed in a compound area. Tenants are  asked by management to leave their garbage on development grounds if they are unable or choose not to use the chutes. Most tenants dispose of their trash using the trash chutes. Once waste is collected from the grounds, waste is stored in an exterior compactor. 

The consolidation reported that, on average, fewer than 100 compactor bags (40 lbs. bags)  are disposed of from Wyckoff Gardens daily. There are three exterior compactors at this consolidation and four interior compactors, which are all functioning and accessible. There are also two 30-yard bulk containers.

According to the survey, there are external sources of trash and bulk waste illegally dumped at this site. When it happens, it is from local vendors dumping garbage and debris on the sites. Ms. Reid reports that the main obstacle facing staff is when residents dispose of waste after operational hours. She also noted that there are pest problems, but measures have been taken such as closing up windows and installing door sweeps. Exterminators also come as needed to alleviate the problem. 

3. \textbf{Additional Context} 

In a June 24, 2020 report, the Monitor Cleanliness Team gave Wyckoff Gardens an A rating. 