

\textbf{Inspection and Collection Requirement}

In a compliance interview conducted on October 1st, 2019, the consolidation appeared to be in compliance with the inspection requirements of Paragraph 45 of the HUD agreement. The consolidation reported sufficient staff to meet the collection portion of the requirements. At the time of this interview, the site was an Alternative Work Schedule (AWS) site. 

---

The Supervisor of Grounds, Dennis Peterson, reported that it does have enough staff to correct observed deficiencies and caretakers can usually complete all of their tasks in a day. NYCHA caretakers picked up trash inside the buildings 1-2 times a day, including weekends. NYCHA caretakers also conducted ground inspections and picked up litter 1-2 times a day, including weekends. Staff begins collecting trash between 8:00 AM -- 10:00 AM and ends after 5:00 PM daily. 

\textbf{Removal or Storage Requirement}

At the time of the compliance interview, Ingersoll appeared to be in compliance with the storage and removal requirement of Paragraph 45 of the HUD Agreement because it does have containers in the form of exterior compactors to store waste in a manner that prevents pests on the days DSNY does not come to pick up waste.

---

Ingersoll reported at the time of the interview that DSNY comes on Tuesdays and Fridays. The consolidation also reported that it received 7-8 bulk tickets for the removal of bulk waste. The Supervisor of Grounds stated that more were needed due to the volume of recent move-outs. Bulk trash sits in a yard with an exterior container before being picked up by the vendor. In terms of storage, residents of this consolidation do not have access to trash chutes and may not drop their waste at 45 additional sites on the premises. After the trash is collected from the drop-off sites, it is placed in the exterior compactor. Tenants are asked by management not to leave their garbage on development grounds if they choose not to use the chutes. Most tenants dispose of their trash using trash bins at the drop-off sites. Waste is stored in the exterior compactor. 

A single site visit on October 1st, 2019 showed little to no trash on the grounds upon both arrival and departure.  Trash bins with open lids and recycling bins were placed throughout the site. It also showed that waste was not stored in a way that prevents pests on that day.[SUPERVISOR] stated in the Compliance Interview that Ingersoll did not have a pest problem.

The consolidation reported that on average, 100-200 compactor bags (40 lbs. bags) are disposed of from Ingersoll daily. There are three exterior compactors at this consolidation. 

According to the Compliance interview, there are external sources of trash and bulk waste illegally dumped at this site. When it happens, it is from a storage facility and unknown sources. Mr. Peterson stated that there are less caretakers due to AWS, and it is an obstacle which is preventing his staff from keeping Ingersoll free of trash and litter. 

3. \textbf{Additional Context}

In a June 24, 2020 report, the Monitor Cleanliness Team gave Ingersoll a B+ rating. 