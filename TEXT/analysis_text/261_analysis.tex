

\textbf{Inspection and Collection Requirement}

The consolidation appeared to be in compliance with the inspection and collection requirements of Paragraph 45 of the HUD agreement. Compliance could not conduct a site visit during the 2019-2020 period; however, in a phone interview conducted in the summer of 2020, the consolidation reported the following conditions.

The Property Manager, Priti Chatterjee, reported that Unity Plaza does not have enough staff to correct observed deficiencies and caretakers cannot usually complete all of their tasks in a day. NYCHA caretakers pick up trash inside the buildings three times a day, including weekends. NYCHA caretakers also conduct ground inspections and pick up litter three times a day, including weekends. Staff begins collecting trash at 6:00 AM and ends at 6:30 PM daily. 

\textbf{Removal or Storage Requirement}

All of the developments of this consolidation have their waste collected from the curbside and because DSNY does not pick up from the curb everyday there is a high likelihood that this consolidation is not in compliance as they cannot store waste in an exterior compactor on days when DSNY cannot pick up.

Unity Plaza reported at the time of the survey that DSNY comes Mondays, Thursdays and Saturdays. The consolidation also reported that it received six to nine bulk tickets a month for the removal of bulk waste. Bulk trash sits in a yard with an exterior container before being picked up by the vendor. In terms of storage, residents of this consolidation have access to trash chutes and may drop their waste at 15 additional sites on the premises. After the trash is collected from the drop-off sites, it is placed at sanitation pick-up sites. Most tenants dispose of their trash using trash chutes. Once waste is collected from the grounds, waste is stored at sanitation pick-up sites.

The consolidation reported that, on average, fewer than 100 compactor bags (40 lbs. bags) are disposed of from Unity Plaza daily. There are two 30-yard bulk containers and 30 interior compactors. Only 20 of the interior compactors are accessible and one is shutdown due to flooding

According to the survey, there are external sources of trash and bulk waste illegally dumped at this site. When it happens, it is from neighboring stores and households dumping bulk waste. Ms. Chatterjee reports that short staffing is the main issue facing this consolidation. She states that the developments need more caretakers and bulk waste tickets to address the problems.