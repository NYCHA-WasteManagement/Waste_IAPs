Rangel Analysis: 



\textbf{Inspection and Collection Requirement}s 

 

The consolidation appeared to be in compliance with the inspection and collection requirements of Paragraph 45 of the HUD agreement. Compliance could not conduct a site visit during the 2019-2020 period; however, in a survey conducted in the summer of 2020, the consolidation reported the following conditions.



Edwin Burgos, the Supervisor of Grounds, reported that Rangel does not have enough staff to correct observed deficiencies, and caretakers cannot usually complete all of their tasks in a day. NYCHA caretakers pick up trash inside the buildings frequently throughout the day, including weekends. According to Mr. Burgos, the truck driver and helper are picking up trash consistently throughout the day.  Staff begins collecting trash around 6:00 AM and ends around 7:00 PM daily.

 

Removal or Storage Requirement 



The consolidation appeared to be in compliance with the removal or storage requirement of Paragraph  45 of the HUD Agreement. Based on the same summer of  2020 survey, the consolidation reported the following conditions.

 

The Rangel Consolidation reported at the time of the survey that DSNY comes when the compactors are full, usually two times a week. The consolidation also stated that it received five to six bulk tickets for the removal of bulk waste. Bulk trash sits in a yard with an exterior container before being picked up by the vendor. In terms of storage, residents of this consolidation have access to trash chutes and may drop their waste at nine additional sites on the premises.  Most tenants dispose of their trash by leaving it by front and rear exits of the buildings. Once the waste is collected, it is stored in exterior compactors.

 

Mr. Burgos stated in the survey that Rangel was able to store its waste in a manner that prevents pests, and it did not have a pest problem at the moment. The consolidation reported that, on average, 90 compactor bags (40 lbs. bags)  are disposed of from Patterson daily. There are three exterior compactors at this consolidation, and welding is needed on all three units. 



Patterson reports that it does not take its waste to any other developments nor accept waste from any other developments. According to the survey,  there are no external sources of trash and bulk waste illegally dumped at this site. Mr. Burgos stated that the most significant obstacles Patterson faces for waste management are tenants not using the garbage chutes in the buildings; there being too many barbeque areas on the grounds; and the tenant association handing out food in cardboard boxes to all tenants which are disposed of properly.



Additional Context 



In a June 24, 2020 report, the Monitor Cleanliness Team gave the Rangel Consolidation a C-/D+ rating.