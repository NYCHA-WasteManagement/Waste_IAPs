Building Management Associates (BX 1):

\textbf{Inspection and Collection Requirement}

The consolidation appeared to be in compliance with the inspection and collection requirements of Paragraph 45 of the HUD agreement. Compliance could not conduct a site visit during the 2019-2020 period; however, in a survey conducted in the summer of 2020, the consolidation reported the following conditions.

The Maintenance Manager, Wayne Wright, reported that Building Management Associates (BX 1) does have enough staff to correct observed deficiencies and caretakers can usually complete all of their tasks in a day. NYCHA caretakers pick up trash inside the buildings two times a day, including weekends. NYCHA caretakers also conduct ground inspections and pick up litter two times a day, including weekends. Staff begins collecting trash at 7:00 AM and ends at 3:00 PM daily. 

\textbf{Removal or Storage Requirement}

The developments at this consolidation have their waste collected from the curbside and because DSNY does not pick up from the curb everyday there is a high likelihood that this site is not in compliance as they cannot store waste in an exterior compactor on days when DSNY cannot pick up.

Building Management Associates (BX 1) reported at the time of the survey that DSNY comes three times a week. The consolidation also reported that it does not receive bulk tickets for the removal of bulk waste. In terms of storage, residents of this consolidation have access to trash chutes and may drop their waste at additional sites outside each building. Most tenants dispose of their trash in the trash chutes or at the drop sites. Once waste is collected from the grounds, waste is placed in closed garbage cans outside. 

The consolidation reported that, on average, 100-200 compactor bags (40 lbs. bags) are disposed of from Building Management Associates (BX 1) daily. There are 16 interior compactors, all of which are accessible. 

According to the survey, there are external sources of trash and bulk waste illegally dumped at this site. Mr. Wright reports that the trash chutes are too small and that recycling is often disposed down them along with the trash. Management of this consolidation has begun to educate the residents through sanitation notices. There are no pest problems at the time of the survey, but exterminators come regularly when needed.