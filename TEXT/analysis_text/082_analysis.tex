Douglass Analysis: 



\textbf{Inspection and Collection Requirement}s 

 

The consolidation appeared to be in compliance with the inspection requirements of Paragraph 45 of the HUD agreement. Compliance could not conduct a site visit during the 2019-2020 period; however, in a survey conducted in the summer of 2020, the consolidation reported the following conditions.



The Superintendent, Leroy Gibbs, reported that the Douglass consolidation does have enough staff to correct observed deficiencies, and caretakers can usually complete all of their tasks in a day. NYCHA caretakers pick up trash inside the buildings two to three times a day, including weekends. NYCHA caretakers also conduct ground inspections and pick up litter at least twice daily. Staff begins collecting trash at 6:00 AM and ends around 7:00 PM daily. 



Removal or Storage Requirement 



The consolidation appeared not to be in compliance with the removal or storage requirement of Paragraph  45 of the HUD Agreement. Based on the same summer of  2020 survey, the consolidation reported the following conditions. This site is at least partially curbside, and because DSNY does not pick up from the curb every day, there is a high likelihood that this site is not in compliance as they cannot store waste in an exterior compactor on days when DSNY cannot pick up. The curbside developments at this consolidation is 830 Amsterdam Avenue. 

 

The consolidation reported in the survey that DSNY comes three times a week. The consolidation also stated that it received twelve bulk tickets a month to remove bulk waste, but it was not enough and could use 14 tickets. Bulk trash sits in a yard with an exterior container before being picked up by the vendor. In terms of storage, residents of this consolidation have access to trash chutes and may drop their waste at 19 additional sites on the premises. After the trash is collected from the drop-off sites, it is placed in exterior compactors. Most tenants dispose of their trash using the trash chutes, but a lot of garbage is left in front of the buildings. Once the waste is collected by grounds staff, it is stored in the exterior compactors. Since 830 Amsterdam is a curbside site, it is unclear how waste is stored at this development. More follow up is needed to determine compliance.



Mr. Gibbs stated in the survey that Douglass did have a pest problem. However, even though there are always mice, rats, and roaches, it has highly trained exterminators that handle its needed immediately.



The consolidation reported that, on average,  300  - 400 compactor bags (40 lbs. bags) are disposed of from Douglass daily. There are three exterior compactors at this consolidation that were all in good condition at the time of the survey. 



Douglass reports that it does not take its waste to other developments nor take in others waste. According to the survey, there are external sources of trash and bulk waste illegally dumped at this site.  Mr. Gibbs is unsure of the source, but it usually consists of bulk debris and furniture. He also noted the most significant obstacles the consolidation faces for waste management are having the vehicle break down a lot, illegal dumping, inadequate amount of bulk tickets, and AWS.  



Additional Context 



In a June 24, 2020 report, the Monitor Cleanliness Team gave Douglas I an A rating, Douglas II an A rating, and Douglass Addition an A rating. 830 Amsterdam Avenue has yet been graded.