Rutgers Analysis: 

\textbf{Inspection and Collection Requirement} 

 

The consolidation appeared to be in compliance with the inspection and collection requirements of Paragraph 45 of the HUD Agreement. Compliance could not conduct a site visit during the 2019-2020 period; however, in a survey conducted in the summer of 2020, the consolidation reported the following conditions.

The Supervisor of Grounds, Jeffrey Ferguson, reported that the Rutgers consolidation does have enough staff to correct observed deficiencies. However, caretakers can usually complete all of their tasks in a day. NYCHA caretakers pick up trash inside the buildings two times a day, including weekends. NYCHA caretakers also conduct ground inspections and pick up litter at least twice daily. Staff begins collecting trash at 6:00 AM and ends at 6:30 PM daily. 

\textbf{Removal or Storage Requirement} 

The consolidation appeared to be in compliance with the removal or storage requirement of Paragraph  45 of the HUD Agreement because it has containers in the form of an exterior compactor to store waste in a manner that prevents pests on the days DSNY does not come to pick up the waste. Based on the same summer of  2020  survey, the consolidation reported the following conditions.

Rutgers reported at the time of the survey that DSNY comes two times a week. The consolidation also stated that it received four bulk tickets a month for the removal of bulk waste.  Bulk trash sits in a yard with an exterior container before being picked up by the vendor.  In terms of storage, residents of this consolidation have access to trash chutes and may drop their waste at five additional sites on the premises. After the trash is collected from the drop-off sites, it is placed in the exterior compactor. Most tenants dispose of their trash by using the trash chutes. Once the waste is collected from the grounds, it is stored in the exterior compactor.  

 

Mr. Ferguson stated in the survey that the consolidation did not have a big pest problem. An exterminator treats the area frequently. According to the Rutgers Rat Reduction Action Plan, the development was down to 25 burrows in February 2019 compared to 42 burrows the year prior. The consolidation reported that, on average, 60 compactor bags (40 lbs. bags)  are disposed of from Rutgers daily. There is one exterior compactor at this consolidation that was in good condition with no holes at this time of the survey.

Rutgers does not take its waste to, nor accept waste from, other developments. According to the survey, there are no external sources of trash and bulk waste illegally dumped at this site. Mr. Ferguson said the most important things Management/Operations has done to improve trash management are to request more dump tickets and call DSNY for early/extra pick-ups.

\textbf{Additional Context}  

In a June 24, 2020 report, the Monitor Cleanliness Team gave Rutgers Houses an A- rating.  