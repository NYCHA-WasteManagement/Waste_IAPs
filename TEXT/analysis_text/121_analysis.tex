

\textbf{Inspection and Collection Requirement}

In a Compliance interview conducted on October 16, 2019, the Mott Haven Consolidation appeared to be in compliance with the inspection and collection requirements of Paragraph 45 of the HUD agreement. At the time of this interview, the site was not an Alternative Work Schedule (AWS) site. A follow-up phone call to the site in the summer of 2020 confirmed that the development was checking the site and removing waste seven days a week.

The Supervisor of Grounds, Gregory Williams, reported that the Mott Haven consolidation does not have enough staff to correct observed deficiencies and caretakers cannot usually complete all of their tasks in a day. NYCHA caretakers picked up trash inside the buildings one to two times a day, including weekends. NYCHA caretakers also conducted ground inspections and picked up trash at least twice daily. Staff begins collecting trash between 8:00 AM -- 10:00 AM and ends before 4:00 PM daily.

\textbf{Removal or Storage Requirement}

At the time of the Compliance interview, the Mott Haven consolidation appeared to be in compliance with the storage and removal requirement of Paragraph 45 of the HUD Agreement because it has containers in the form of exterior compactors to store waste in a manner that prevents pests on the days DSNY does not come to pick up the waste. Despite having the proper storage equipment, reports do not show adequate pest prevention, and more follow up is needed.

Mott Haven reported at the time of the interview that DSNY comes when the compactor is full, usually three to four times a week. The consolidation also stated that it received seven to eight bulk tickets a month to remove bulk waste. Bulk trash sits in a yard with an exterior container before being picked up by the vendor. In terms of storage, residents of this consolidation have access to trash chutes and may drop their waste at eight additional sites on the premises. After the trash is collected from the drop-off sites, it is placed in exterior compactors. Tenants are asked by management to leave their garbage on development grounds if they are unable or choose not to use the chutes. Most tenants dispose of their trash using the trash chutes. Once the waste is collected from the grounds, waste is stored in exterior compactors.  

A single site visit in October showed exposed debris and litter exposed on the grounds. It also showed that waste was not stored in a way that prevents pests on that day. In the interview, Mr. Williams stated that Mott Haven did have a pest problem, but a vendor came out and is conducting follow-ups.

The consolidation reported that, on average, less than 100 compactor bags (40 lbs. bags) are disposed of from Mott Haven daily. There are two exterior compactors at this consolidation. There are holes in the compactors, and the supervisor said that the consolidation is reaching out to Arrow Steel to weld the holes.

Mott Haven reports that if necessary, it can take its trash to Patterson Houses and Mitchell Houses and may receive trash from those same developments. According to the Compliance interview, there are external sources of trash and bulk waste illegally dumped at this site. When it happens, it is from nearby sources, and it usually consists of construction materials, flyers, furniture, and appliances. According to Mr. Williams, the most significant obstacles Mott Haven faces regarding waste management are lack of manpower, tenants throwing trash out the windows, illegal dumping, and a significant homeless population.

3. \textbf{Additional Context}

In a June 24, 2020 report, the Monitor Cleanliness Team gave Mott Haven a D+/C- rating. 