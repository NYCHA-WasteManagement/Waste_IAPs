Lafayette Consolidation Analysis

\textbf{Inspection and Collection Requirement}

In a compliance interview conducted on November 15 2019, the consolidation appeared to be in compliance with the inspection requirements of Paragraph 45 of the HUD agreement. The consolidation reported sufficient staff to meet the collection portion of the requirements. At the time of this interview, the site was not an Alternative Work Schedule (AWS) site. As of July 2020, all sites work on an AWS.

The Supervisor of Grounds, Pascall Dougbey, reported that it does have enough staff to correct observed deficiencies and caretakers can usually complete all of their tasks in a day. NYCHA caretakers pick up trash inside the buildings three to four times a day, including weekends. NYCHA caretakers also conducted ground inspections and picked up litter one to two times a day, including weekends. Staff begins collecting trash between 8:00 AM -- 10:00 AM and ends before 4:00 PM daily. 

\textbf{Removal or Storage Requirement}

At the time of the compliance interview, the Lafayette appeared to be in compliance with the storage and removal requirement of Paragraph 45 of the HUD Agreement because it does have containers in the form of exterior compactors to store waste in a manner that prevents pests on the days DSNY does not come to pick up waste.

Lafayette reported at the time of the interview that DSNY comes Mondays, Fridays and Saturdays. The consolidation also reported that it received three to four bulk tickets for the  removal of bulk waste. Bulk trash sits in a yard with an exterior container before being picked up by the vendor. In terms of storage, residents of this consolidation have access to trash chutes and may drop their waste at six additional sites on the premises. After the trash is collected from the drop-off sites, it is placed in an exterior compactor. Tenants are asked by management to leave their garbage on development grounds if they choose not to use the chutes. Most tenants dispose of their trash using the drop-off sites, but not in bins. Waste is stored in a designated area with bins before off-site removal. 

A single site visit on November 15 2019 showed exposed trash on curbside or development grounds. It also showed that waste was not stored in a way that prevents pests on that day. Mr. Dougbey stated in the Compliance Interview that Lafayette did have a pest problem.

The consolidation reported that on average, 100-200 compactor bags (40 lbs. bags) are disposed of from Lafayette daily. There are five exterior compactors at this consolidation. The consolidation reported at least one compactor had a hole that required welding. Mr. Dougbey stated that he intended to reach out to contact Arrow Steel or another contractor to weld the hole.

According to the Compliance Interview, there are external sources of trash and bulk waste illegally dumped at this site.  When it happens, it is from construction companies and passerby's. Mr. Dougbey reported that the consolidation reaches out to residents through flyers and notices about proper waste disposal.

3. \textbf{Additional Context} [optional -- if applies]

As of July 2020, the Monitor Cleanliness Team has yet to give Lafayette a rating.