

\textbf{Inspection and Collection Requirement}

The consolidation appeared to be in compliance with the inspection requirements of Paragraph 45 of the HUD agreement. Compliance could not conduct a site visit during the 2019-2020 period; however, in a online questionnaire conducted in the summer of 2020, the consolidation reported the following conditions.

The Property Manager, Cecilia Almendarez, reported that caretakers ability to complete all assigned tasks is dependent on the amount of bulk trash and other issues that arise. NYCHA caretakers pick up trash inside the buildings two times a day, including weekends. NYCHA caretakers also conducted ground inspections and picked up litter two times a day, including weekends. Staff begins collecting trash at 6:30 AM and ends at 4:15 PM daily. 

\textbf{Removal or Storage Requirement}

The consolidation appeared to be in compliance with the  removal or storage requirement of Paragraph  45 of the HUD Agreement because it has containers in the form of exterior compactors to store waste in a manner that prevents pests on the days DSNY does not come. Based on the same summer of  2020 questionnaire, the consolidation reported the following conditions.

Drew Hamilton reported at the time of the interview that DSNY comes when called for collection of compactors. The consolidation also reported that it received eight to ten bulk tickets a month for the removal of bulk waste. Bulk trash sits in a yard with an exterior container before being picked up by the vendor. In terms of storage, residents of this consolidation have access to trash chutes and may drop their waste at additional sites outside each building. Most tenants dispose of their trash through chutes or containers in front of their building. Waste is stored in exterior compactors. 

The consolidation reported that, on average, fewer than 100 compactor bags (40 lbs. bags) are disposed of from Drew Hamilton daily. There are two exterior compactors at this consolidation. Drew Hamilton also has five interior compactors per building.

Drew Hamilton reports that all four Fred Samuel developments bring bulk to Drew Hamilton. According to the questionnaire, there are no external sources of trash and bulk waste illegally dumped at this site. Pest problems persist and have been addressed through work orders and interventions by IPM. Ms. Almendarez states that a major obstacle facing the consolidation is trash that is thrown out of windows which increases the amount of time needed  for inspections. She recommends installing screens on windows and instilling a sense of pride and personal accountability through fairs and education programs. Issues with the AWS schedule also prevents staff from addressing all needs. 

3. \textbf{Additional Context} [optional -- if applies]

In a June 24, 2020 report, the Monitor Cleanliness Team gave Drew Hamilton a D rating. 