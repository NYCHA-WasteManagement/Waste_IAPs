
\textbf{Inspection and Collection Requirement}

In a compliance interview conducted on November 15th, 2019, the consolidation appeared to be in compliance with the inspection requirements of Paragraph 45 of the HUD agreement. The consolidation reported sufficient staff to meet the collection portion of the requirements. At the time of this interview, the site was an Alternative Work Schedule (AWS) site. 

---

The Supervisor of Grounds, Martin Gonzalez, reported that it does not have enough staff to correct observed deficiencies and caretakers can most complete all of their tasks in a day. However, caretakers at Whitman are able to complete the main tasks for the day. NYCHA caretakers picked up trash inside the buildings more than 4 times a day, including weekends. NYCHA caretakers also conducted ground inspections and picked up litter 3-4 times a day, including weekends. Staff begins collecting trash between 6:00 AM -- 8:00 AM and ends between 4:00 PM and 5:00 PM daily.

\textbf{Removal or Storage Requirement}

At the time of the compliance interview, Whitman did not appear in compliance with the storage and removal requirement of Paragraph 45 of the HUD Agreement because it does not have storage in the form of exterior compactors to store waste in a manner that prevents pests on the days DSNY does not come to pick up waste. While the consolidation does have exterior compactors, at least one has a hole in need of repair. 

---

Whitman reported at the time of the interview that DSNY comes on Mondays, Tuesdays, and twice on Fridays. The consolidation also reported that it received 5-6 bulk tickets for the removal of bulk waste, with an additional 2-3 emergency dump tickets. Bulk trash sits in a drop-off site on the premises before being picked up by the vendor. In terms of storage, residents of this consolidation do not have access to trash chutes and may drop their waste at 9 additional sites on the premises. After the trash is collected from the drop-off sites, it is placed in the 2-yard trash container. Tenants are asked by management not to leave their garbage on development grounds if they choose not to use the chutes. Most tenants dispose of their trash using nondesignated drop-off sites. Waste is stored in an exterior compactor. 

A single site visit on November 15th, 2019 showed exposed trash on development grounds with open-lid garbage and recycling bins throughout the site. . It also showed that waste was not stored in a way that prevents pests on that day. Martin Gonzalez stated in the Compliance Interview that Whitman did have a pest problem because of improper trash disposal. Extermination is being done twice a week to combat this issue.

The consolidation reported that on average, 100-200 compactor bags (40 lbs. bags) and over 8 2-yard containers are disposed of from Whitman daily. There are three exterior compactors at this consolidation At least one of them has a hole in it that requires welding. Martin Gonzalez stated that he intended to reach out to contact Arrow Steel to weld the hole.

Whitman reports that if necessary, it can take its trash to Farragut and Marcy Houses and may receive trash from Ingersoll and Tompkins. According to the Compliance interview, there are external sources of trash and bulk waste illegally dumped at this site. When it happens, it is from construction companies and passersby.) Martin Gonzalez reports that inadequate staffing is an obstacle to keeping Whitman free of litter and trash. 

3. \textbf{Additional Context} 

In a June 24, 2020 report, the Monitor Cleanliness Team gave Whitman an A rating. 