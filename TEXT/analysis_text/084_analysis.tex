
\textbf{Inspection and Collection Requirement} 

In a compliance interview conducted on October 16, 2019, the Mill Brook Consolidation did not appear to be in compliance with the inspection requirements of Paragraph 45 of the HUD agreement. The consolidation reported insufficient staff to meet the collection portion of the requirements. At the time of this interview, the site was not an Alternative Work Schedule (AWS) site. The Property Manager could not be contacted at the time of writing this analysis to confirm whether the shift to AWS has allowed staff to check the site and remove waste daily.

The Supervisor Caretaker, Shameeka Feliciano, reported that Mill Brook does not have enough staff to correct observed deficiencies and caretakers cannot usually complete all of their tasks in a day. NYCHA caretakers picked up trash inside the buildings three to four times a day, not including weekends and thus they were not in compliance. NYCHA caretakers also conducted ground inspections and picked up litter three to four times a day, not including weekends and thus they were not in compliance. Staff begins collecting trash between 8:00 AM -- 10:00 AM and ends before 4:00 PM daily.

\textbf{Removal or Storage Requirement}

At the time of the compliance interview, the Mill Brook Consolidation appeared to be in compliance with the storage and removal requirement of Paragraph 45 of the HUD Agreement because it has containers in the form of exterior compactors to store waste in a manner that prevents pests on the days DSNY does not come to pick up waste.

Ms. Feliciano reported at the time of the interview that DSNY comes when the consolidation calls to collect filled compactors. The consolidation also reported that it received nine bulk tickets a month for the removal of bulk waste, a decrease from the ten it used to receive. Bulk trash sits in a yard with an exterior container before being picked up by the vendor. In terms of storage, residents of this consolidation have access to trash chutes and may not drop their waste at additional sites on the premises. Tenants are asked by management not to leave their garbage on development grounds if they choose not to use the chutes. Most tenants dispose of their trash in trash shoots. Waste is stored in exterior compactors.

A single site visit on October 16, 2019 showed exposed trash on the grounds. While trash and recycling bins were present, they were not all covered by lids. For this reason, the Monitor determined that waste was not stored in a manner that prevented pests on that day. Ms. Feliciano stated in the compliance interview that Mill Brook did have a pest problem.

The consolidation reported that, on average, less than 100 compactor bags (40 lbs. bags) are disposed of from Mill Brook daily. There are four exterior compactors at this consolidation.

According to the compliance interview, there are external sources of trash and bulk waste illegally dumped at this site from various sources. Ms. Feliciano reports obstacles to better waste management such as loitering by nonresidents and tenants who throw waste out of their windows.

3. \textbf{Additional Context}

In a June 24, 2020 report, the Monitor Cleanliness Team gave Mill Brook a C-/D+ rating. 