


\textbf{Inspection and Collection Requirement}

Vladeck Consolidation is in compliance with the inspection and collection requirement of  Paragraph 45 of the HUD Agreement, according to a Compliance Interview conducted on September 25, 2019. The Supervisor of Grounds, Tara Alava, reported that staff patrols the grounds for cleaning and maintenance and has sufficient manpower to correct all observed deficiencies. NYCHA caretakers pick up trash inside the buildings one to two times a day, except Sundays at the time of the interview. However, NYCHA caretakers conduct ground inspections and pick up litter one to two times per day, including weekends. Daily trash collection begins between 8:00 AM -- 10:00 AM. Ms. Alava stated that caretakers are usually able to complete all of their tasks in a day.

\textbf{Removal or Storage Requirement}

Vladeck Consolidation is not in compliance with the storage and removal requirement of Paragraph 45 of the HUD Agreement because removal is only performed 4 times a week by DSNY and, on days where waste is not removed, Vladeck Consolidation is not able to store waste in a manner that prevents access by pests. 



DSNY comes Tuesdays, Thursdays, Fridays and Saturdays. Due to the infrequency of DSNY, waste is not removed from the premises daily. An average of seven to eight bulk tickets are created each month for the removal of bulk waste. Bulk trash sits in a yard with an exterior container before being picked up by the vendor.

In terms of storage, in addition to disposing of litter into interior trash chutes, residents of this consolidation may drop their waste at eight additional sites on the premises. Tenants are asked by management to leave their garbage curbside if they choose not to use the chutes, but most tenants opt for the trash chute. Waste is stored curbside before it is moved by DSNY. Once the garbage is picked up by caretakers at drop-sites of from the interior compactor rooms, it is taken to a drop site to be picked up by sanitation workers. There are no exterior compactors at this site. Because there are no exterior compactors, when residents put their trash curbside or at the drop-off sites (rather than in chutes where the waste is then stored in interior compactors), the trash is subsequently not stored in a manner that prevents access by pests.

Vladeck Consolidation has two bulk containers and 54 interior compactor rooms that were all accessible and working at the time of the Compliance Interview. Vladeck disposes of 100 - 200 compactor bags (40 lbs. bags) daily. When the trash is not able to be removed from the premises, it is not able to be stored in a way to prevent pests (e.g. trash bins), according to the Compliance Interview. If the trash is not moved from the premises, it stays on the curb with no storage. The supervisor also stated that Vladeck has a pest problem but did not list any measures to combat this problem. According to Rat Reduction Action Plan, Vladeck was up from 95 burrows in 2017 to 175 burrows in 2019.

Vladeck does not take its waste to any other developments, nor does it allow any other developments to dispose of its waste on their property. According to the Compliance Interview, there are external sources of trash, and bulk waste illegally dumped at this site, primarily by local people. The illegally dumped waste consists mostly of furniture and appliances. According to Ms. Alava, the biggest waste obstacles that Vladeck faces are that there are no exterior compactors and that residents put their waste, especially recycling, in the wrong place.

In a June 24, 2020 report, the Monitor Cleanliness Team gave Vladeck development an A and Vladeck II development an A.