Boston Sector Analysis: 

\textbf{Inspection and Collection Requirement} 

 

The consolidation appeared to be in compliance with the inspection and collection requirements of Paragraph 45 of the HUD Agreement. Compliance could not conduct a site visit during the 2019-2020 period; however, in a survey conducted in the summer of 2020, the consolidation reported the following conditions.

The Property Manager, Renee Spooner, reported that the Boston Sector consolidation does not have enough staff to correct observed deficiencies, but caretakers can usually complete all of their tasks in a day. NYCHA caretakers pick up trash inside the buildings throughout the day, including weekends. NYCHA caretakers also conduct ground inspections and pick up litter at least twice daily. Staff begins collecting trash at 8:00 AM and stops at the end of their shifts.

\textbf{Removal or Storage Requirement}

The consolidation appeared to be in compliance with the removal or storage requirement of Paragraph  45 of the HUD Agreement because it has containers in the form of exterior compactors to store waste in a manner that prevents pests on the days DSNY does not come to pick up the waste. Based on the same summer of  2020 survey, the consolidation reported the following conditions.

  

Boston Sector reported at the time of the survey that DSNY comes when the compactor is full.  The consolidation also stated that it received seven bulk tickets a month for the removal of bulk waste but could use more. Bulk trash sits in a yard with an exterior container before being picked up by the vendor. In terms of storage, residents of this consolidation have access to trash chutes and may drop their waste at four additional sites on the premises. Most tenants dispose of their trash by leaving it at the drop-off sites or through the chutes. Once the waste is collected from the grounds, it is stored in the exterior compactors.  

Ms. Spooner stated in the survey that Boston Sector did have a small pest problem of roaches and some mice. She noted that exterminators are treating the problem. The consolidation reported that, on average, 60 compactor bags (40 lbs. bags) are disposed of from Boston Sector daily. There are two exterior compactors at this consolidation. At the time of the survey, one had required welding, and Arrow Steel had been contacted to fix the holes. 

Boston Sector does not bring its waste to other developments nor take in trash from other developments. According to the survey, there are external sources of trash and bulk waste illegally dumped at this site, but the consolidation does not know the source. Ms. Spooner stated that the biggest obstacle Boston sector regarding waste management is tenants not disposing of trash properly. She also said that the best thing Management/Operations has done to improve trash management is posting and hand-delivering proper waste procedures for trash disposal and issuing fines after warnings.

\textbf{Additional Context}  

In a June 24, 2020 report, the Monitor Cleanliness Team gave Boston Sector Houses a B rating.  