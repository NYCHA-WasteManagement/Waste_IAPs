
\textbf{Inspection and Collection Requirement}s

Adams Consolidation is in compliance with inspection and collection requirement of Paragraph 45 of the HUD Agreement, according to a Compliance Interview conducted on December 24, 2019. The Supervisor, Joyce Johnson, reported that staff patrols the grounds for cleaning and maintenance and has sufficient manpower to correct most observed deficiencies. Ms. Johnson stated that caretakers are usually able to complete all of their tasks in a day. NYCHA caretakers pick up trash inside the buildings and inspect the grounds two times a day, including weekends. Daily trash collection begins between 8:00 AM -- 10:00 AM and ends before 4:00 PM. 



\textbf{Removal or Storage Requirement}

Adams Consolidation is in compliance with the storage and removal requirement of Paragraph 45 of the HUD Agreement because it is able to store waste in a manner that prevents pests (e.g., exterior compactors).

DSNY comes Mondays and Wednesday to pick-up trash. An average of two bulk tickets is created each month for the removal of bulk waste. Bulk trash sits in a yard with an exterior container before it is picked up. 



In terms of storage, in addition to disposing of litter into interior trash chutes, residents of this consolidation may drop their waste at seven additional sites on the premises. Tenants are not asked by management to leave their garbage curbside or in front of the buildings if they choose not to use the chutes, but most tenants leave their trash in front of their buildings anyways. Waste is stored in front of each building in exposed trash bags without bins. Once the garbage is picked up by caretakers, it is taken to one of the two exterior compactors. At least one of the two exterior compactors had a hole in it that needed welding. Ms. Johnson stated that she planned on using NYCHA Welders through TSD Waste to weld the holes in the compactors.



Adams has one bulk container and seven interior compactor rooms that were all accessible and working at the time of the Compliance Interview. Adams disposes of less than 100 compactor bags (40 lbs. bags) daily. However, it should be noted that Ms. Johnson self-reported that when the trash is not able to be removed from the premises, it is not able to be stored in a way to prevent pests (e.g. trash bins). If the trash is not moved from the premises, it stays in front of the buildings with no storage or goes to exterior compactors. The supervisor also stated that Adams has a pest problem and that exterminators came to treat the area. Ms. Johnson said that the pest problem slows down after the pest treatment but picks back up again after a while.



Adams reports that, if necessary, it can take the trash to St. Mary's Houses and Forest Houses and vice versa. According to the Compliance Interview, there are external sources of trash, and bulk waste illegally dumped at this site, primarily from nearby houses. The waste consists mostly of construction material, food, and furniture and appliances from those houses. According to Ms. Johnson, the biggest obstacles the site faces are that tenants throw trash out of their windows and that there is construction at the development.



In a June 24, 2020 report, the Monitor Cleanliness Team gave Adams consolidation a B rating.