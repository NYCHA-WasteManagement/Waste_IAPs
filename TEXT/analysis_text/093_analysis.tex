 

\textbf{Inspection and Collection Requirement} 

 

The consolidation appeared to be in compliance with the inspection and collection requirements of Paragraph 45 of the HUD Agreement. Compliance could not conduct a site visit during the 2019-2020 period; however, in a survey conducted in the summer of 2020, the consolidation reported the following conditions.

The Supervisor of Grounds, Eduardo Maya, reported that the Saint Mary's Park consolidation does have enough staff to correct observed deficiencies due to AWS. Caretakers cannot usually complete all of their tasks in a day. NYCHA caretakers pick up trash inside the buildings three to five times a day, including weekends. NYCHA caretakers also conduct ground inspections and pick up litter at least twice daily. The staff begins collecting trash at 6:00 AM and ends at 3:45 PM daily.

\textbf{Removal or Storage Requirement} 

The consolidation appeared to be in compliance with the removal or storage requirement of Paragraph  45 of the HUD Agreement because it has containers in the form of exterior compactors to store waste in a manner that prevents pests on the days DSNY does not come to pick up the waste. Based on the same summer of  2020  survey, the consolidation reported the following conditions.

 

At the time of the survey, Mr. Maya reported that DSNY comes to Saint Mary's Park Houses every two to three days, and Moore Houses every four to five days. The consolidation also stated that it received ten or more bulk tickets a month to remove bulk waste. Bulk trash sits in a yard with an exterior container before being picked up by the vendor. In terms of storage, residents of this consolidation have access to trash chutes and may drop their waste at ten additional sites on the premises. Most tenants dispose of their trash by leaving it in front of the building or throwing it out the window. Once the waste is collected from the grounds, it is stored in the exterior compactors.  

 

Mr. Maya stated in the survey that consolidation did have a pest problem. The problem is being treated with chicken wire and bait stations at hot spots around the buildings and grounds. The consolidation reported that, on average, 20 -- 30 compactor bags (40 lbs. bags)  are disposed of from Saint Mary's Park daily. There are three exterior compactors at this consolidation. The exterior compactors do not have holes, but they all need constant repairs due to the ages of the machines. 

Saint Mary's Park does not take waste to, nor accept trash from, other developments. According to the survey, there are external sources of trash and bulk waste illegally dumped at this site.  When it happens, it is from the houses on 156th Street and the buildings on Jackson Avenue. Mr.  Maya said the most significant obstacle Saint Mary's Park faces regarding waste management is residents not disposing of trash where they are supposed to. They need to held responsible for littering.

\textbf{Additional Context}  

In a June 24, 2020 report, the Monitor Cleanliness Team gave Saint Mary's Park Houses an A rating, and Moore Houses a B+ rating.  

 