Gompers Analysis: 

\textbf{Inspection and Collection Requirement} 

 

The consolidation appeared to be in compliance with the inspection and collection requirements of Paragraph 45 of the HUD Agreement. Compliance could not conduct a site visit during the 2019-2020 period; however, in a survey conducted in the summer of 2020, the consolidation reported the following conditions. 

The Property Manager, Valerie Galloway, reported that the Gompers consolidation does not have enough staff to correct observed deficiencies, and caretakers cannot usually complete all of their tasks in a day. NYCHA caretakers pick up trash inside the buildings throughout the day, including weekends. Caretaker-X comes to the six sites three times a day, and Caretaker-J comes to the premises three times a day, but twice on the weekends. NYCHA caretakers also conduct ground inspections and pick up litter at least twice daily. Staff begins collecting trash around 6:15 AM and ends around 3:45 PM daily.

\textbf{Removal or Storage Requirement} 

  

This site is at least partially curbside. Because a provider does not pick up from the curb every day, there is a high likelihood that this site is not in compliance as they cannot store waste in an exterior compactor on days when a provider cannot pick up from the curbside locations. Curbside developments at this consolidation include 45 Allen Street and Hernandez Houses. Based on the same summer of  2020 survey, the consolidation reported the following conditions.

 

Gompers reported at the time of the survey that a private hauler, IESI, comes when the compactors are full. The consolidation also stated that it received five bulk tickets a month for the removal of bulk waste. Bulk trash sits in a yard with an exterior container before being picked up by the vendor.  In terms of storage, residents of this consolidation have access to trash chutes and may not drop their waste at additional sites on the premises, but tenants do drop waste off at illegal drop-off sites.  Most tenants dispose of their trash by using the trash chutes and frequently place garbage outside of the buildings. Once the waste is collected from the grounds, it is stored in the exterior compactor. If the exterior compactors are full, Gompers reaches out to surrounding developments for assistance. The waste at 45 Allen Street and Hernandez Houses is picked up by Caretaker-X (truck drivers) so it is not left at the sites for an extended amount of time. However, more follow up is needed to determine if this meets compliance or not. 

 

Ms. Galloway stated in the survey that consolidation did have a pest problem and that the Mayor's Rat Initiative is being utilized. The consolidation reported that, on average, 300 compactor bags (40 lbs. bags)  are disposed of from Gompers daily. There is one new exterior compactor at this consolidation that is in working condition with no holes. 

Gompers reports that if necessary, it can take its trash to Baruch Houses, Wald Houses, Smith Houses, Riis Houses, and LaGuardia Houses and may receive trash from those developments as well. According to the survey, there are external sources of trash and bulk waste illegally dumped at this site.  When it happens, it is from local stores dumping their bulk waste. Ms. Galloway stated the most significant obstacles Gompers faces regarding waste management are improper disposal by tenants and illegal dumping. Management/Operations

\textbf{Additional Context}  

In a June 24, 2020 report, the Monitor Cleanliness Team gave Gompers Houses a B- rating, 45 Allen Street, an A- rating, Hernandez Houses an A rating, Lower East Side I Infill an A rating, Meltzer Tower an A rating, and Seward Park Extension a D rating.  