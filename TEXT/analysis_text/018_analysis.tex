
\textbf{Inspection and Collection Requirement}s

The consolidation appeared to be in compliance with the inspection and collection requirements of Paragraph 45 of the HUD agreement. Compliance could not conduct a site visit during the 2019-2020 period; however, in a phone interview conducted in the summer of 2020, the consolidation gave the following responses. 

The Superintendent, Caroline Soriano, reported that it does not have enough staff to correct observed deficiencies, but caretakers can usually complete all of their tasks in a day. NYCHA caretakers pick up trash inside the buildings at least two times a day, including weekends. NYCHA caretakers also conduct ground inspections and pick up litter at least two times a day. Staff begins collecting trash around 6:00 AM and ends around 7:00 PM daily.

\textbf{Removal or Storage Requirement}

The consolidation appeared to be in compliance with the removal or storage requirement of Paragraph  45 of the HUD  Agreement.  Based on the same summer of 2020 phone interview, the consolidation gave the following responses.  



Riis reported at the time of the interview that DSNY comes when the compactors are full, every two to three days. The consolidation also stated that it received eight bulk tickets per month for the removal of bulk waste. It reported that eight bulk tickets are not enough, and it needs at least 11 bulk tickets per month.  Bulk trash sits in a yard with an exterior container before being picked up by the vendor. In terms of storage, residents of this consolidation have access to trash chutes and may drop their waste at additional sites on the premises. After the trash is collected from the drop-off sites, it is placed in exterior compactors. Most tenants dispose of their trash using the trash chutes, but they also frequently leave it outside the buildings.  



In the phone interview, Mrs. Soriano stated that Riis did not have a pest problem and was able to store its waste in a manner that prevents pests. According  the Rat Reduction  Action  Plan, both Riis I and Riis II  have seen a decrease in rat burrows. The consolidation reported that, on average, 200 compactor bags (40 lbs. bags) are disposed of from Riis daily. There are three new exterior compactors at this consolidation that are all working and have no holes. 

Riis reports that it does not take its waste to any other nearby developments. At the time of the phone interview, nearby Elliot Houses' exterior compactor was broken. Riis was letting Elliot Houses use one of the exterior compactors. According to the phone interview, there are no longer external sources of trash and bulk waste illegally dumped at this site since installing a fence. Ms. Soriano stated that the biggest obstacle that Riis faces is being under construction, but putting more garbage cans in front of the buildings has improved waste management.  

3. Additional Context

In a June 24, 2020 report, the Monitor Cleanliness Team gave Riis and Riis II both a B rating.