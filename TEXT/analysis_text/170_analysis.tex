

1. \textbf{Inspection and Collection Requirement}

The consolidation appeared to be in compliance with the inspection and collection requirements of Paragraph 45 of the HUD agreement. Compliance could not conduct a site visit during the 2019-2020 period; however, in a phone interview conducted in the summer of 2020, the consolidation reported the following conditions.

---

The Assistant Property Manager Supervisor, Ricardo Virella, reported that Surfside Gardens does not have enough staff to correct observed deficiencies and caretakers cannot usually complete all of their tasks in a day. NYCHA caretakers pick up trash inside the buildings 3 times a day, including weekends. NYCHA caretakers also conduct ground inspections and pick up litter 3 times a day, including weekends. Staff begins collecting trash at 6:00 AM and ends before   PM daily. 

2. \textbf{Removal or Storage Requirement}

The consolidation appeared to be in compliance with the removal or storage requirement of Paragraph 45 of the HUD Agreement because it has containers in the form of exterior compactors to store waste in a manner that prevents pests on the days DSNY does not come to pick up waste. Based on the same summer of 2020 survey, the consolidation reported the following conditions.

---

Surfside Gardens reported at the time of the interview that DSNY comes 3 times a week. The consolidation also reported that it received 6 bulk tickets a month for the removal of bulk waste. Bulk trash sits in a yard with an exterior container before being picked up by the vendor. In terms of storage, residents of this consolidation do not have access to trash chutes and may drop their waste at 12 additional sites on the premises. After the trash is collected from the drop-off sites, it is placed either curbside or in the exterior compactor. Tenants are not asked by management to leave their garbage on development grounds if they are unable or choose not to use the chutes. Most tenants dispose of their trash behind their buildings. Once waste is collected from the grounds, waste is stored in the exterior compactor.

The consolidation reported that, on average, 100-200 compactor bags (40 lbs. bags) are disposed of from Surfside Gardens daily. There are four exterior compactors at this consolidation.

According to the survey, there are external sources of trash and bulk waste illegally dumped at this site. When it happens, it is from construction. A shortage of staff is an obstacle preventing Surfside Gardens caretakers from keeping the consolidation free of litter and trash. 

3. \textbf{Additional Context}

In a May 28th, 2020 report, the Monitor Cleanliness Team gave Surfside Gardens a B- rating.