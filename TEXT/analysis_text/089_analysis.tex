Analysis 

Pink Houses appears to be in compliance with Paragraph 45 of the HUD agreement, according to a Compliance Interview conducted on December 13th, 2019. The Supervisor of Caretakers, Carmen Maduro, stated that there is a sufficient amount of manpower to handle all pest and waste problems at the site. Their work is monitored by Ms. Maduro, and she stated that there is nothing keeping them from completing their tasks in one day. Caretakers inspect the grounds for waste/pest issues, along with litter around the premises 1-2 times a day, including weekends. Residents are asked to leave their garbage in front of the building for pick-up, and most tenants do. There are also 22 drop-off sites for residents across the development. External sources of trash may include publishing, furniture/appliances, and random people. 

Caretakers pick up trash from drop-off sites to take to a 30-yard exterior compactor between 6am-8am and ends after 5pm. There are four exterior compactors on this site. 

There are 22 interior compactor rooms on this site. They are all accessible. Caretakers pick up trash from interior compactor rooms more than 4 times daily. On average, there are about 100-200 compactor bags disposed of daily according to Ms. Maduro. 

Bulk trash is stored in three bulk containers located on the site. 

DSNY picks up bulk waste between Wednesdays and Fridays, 3-4 times weekly. Pink Houses is granted 7-8 bulk tickets each month. Trash at this site is stored in a way that prevents pest access if not removed from the premises, according to Ms. Maduro.  

In a June 9th, 2020 report, the Monitor Cleanliness Team gave the Pink Houses development an A. 