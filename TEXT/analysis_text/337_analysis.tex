
\textbf{Inspection and Collection Requirement}

The consolidation appeared to be in compliance with the inspection and collection requirements of Paragraph 45 of the HUD agreement. Compliance could not conduct a site visit during the 2019-2020 period; however, in a survey conducted in the summer of 2020, the consolidation reported the following conditions.

The Property Maintenance Supervisor, Ramon Granados, reported that Lower East Side does not have enough staff to correct observed deficiencies and caretakers cannot usually complete all of their tasks in a day. NYCHA caretakers pick up trash inside the buildings once a day, including weekends. NYCHA caretakers also conduct ground inspections and pick up litter once a day, including weekends. Staff begins collecting trash at 6:00 AM and ends at 1:00 PM daily. 

\textbf{Removal or Storage Requirement}

Four of the five developments at this consolidation have waste collected from the curbside and because DSNY does not pick up from the curb everyday there is a high likelihood that this site is not in compliance as they cannot store waste in an exterior compactor on days when DSNY cannot pick up. Campos Plaza II is the only development that does not have curbside collection. This development brings its waste to Riis Houses, which is part of a different consolidation.

Lower East Side reported at the time of the survey that DSNY comes two days a week. The consolidation also reported that it received three bulk tickets a month for the removal of bulk waste.  Bulk trash sits in the parking area with an exterior container before being picked up by the vendor. In terms of storage, residents of this consolidation have access to trash chutes and may drop their waste at  additional sites on the premises. Tenants at each development dispose of trash in different ways because of the differences between each campus. Once waste is collected from the grounds, waste is either brought to the curbside for collection or brought to Riis Houses depending on the development. 

The consolidation reported that, on average, fewer than 100 compactor bags (40 lbs. bags)  are disposed of from Lower East Side daily. There is one 30-yard bulk container and 15 interior compactors, all of which are accessible. 

According to the survey, there are external sources of trash and bulk waste illegally dumped at this site. When it happens, it is from construction companies. Mr. Granados reports needing more staff to fulfill all duties. He also noted weather related issues when transporting waste to another development. 

3. \textbf{Additional Context} 

In a June 24, 2020 report, the Monitor Cleanliness Team gave Lower East Side a B rating. 