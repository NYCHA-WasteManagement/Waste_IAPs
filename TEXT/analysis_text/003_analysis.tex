Harlem River Analysis: 

\textbf{Inspection and Collection Requirement} 

 

The consolidation appeared to be in compliance with the inspection and collection requirements of Paragraph 45 of the HUD Agreement. Compliance could not conduct a site visit during the 2019-2020 period; however, in a survey conducted in the summer of 2020, the consolidation reported the following conditions.

The Supervisor of Caretakers, Chad Lewin, reported that the Harlem River consolidation does have enough staff to correct observed deficiencies. However, caretakers can usually complete all of their tasks in a day. NYCHA caretakers pick up trash inside the buildings three times a day, including weekends. NYCHA caretakers also conduct ground inspections and pick up litter at least twice daily. The staff begins collecting trash at 6:00 AM.

\textbf{Removal or Storage Requirement} 

The consolidation appeared to be in compliance with the removal or storage requirement of Paragraph  45 of the HUD Agreement because it has containers in the form of exterior compactors to store waste in a manner that prevents pests on the days DSNY does not come to pick up the waste. Based on the same summer of  2020  survey, the consolidation reported the following conditions.

  

Harlem River reported at the time of the survey that DSNY comes three times a week. The consolidation also stated that it received eight bulk tickets a month for the removal of bulk waste. Bulk trash sits in a yard with an exterior container before being picked up by the vendor. In terms of storage, residents of this consolidation have access to trash chutes and may drop their waste at ten additional sites on the premises. After the trash is collected from drop-off sites, it is collected by the trucks and placed in the exterior compactors.  Most tenants dispose of their trash by placing it at the drop-sites. Once the waste is collected from the grounds, it is stored in the exterior compactors.  

 

Mr. Lewin stated in the survey that consolidation did have a pest problem. The consolidation reported that, on average, 12 compactor bags (40 lbs. bags)  are disposed of from Harlem River daily. There are three exterior compactors at this consolidation that are new with no holes and in good condition.

Harlem River does not take its waste to any other developments nor accept waste from developments. According to the survey, there are external sources of trash and bulk waste illegally dumped at this site. Mr. Lewin said the most significant obstacle Harlem River faces regarding waste management is being short-staffed.

\textbf{Additional Context}

  

In a June 24, 2020 report, the Monitor Cleanliness Team gave Harlem River and Harlem River II a B/B+ rating, and Audubon Houses an A/B+ rating. Bethune Gardens, Marshall Plaza, and Washington Heights Rehab Phase III (Harlem River) have not yet been graded.  

 