

The Reid Apartments consolidation is a diverse group of developments scattered around the Flatbush, East Flatbush and Brownsville sections of Brooklyn. 

104-14 Tapscott Street is a federally-funded, turnkey development acquired by NYCHA in 1972. Located in Brownsville, this single building has 30 units and an official population of 68. It spans across 0.23 acres. 

Fenimore-Lefferts is a series of federally-funded homes acquired by NYCHA in 1969 both on Fenimore Street and Lefferts Avenue in the Flatbush section of Brooklyn. The 0.77-acre development consists of 18  two-story buildings and has 36 units. The official population is 93. 

Lenox Rd-Rockaway Parkway is 3 four-story federally-funded residential buildings located on the corner of Lenox Rd \& Rockaway Pkwy in the East Flatbush section of Brooklyn.  This 0.55-acre development has 74 units and an official population of 151.

Ralph Avenue Rehab is a series of 5 four-story federally-funded residential buildings acquired by NYCHA in 1986 located in the Brownsville section of Brooklyn. This 1.62-acre development has 118-units and an official population of 268. Ralph Ave Rehab also has a playground. 

Reid Apartments is a single 20-story residential building for seniors located in the Brownsville section of Brooklyn. Built in 1969, this federally-funded conventional development has 230 units and an official population of 233. Spanning across 1.58 acres, Reid Apartments also have green space. 

Rutland Towers is a federally-funded conventional development built in 1977 in the East Flatbush section of Brooklyn. This is a single six-story, 61-unit building with an official population of 98. Rutland Towers also has a playground. 

Sutter Ave-Union St is a series of 3 four to six-story residential buildings located in the Brownsville section of Brooklyn. Acquired by NYCHA in 1995, this 0.86-acre development has 100 units and an official population of 249. It also has a playground. 

Tapscott Street Rehab is a series of 8 four-story residential buildings acquired by NYCHA in 1986. This 1.49-acre development has 155 units and an official population of 327. Tapscott Street also has a playground. 