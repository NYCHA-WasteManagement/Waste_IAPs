Park Rock Consolidation Overview

Park Rock Consolidation consists of seven developments in the Crown Heights, Brownsville and East New York neighborhoods of Brooklyn. The Crown Heights development is bordered by Buffalo and Ralph Avenues, Bergen Street and St. John's Place. This turnkey development came out of a federal program and was completed in 1986. Crown Heights consists of eight, four-story buildings on a 1.18 acre site. There are 120 households with an official population of 238 residents. 

The Howard Avenue development is bordered by East New York and Sutter Avenues, and Grafton and Tapscott Streets. This turnkey development came out of a federal program and completed construction in 1988. Howard Avenue consists of five, three-story buildings on a 3.05 acre site. There are 146 households with an official population of 346 residents.

The Howard Avenue-Park Place development is boarded by Howard Avenue, Sterling and St. John's Places, and Eastern Parkway. This turnkey development came out of a federal program and completed construction in 1994. The development consists of eight, three-story buildings on a 4.54 acre site. There are 154 households with an official population of 429 residents. 

The Ocean Hill-Brownsville development is bordered by Ralph, Atlantic and Saratoga Avenues, and Dean Street. This conventional development came out of a federal program and completed construction in 1986. Ocean Hill-Brownsville consists of five, four-story buildings on a 5.56 acre site. There are 120 households with an official population of 291 residents.

Park Rock Rehab is bordered by Belmont and Sutter Avenues, and Jerome and Barbey Streets. This turnkey development came out of a federal program and completed rehabilitation in 1986. Park Rock Rehab consists of nine, four-story buildings on a 1.24 acre site. There are 129 households with an official population of 297 residents. 

The final two developments are part of the Sterling Place Rehabs, a federal turnkey project. Saint Johns-Sterling is bordered by St. Johns, Park and Sterling Places and Buffalo, Utica and Ralph Avenues. Completed in 1991, the development consists of five, four story buildings on a 1.13 acre site. There are 82 households with an official population of 268 residents. 

The Sterling-Buffalo development is bordered by St. Johns, Park and Sterling Places and Buffalo, Utica and Ralph Avenues. Completed in 1991, the development consists of seven, four story buildings on a 1.12 acre site. There are 125 households with an official population of 329 residents.

Waste is placed on the curbside for collection by DSNY for all developments.