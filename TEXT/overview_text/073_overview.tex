Located in the Bedford-Stuyvesant area of Brooklyn, the Sumner Consolidation is composed of three developments: Sumner Houses, 303 Vernon Avenue, and Bedford-Stuyvesant Rehab. All developments in the consolidation receive federal funding. The staff are deployed from the management office located in a Sumner building.

 Built in 1958, Sumner is a 22-acre development with 13 buildings ranging from

7-12 floors, which house 1,088 families. The development also features a basketball court, green areas and parking lots. Sumner's modern ``towers-in-the-park'' model comes with a trash chute system that is designed to streamline household trash collection. While this system is made to be convenient for residents, it prioritizes trash over recycling by only having a single small chute. a feature, along with larger apartments and large spaces between buildings, was designed to rid urban areas of longtime problems concerning health and welfare.

Like Sumner, 303 Vernon is a conventional development with green spaces and a parking lot. It is a 24-story building built several years later in the summer of 1967. While it is a standalone building, 303 Vernon also has a chute system for convenient trash disposal.

Bedford-Stuyvesant Rehab is a turnkey development acquired by NYCHA in 1983 comprised of five buildings between 4-6 stories that were not constructed originally for public housing. These pre-war tenement buildings, built in the early 1930s, do not come with a chute system like the other developments in Sumner Consolidation. Residents leave their waste at the curb for pickup by NYCHA caretakers. This can promote more recycling; however, it is more labor intensive for residents.