The Latimer Gardens Consolidation consists of four developments in the Flushing and College Point neighborhoods of Queens. Bland Houses, named for composer James Alan Bland, is bordered by the Long Island Rail Road, Roosevelt Avenue and Prince and Lawrence Streets. This conventional development came out of a federal program and finished construction in 1952. Bland Houses consists of five, ten-story buildings on a 6.19 acre site. There are 399 households with an official population of 854 households. Waste is brought to the curbside for collection by DSNY.\par \vspace{.7\baselineskip}Latimer Gardens, named for prominent inventor Lewis H. Latimer, is bordered by 34th and 35th Avenues, Linden Place, and Leavitt and 137th Streets. This conventional development came out of a federal program and finished construction in 1970. Latimer Gardens consists of four, 10-story buildings on a 3.84 acre site. There are 419 households with an official population of 781 residents. There are multiple recycling bins and one exterior compactor for waste storage.\par \vspace{.7\baselineskip}The Leavitt Street-34th Avenue development is bordered by Leavitt and Union Streets, 34th Avenue and 34th Road. This turnkey development came out of a federal program and finished construction in 1974. The development consists of a single six-story building on a 0.46 acre site. Designed exclusively for seniors, the building has 82 households with an official population of 103 residents. Waste is brought to the exterior compactor at Latimer Gardens for storage.\par \vspace{.7\baselineskip}Rehab Program (College Point) is bordered by 125th and 126th Streets and 22nd Avenue. This conventional development came out of a federal program and completed rehabilitation in 1964. College Point consists of a single, one-story building on a 0.34 acre site. Designed exclusively for seniors, the building has 13 households and an official population of 13 residents. Waste is brought to the exterior compactor at Latimer Gardens for storage.