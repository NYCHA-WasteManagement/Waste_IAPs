Analysis\par \vspace{.7\baselineskip}Inspection \& Collection Requirement\par \vspace{.7\baselineskip}Glenwood Houses appears to be in compliance with the inspection and collection requirement of paragraph 45 of the HUD agreement, according to a Compliance Interview conducted on December 6th, 2019. The staff inspects the premises for trash and pest issues 1-2 times a day and once on weekends, according to Ronda Porter, Supervisor of Grounds. However, the supervisor reports that they had insufficient manpower to correct most observed deficiencies, and the staff isn't able to complete all tasks in one day. This development is not an AWS site.\par \vspace{.7\baselineskip}Trash collection begins between 8-10am and ends between 4-5pm, and is also done 1-2 times daily. Recycling bins are placed throughout the development. There are six drop-off sites located throughout the site, but some buildings are too far from some buildings. As a result, residents place household trash at nondesignated drop-off sites with no bins, usually at the front of their building even though they are asked not to. Trash is then taken to the exterior compactor for removal or left at the drop-off site if not removed from the premises. \par \vspace{.7\baselineskip}Obstacles facing Glenwood's caretakers include their team's staff shortage and residents not disposing of their trash as requested. Letters have been sent to tenants regarding proper trash disposal.\par \vspace{.7\baselineskip}Removal and Storage Requirement\par \vspace{.7\baselineskip}Glenwood does not appear to be in compliance with the removal and storage requirement of paragraph 45 of the HUD agreement. Trash is not stored in a way that prevents access from pests. Trash is either taken to the EZ pack for storage or stays at the drop-off site if not removed from the premises. However, these nondesignated drop-off sites have no bins. There are two bulk containers. Bulk trash sits in the yard with the exterior compactors. There are three exterior compactors located on this site. There are 40 interior compactor rooms, two of which are inaccessible due to pests. There is a pest infestation problem at Glenwood Houses, and extermination is done once a month to tackle this issue. 100-200 compactor bags are disposed of each day at this site.