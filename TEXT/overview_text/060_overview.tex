Baruch Consolidation Overview

The Baruch Consolidation, named after Bernard M. Baruch, is located in the Lower East Side neighborhood of Manhattan. The consolidation consists of two developments that are adjacent to one another. The Baruch Houses are bordered by E Houston Street, Delancey Street, FDR Drive and Columbia Street. This conventional development was federally funded and finished construction in 1959. There are 17 buildings within the development consisting of 7, 13 and 14-story residential buildings. Baruch Houses is a 27.46-acre NYCHA development, the largest in Manhattan. The development has 2,154 households with an official population of 4,724. There are multiple recycling bins on site as well as four exterior compactors for waste storage.

The Baruch Consolidation Addition is a singular building within the same superblock as the Baruch Houses. It is also a conventional, federally funded site. Finishing construction in 1977, the 1.08 acre site was designed as a senior only building. There 192 households with an official population of 241. Waste is stored in exterior compactors on the adjacent development.