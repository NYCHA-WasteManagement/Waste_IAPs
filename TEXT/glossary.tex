\textit{Italicized words indicate a term that will be described later in the glossary}

\textbf{Bulk Waste Container --} A 30-cubic-yard bin, typically uncovered, used to hold non-recyclable bulk waste such as furniture, wood, etc.

\textbf{Cardboard baler --} A machine used to automatically compress loose cardboard into bundles as an alternative to manually breaking down boxes and tying or bagging them for curbside collection

\textbf{Compactor Bags --} 40-lb bags of compacted trash from \textit{interior compactors}

\textbf{Consolidation --} Name given to one or many of developments that are managed by the same location or management office and are assigned a unique 3-digit numeric ID in the Tenant Data System (TDS), e.g., the Sumner Consolidation TDS\# 073

\textbf{Containerization --} Storage of waste that is pest-resistant

\textbf{Development --} Individual NYCHA properties that are assigned an individual development TDS number, e.g., 303 Vernon Ave TDS\# 156

\textbf{Drop site --} Also known as secondary waste areas, these are designated areas where waste may be placed by residents for collection and disposal by staff; the site may accommodate both trash and recycling bins. Central office also uses the term ``staff drop sites'' to describe areas where waste is placed before being brought to the \textit{waste yard}

\textbf{DSNY --} City of New York Department of Sanitation

\textbf{E-waste --} Electronics such as TVs and computers that must be discarded through the manufacturer, a recycling location, or one of DSNY's special recycling programs

\textbf{Exterior Compactor --} Often referred to on our developments as an \textit{EZ-pack}. Similar to an \textit{interior compactor}, this machine compacts and containerizes waste into 30-cubic-yard containers before removal by DSNY

\textbf{EZ-Pack --} Another term used at the developments for exterior compactor. The term is also used by DSNY and central office to describe waste containers of various sizes that are designed to be dumped directly into a DSNY truck

\textbf{Hopper Doors --} Doors to trash chutes, traditional hopper doors' areas are $1/3$ the chute's area. Traditional hopper doors comfortably can accommodate a 13-gallon trash bag. Enlarged hopper doors are $2/3$ the chute's area and can comfortably accommodate a 30-gallon trash bag

\textbf{Interior Compactor --} A machine at the base of a \textit{trash chute} that uses a ram to compress waste material and reduce its total volume; mostly located in the basement of developments

\textbf{Mattress Containers --} Locked shipping containers serviced under a mattress recycling contract where staff at participating developments bring mattresses

\textbf{NYCHA 2.0 Waste Management Plan --} NYCHA's comprehensive plan created in 2019 designed to make NYCHA buildings and grounds visibly clean and free of pests

\textbf{Paragraph 45 --} Part of the agreement between HUD, SDNY, and NYCHA pertaining to waste management through inspection, collection, and containerization. The text of the paragraph is as follows:
\begin{quotation}
\small\textit{Within six months of the Effective Date, NYCHA shall, no less than once every 24 hours, inspect the grounds and commons areas of each building for cleaning and maintenance needs, including pests and trash, and correct such conditions. In particular, NYCHA shall ensure that trash on the grounds or common areas of each NYCHA building is collected and either removed from the premises or stored in a manner that prevents access from pests at least once every 24 hours.}
\end{quotation}

\textbf{Recyclable --} All material that is separated and collected for recycling

\textbf{Textiles --} Unwanted clothing, towels, blankets, curtains, shoes, handbags, belts, and other textiles and apparel that can be collected for re-use or recycling

\textbf{Trash --} All material not separated for recycling that will be transported to landfills or incinerators for disposal

\textbf{Trash Chute --} A vertical shaft inside a building used for transferring trash by gravity to the interior compactor at the bottom

\textbf{Types of DSNY Disposal:}
\begin{itemize}
\setlength{\itemsep}{0pt}%
    \setlength{\parsep}{0pt}%
    \setlength{\parskip}{0pt}%
\item \textbf{Curbside --} Material is moved from building compactors and grounds within the development by staff to a secondary storage area until it is placed at DSNY collection locations on sidewalks adjacent to or along the perimeter of the development
\item \textbf{Shared --} Material is moved (aka shared) from one development without assets to ensure containerization to another development that has an exterior compactor
\item \textbf{Exterior Compactor Containerization --} Material is moved from building compactors and grounds within the development and potentially from shared developments to a waste storage area that contains an exterior compactor
\end{itemize}

\textbf{Waste --} All discarded material including both trash and recyclables

\textbf{Waste Yard --} Centralized waste facility for containerized collection including equipment such as exterior compactors, bulk crushers and bulk waste containers. Waste yards may also include storage and equipment for recyclables

