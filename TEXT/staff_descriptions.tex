\\\bigskip At the consolidation level, responsibility for waste management falls on caretakers as well as those who manage them -- Supervisor of Caretakers (SOC), and Supervisor of Groundskeepers (SOG). The duties of specific caretaker roles are outlined below:
\begin{itemize}[noitemsep]
\item Caretaker X: Authorized to drive vehicles necessary for large-scale movement of waste, such as skid-steer loaders used to manipulate 30-yard containers. These caretakers may also conduct a range of duties otherwise assigned to caretakers in the G or J titles. 
\item Caretaker G: Primarily responsible for groundskeeping tasks, such as cutting lawns, trimming trees and hedges, and tending to beds. 
\item Caretaker J: Conduct a range of janitorial tasks, including removing garbage from compactor rooms, servicing equipment such as compactors, and cleaning indoor and outdoor spaces of debris. These caretakers may also conduct groundskeeping work, including cutting lawns and trimming hedges.
\end{itemize}
Moving forward, the Department of Waste Management and Pest Control will oversee NYCHA's progress in these areas, manage inspections to assess development cleanliness, and develop new initiatives.