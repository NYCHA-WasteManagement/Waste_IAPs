\chapter{\textcolor{darkBlue}{Preface}}

    \section{Letter from the Chair}\label{sec:Section1}
    \clearpage
    {\fontfamily{phv}\selectfont
    \section{What is an Individual Action Plan?}

The Individual Action Plans (IAPs) were developed through a collaboration between Capital Projects, Strategic Planning, Operations, and the Federal Monitor during the fall of 2019. 

The purpose of the IAPs is intended to improve waste management at NYCHA, as agreed to in Paragraph 45 in the HUD Agreement. Paragraph 45 refers to Exhibit B, Section D, and states:

\begin{quote}

Within six months of the Effective Date, NYCHA shall, no less than once every 24 hours, inspect the grounds and common areas of each building for cleaning and maintenance needs, including pests and trash, and correct such conditions. In addition, NYCHA shall ensure that trash on the grounds or common areas of each NYCHA buildings is collected and either removed from the premises or stored in a manner that prevents access by pests at least once every 24 hours.

\end{quote}

The  IAPs serve as a stepping-stone toward project-based property management that addresses the unique needs of each consolidation. They will also be used to create a planning and oversight tool, which can be the stepping-stone upon which tailored and holistic capital and operational solutions can be created for waste management at each of the consolidations.

Everyone deserves a home they can be proud of, and waste management directly affects our residents' health and safety as well as their quality of life. Proper waste management is also vital to protecting our waterways and overall environment. Unfortunately, insufficient staffing and equipment have impacted waste management at NYCHA for years. 

The goals of the IAPs are to: 

\begin{enumerate}[noitemsep]

\item Empower development staff with the resources they need to coordinate and communicate with the central offices;

\item Clarify the complex system of waste management at our consolidations; and

\item Guide adaptation to changing assets and flows at each consolidation to make them cleaner and safer for residents and employees alike. 

\end{enumerate}

We strive to create the most transparent and accurate IAP as possible, but there is room for error, and we cannot guarantee that all information is correct at this point in the process. The IAPs are living, breathing documents that will be updated annually with the latest information and data.  They will be distributed to each consolidation online in a digital format. 

Please feel free to contact us with any questions, concerns, or comments, including necessary updates to the IAPs, at wasteIAP@nycha.nyc.gov.